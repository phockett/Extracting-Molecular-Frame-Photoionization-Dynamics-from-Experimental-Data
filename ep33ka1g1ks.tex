
% Literature review
% [May be integrated above?]
% Recent developments
% Outlook
% Resources

\section{Summary \& Outlook \label{sec:summary-outlook}}

From this topical review, it is hoped that the reader has gained a solid grounding in photoionization physics, and general reconstruction methods, for both MF observable reconstruction and full matrix element retrieval (``complete" photoionization or quantum tomography treatments), and that a suitable toolkit and platform for interested researchers has been developed. 

Over the last decade or two, such experiments, as well as theoretical treatments and unified data-analysis techniques have developed significantly, and become more accessible to non-specialists. It is our hope 
%that this trend continues, and 
that the new platform and tools discussed herein (Sect. \ref{sec:resources}) will prove useful, and help photoionization (and related) studies involving MF and full matrix element retrieval to (continue to) become more routine and successful. Whilst significant new work has been presented herein, development work is ongoing, with aims to make the code-base more robust and user-friendly, test new cases - particularly for more complex molecular systems, and dynamical cases - and implement new features, such as metrics for information content for a given basis set that will aid experimental planning, and further density-matrix based analysis methods. As well as publication in the literature, new developments and results will also be made available via the open-source platform, and it is hoped that other researchers will also contribute to grow these efforts over time.

Meanwhile, experimental methodologies continue to improve and grow in sophistication, allowing more routes, and often more direct routes, to high information-content observables, whether in the lab or molecular frame, in a range of interactions and scenarios, including for larger molecules and probing dynamical effects. 

% [Something on recent pubs/demonstrations here?]

Recent work in this vein from the current authors and coworkers includes (atomic) matrix element retrieval from photoelectron imaging measurements with polarization multiplexing (via shaped laser pulses) \cite{hockett2014CompletePhotoionizationExperiments, hockett2015MaximuminformationPhotoelectronMetrology,hockett2015CoherentControlPhotoelectron, hockett2015CompletePhotoionizationExperiments}, (hyperfine) quantum beat spectroscopy from time-resolved photoelectron imaging experiments \cite{forbes2018QuantumbeatPhotoelectronimagingSpectroscopy}, and quantum tomographic determination of (LF) density matrices including electronic dynamics - theory \cite{gregory2022LaboratoryFrameDensity} and application to $NH_3$ (manuscript in preparation). These types of investigations indicate the potential for further general developments, and the utility of the MF retrieval and reconstruction techniques discussed herein as a foundation for a broader range of problems in molecular photoionization and dynamics, ultimately building to a class of molecular quantum state retrieval methods from photoelectron measurements \cite{hockett2018QMP1, hockett2018QMP2}. 




