\section{Framing the problem}

\subsection{Molecular properties}
A very general problem in molecular physics is the determination of intrinsic molecular properties from experimental measurements, and comparing such measurements with theoretical predictions. The key difficulty is, usually, the averaging or integration over unobserved degrees-of-freedom (DOF) in the measurements. From a spectrocopic perspective, one usually considers the DOFs in terms of a Born-Oppenheimer separation of the wavefunction, i.e. in terms of electronic, vibrational and rotational DOFs, which may approximately correlate with the choice of spectroscopic methodology applied. One can also consider this issue in terms of a general quantum mechnical language of averaging over eigenstates, or wavefunctions, or wavepackets (more appropriate for dynamical systems), or density matrices. 

In favourable cases, careful experimental design can obviate this issue via, for instance, state-selection prior to measurement; however, in most cases this is not feasible due to the inherent complexity of the system. The problem becomes significantly worse for larger systems as the number of DOFs (hence density of states) increases, and/or if the DOFs are continuous, rather than quantised, properties, and/or if many states are populated.