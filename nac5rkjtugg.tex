\section{Framing the problem}

\subsection{Molecular properties}
A very general problem in molecular physics is the determination of intrinsic molecular properties from experimental measurements, and comparing such measurements with theoretical predictions. The key difficulty is, usually, the averaging or integration over unobserved degrees-of-freedom (DOFs) in the measurements. The exact nature of the DOFs will, naturally, be system and problem dependent, as will the coupling strength of the unobserved DOFs to the system properties of interest. From a spectroscopic perspective, one usually considers the DOFs in terms of a Born-Oppenheimer separation of the full molecular wavefunction, i.e. in terms of electronic, vibrational and rotational DOFs, which may approximately correlate with the choice of spectroscopic methodology applied, and often provide a good approximation to the fully-coupled s. One can also consider this issue in terms of a general quantum mechanical language of eigenstates, wavefunctions, wavepackets (more appropriate for dynamical systems), or density matrices. Framing the problem in this language perhaps highlights the generality of the molecular case (as a rather general many-body quantum mechanical matter system, to which certain separations are often applicable) but, regardless of the language, the problem remains that many-body systems can be rather complicated!

In favourable cases, careful experimental design can obviate the issue of DOF averaging in specific cases via, for instance, state-selection of the system prior to measurement to reduce the complexity of the problem at hand; however, in most cases this is not feasible due to the inherent complexity of the system, and/or coupling (non-separability) of states, and/or experimental issues/limitations. The problem becomes significantly worse for larger systems as the number of DOFs (hence density of states) increases, and/or if the DOFs are continuous, rather than quantised, properties, and/or if many states are populated. In general, then, this is a problem which must ideally be address at a high level, by a combination of experiment and theory.