\section{Framing the problem\label{sec:Framing}}

\subsection{Molecular properties}
A very general problem in molecular physics is the determination of intrinsic molecular properties from experimental measurements, and comparing such measurements with theoretical predictions. The key difficulty is, usually, the averaging or integration over unobserved degrees-of-freedom (DOFs) in the measurements. The exact nature of the DOFs will, naturally, be system and problem dependent, as will the coupling strength of the unobserved DOFs to the system properties of interest. From a spectroscopic perspective, one usually considers the DOFs in terms of a Born-Oppenheimer separation of the full molecular wavefunction, i.e. in terms of electronic, vibrational and rotational DOFs, which may approximately correlate with the choice of spectroscopic methodology applied, and often provide a good approximation to the fully-coupled system. One can also consider this issue in terms of a general quantum mechanical language of eigenstates, wavefunctions, wavepackets (more appropriate for dynamical systems), density matrices and so forth. Framing the problem in this language perhaps highlights the generality of the molecular case as a many-body quantum mechanical matter system, to which certain formal DOF separations are often applicable - but, regardless of the language, the problem remains that many-body systems can be rather complicated.

In favourable cases, careful experimental design can obviate the issue of DOF averaging in specific cases via, for instance, state-selection of the system prior to measurement to reduce the complexity of the problem at hand (typical in frequency-resolved spectroscopy, e.g. \cite{bunkerMolSymm, herzberg1945molecular, hollasHighRes}) or preparing a specific ``zeroth-order" wavepacket (typical in time-resolved spectroscopy, e.g. \cite{Tannor2007,Stolow2008,wu2011TimeresolvedPhotoelectronSpectroscopy}); however, in many (most?) cases this is not feasible due to the inherent complexity of the system, and/or coupling (non-separability) of states, and/or experimental issues/limitations. The problem becomes significantly worse for larger systems as the number of DOFs (hence density of states) increases, and/or if the DOFs are continuous, rather than quantised, properties, and/or if many states are populated. In general, then, this is a problem which must ideally be addressed at a high level, by a combination of experiment and theory. Happily, both are increasingly possible, and becoming more routine, as technology improves; of particular relevance to the photoionization case at hand is the advent of photoelectron imaging, advances in short laser pulses and control, and the ongoing march of available computational power and software.

In terms of photoionization studies, the aims can be viewed both in terms of control and in terms of measurement and reconstruction. For instance, a basic photoelectron imaging experiment may seek to measure photoelectron energy and lab-frame (LF) angular distributions from a given system. A more sophisticated experiment may seek to control these observables in some way, e.g. via state-preparation or ionizing pulse polarization, or may seek to use these measurements as a sensitive probe of some other DOF of interest, e.g. vibrational motion. A yet more sophisticated methodology may aim to obtain a ``complete" quantum mechanical description of the photoionization process from a set of measurements, which may be an end in itself, or serve as a more sensitive probe of DOFs of interest. (For further general discussion along these lines, see for example refs. \cite{hockett2018QMP1, kleinpoppen2013perfect, Reid2012, Stolow2008}.)