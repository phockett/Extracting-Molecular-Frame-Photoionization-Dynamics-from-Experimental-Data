\subsubsection{Retrieval \& reconstruction techniques}

% Very generally, the problem of retrieval can be considered in terms of completeness, i.e. the degree to which the full quantum state of the system is know. In photoionization, ``complete" experiments, in which the full system wavefunction (matrix elements), denote the class of measurements in which full matrix element retrieval is the goal, as distinct from measurements which aim at specific observables
% This is currently briefly in Sect. 3.1, but probably should be expanded!

Following the tensor notation presented above, a ``complete" photoionization experiment can be characterized as recovery of the matrix elements $I^{\zeta}(\epsilon)$ from the experimental measurements or, equivalently, the density matrix $\mathbf{\rho}^{\zeta\zeta'}$. (For further discussion, see refs. \cite{Reid2003,kleinpoppen2013perfect,hockett2018QMP1}.) This may be possible provided the channel functions are known, and the information content of the measurements is sufficient. (Note here that the matrix elements are assumed to be time-independent, although that may not be the case for the most complicated examples including vibronic dynamics \cite{hockett2018QMP2}.) 

Of particular import here is the phase-sensitive nature of the observables, which is required in order to obtain partial wave phase information - hence PADs can also be considered as angular interferograms, and reconstruction can be considered conceptually similar to other phase-retrieval problems, e.g. optical field recovery with techniques such as FROG \cite{trebino2000FrequencyResolvedOpticalGating}, and general quantum tomography \cite{MauroDAriano2003}. The nature of the problem suggests that a fitting approach can work, in general, which can be expressed in the standard way as a least-squares minimization problem:

\begin{equation}
\chi^{2}(\mathbb{I}^{\zeta\zeta'})=\sum_{u}\left[\beta^{u}_{L,M}(\epsilon,t;\mathbb{I}^{\zeta\zeta'})-\beta^{u}_{L,M}(\epsilon,t)\right]^{2}\label{eq:chi2-I}
\end{equation}

Where $\beta^{u}_{L,M}(\epsilon,t;\mathbb{I}^{\zeta\zeta'}$ denotes  the values from a model function, computed for a given set of (complex) matrix elements $\mathbb{I}^{\zeta\zeta'}$, and $\beta^{u}_{L,M}(\epsilon,t)$ the experimentally-measured parameters, for a given configuration $u$. Implicit in the notation is that the matrix elements are independent of $u$ (or otherwise averaged over $u$). Once the matrix elements are obtained in this manner then MF observables, for any $u$, can be calculated. An example of such a protocol is shown in Fig. \ref{781808}, as previously discussed, and the practical realisation of such a methodology is the topic of Sect. \ref{sec:bootstrapping} (see also refs. \cite{hockett2018QMP2,marceau2017MolecularFrameReconstruction} for further discussion). Outstanding questions with such protocols remain, in particular fit generality and reproducibility, and how well they will scale to larger problems (more matrix elements/partial waves).

An alternative methodology has recently been demonstrated, in which the MF observables are determined via a matrix inversion protocol \cite{gregory2021MolecularFramePhotoelectron}. This method does not require - potentially time-consuming - numerical fitting, although still requires knowledge of the channel functions. A full outline of the matrix inversion method is given in Appendix \ref{app:mat-inversion}, and a brief overview below.

For the matrix-inversion approach, the relationship between the LF/AF and MF is considered in terms of a matrix transform:

\begin{equation}
\mathbf{C}^{mol}=\mathbf{G}\mathbf{C}^{lab},\label{eq:basic}
\end{equation}

Where $\mathbf{C}$ are similar to the channel functions defined above, but \textit{without} the inclusion of alignment or frame rotation effects, and will be termed herein as ``reduced" channel functions. The relevant parameters can be explicitly defined as:

\begin{equation}
C_{PR}^{LM}(\epsilon,\Delta q)=\sum_{\zeta\zeta'}\mathbb{I}^{\zeta\zeta'}(\epsilon)\Gamma_{PR\Delta q}^{\zeta\zeta'LM}
\end{equation}

for the MF, and:

\begin{equation}
\bar{C}_{KQS}^{LM}(\epsilon)=\sum_{\zeta\zeta'}\mathbb{I}^{\zeta\zeta'}(\epsilon)\Gamma_{KQS}^{\zeta\zeta'LM}
\end{equation}

Where $\mathbf{\Gamma}$ are the reduced channel functions, explicitly indexed by all required quantum numbers for the LF and MF definitions (as previously, $\zeta$ denotes all other required indices). Again, full details can be found in Appendix \ref{app:mat-inversion}. Given these, the transformation matrix can be written as:

\begin{equation}
\mathbf{G}_{L'M'KS}^{LMP\Delta q}=\mathbf{\Gamma}_{P0\Delta q}^{\zeta\zeta'LM}(\mathbf{\Gamma}_{K0S}^{\zeta\zeta'L^{\prime}M^{\prime}})^{+}
\end{equation}

Here $()^{+}$ indicates the Moore-Penrose inverse matrix of a reduced channel function, which can be computed numerically. Significantly, the matrix elements are not required for inversion, provided that either $\mathbf{C}^{mol}$ or $\mathbf{C}^{lab}$ is known (e.g. from a measurement), and that the reduced channel functions are also known. Therefore, this method does not provide a route to reconstruction of a full set of matrix elements, but can be used to obtain MF observables, and has been demonstrated to work for linear ($N2$$, $D_{\infty h}$ symmetry) and symmetric top ($C_{2}H_{4}$, $D_{2h}$ symmetry) examples. Although not as powerful as a complete experiment methodology, this technique is expected to scale more readily to larger problems (more matrix elements), for which a complete matrix element retrieval via fitting may be impossible.


% - General notes on fitting? Some already below, and may stay there.
% - Matrix inversion method overview?