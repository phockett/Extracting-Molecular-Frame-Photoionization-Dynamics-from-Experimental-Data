\subsubsection{Retrieval \& reconstruction techniques}

% Very generally, the problem of retrieval can be considered in terms of completeness, i.e. the degree to which the full quantum state of the system is know. In photoionization, ``complete" experiments, in which the full system wavefunction (matrix elements), denote the class of measurements in which full matrix element retrieval is the goal, as distinct from measurements which aim at specific observables
% This is currently briefly in Sect. 3.1, but probably should be expanded!

Following the tensor notation presented above, a ``complete" photoionization experiment can be characterized as recovery of the matrix elements 

Of particular note in the following is the phase-sensitive nature of the observables, hence PADs can also be considered as angular interferograms. (Cf. FROG, tomography)

- General notes on fitting? Some already below, and may stay there.
- Matrix inversion method overview?