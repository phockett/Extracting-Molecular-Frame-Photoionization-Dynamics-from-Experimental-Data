% 28/02/23 rev: got this far, above expanded a bit, may still need to review.

\subsection{Photoionization observables, properties and dimensionality\label{sec:Photo-into}}

Fundamentally, experimental observables depend on intrinsic molecular properties. In the  context of molecular physics, such properties are naturally expressed in the molecular frame - for instance: bond lengths, bond angles, polarisabilities, absorption cross-sections, dipole matrix elements and so on. Whilst it is often the case that such properties may be viewed as stationary, and possibly approximated semi-classical, ultimately the dynamics of the system are also of great interest. For instance, in a stationary state the equilibrium or average bond-lengths of a system are often viewed as a well-defined molecular property, but really it is the associated vibrational wavefunction that defines this. However, in a state-resolved experiment, the averaging over the vibrational DOF may be small (i.e. the wavefunction is localised), so this approximation holds. More generally this may not be the case, and will not be the case in a time-resolved experimental methodology wherein a superposition of vibrational states  - a wavepacket - may be prepared. In this sense, the observables in a given case may also be affected by experimental conditions and methods, and degrees-of-freedom or dynamics, all of which may affect what are viewed as fundamental molecular properties. 

As well as MF observables, one may also be interested in determining the underlying intrinsic molecular properties which govern the observable, and/or exploring extrinsic properties which modify or control the observable. For instance, the response to application of different laser field(s) (wavelength, intensity, polarization...), preparation of specific molecular states, or wavepacket dynamics in time-resolved studies. In some cases the mapping between observable(s) and property/ies of interest may be relatively direct, in other cases much less so!

To mitigate these issues, one generally seeks a ``maximum information" measurement to try and understand, untangle or even quantitatively determine the various effects at play. Clearly, the amount of information required will depend on the type of analysis, and complexity of the problem. % As a short-hand for this, the 
This concept is expanded below in terms of a fundamental 1D example (Sect. \ref{sec:1D-case}), and the discussion is extended to the more complex (higher dimensionality) case of the measurement of photoelectron angular distributions in the remainder of this section (Sect. \ref{sec:flux-intro}). An important caveat to this is that, for analysis or retrieval of phases, the measurements must have some degree of phase-sensitivity and coherence - this may be inherent to the observables or DOFs chosen, or imparted experimentally via  interferometric (coherent multi-path) schemes.

\subsubsection{A brief discussion of rotation: 1D problems and beyond\label{sec:1D-case}}

A basic, traditional, example - at least for simple molecules - is the determination of bond lengths from LF measurements of rotational energy spectra. In this case, the intrinsic molecular properties (equilibrium bond lengths and geometry) map relatively cleanly to the observable energy levels; one can thus uniquely determine the properties of interest via a suitable data analysis methodology (for a general introduction, see ref. \cite{hollasHighRes}). For instance, in the simplest case of a rigid homonuclear diatomic system, the problem can be regarded as 1-dimensional, with a single bond length to be determined from a 1D experimental measurement made in the LF. A measurement of the rotational energy spectrum is sufficient to provide bond-length information. In this case, there is no issue with orientational averaging, since the position of features in the energy spectrum are invariant to orientation, although their magnitudes are not. 
(In a more precise description, the observed spectrum is determined by the (rotational) transition moments, which are tied to the molecular symmetry axes. The magnitudes are the projection into the LF of these transition moments, but their energy eigenstates are invariant to orientation; an angle-integrated measurement will thus provide a sufficient information content.)
In this case, the intrinsic molecular properties can therefore be mapped to the LF observable relatively clearly, and the retrieval of these properties from a measurement is similarly direct. However, for more complicated systems, additional information may be required, either in terms of a richer ND (N-dimensional) measurement technique and/or a series of simple measurements with different ``control" parameters, which are chosen to affect the observable but not the intrinsic molecular properties.

% For more complicated systems, additional information may be required, e.g.

More generally, one can regard a hierarchy of difficulty in the determination of molecular properties, based on the complexity or dimensionality of the system and observable(s).
\begin{itemize}
\item Simple systems, ``1D” methods, where a dataset of a single variable (e.g. rotational energy spectrum) is measured. For simple cases, e.g. rigid diatomic molecules, the determination of molecular properties - the rotational energy levels and, hence, the bond length - is relatively direct. (See, for example, Chpt. 5 in Ref. \cite{hollasHighRes}.) % [Illustration?]   
\item Intermediate cases, e.g. larger, but fairly rigid, molecules (for instance, formaldehyde ($H_2CO$) and aniline ($C_6NH_7$), as discussed in Ref. \cite{hollasHighRes}), possibly with congested spectra. In such cases the ``direct” determination of molecular properties may not be possible without more sophisticated modelling and \textit{ab initio} computations. Additional information from higher-dimensionality observables, e.g. polarization-dependent energy spectra, may also be required. An interesting example, and relevant to latter discussion herein, is the use of rotational coherence spectroscopy (RCS) in such cases \cite{Felker1987} - this can be considered as the narrow-wavepacket limit of more general (non-adiabatic) rotational wavepacket based methods (Sect. \ref{sec:MF-control}).
% \textbf{I think RCS is also used for such systems. Should probably cite something from Fleker.}
\item For the most complex cases, e.g. large, floppy molecules or complexes \cite{bunkerMolSymm,schmiedt2015SymmetryExtremelyFloppy}, determination of molecular properties in an absolute sense may be impossible or, rather, may be a poorly posed question since DOFs may become strongly coupled, and simplifying approximations such as fixed point-group symmetry and the BO separation break down \cite{bunkerMolSymm}. High level computational results (if possible/tractable) may be required to understand the spectrum, and associated dynamics, along with multi-dimensional observables (e.g. temperature dependent spectra, dependence on isotopically-substituted species and so on). Such cases remain at the cutting-edge of research in molecular spectroscopy \cite{schmiedt2015SymmetryExtremelyFloppy}.
\end{itemize}    

Whilst this discussion may seem a little arbitrary initially, aside from the general concepts, there is a strong link between traditional rotational spectroscopy and modern molecular alignment techniques (as will be discussed later, Sect. \ref{sec:MF-control}). Furthermore, control of the averaging over rotational (geometric) DOFs is a key to stepping into the MF in time-domain experiments.

\subsubsection{Photoelectron flux in the LF and MF\label{sec:flux-intro}}

A (generally) more complicated example is the topic of this article, i.e. the determination of MF photoelectron distributions, and the intrinsic molecular photoionization dynamics which underlie the observables.  The former may involve direct or indirect techniques; the latter is usually defined as the retrieval of the (complex valued) ionization dipole matrix elements, hence is indirect by definition (since phase information may not be directly measured). 

A broad conceptual overview for a time-dependent measurement scheme and associated observables is illustrated in Fig. \ref{781808}; although the figure shows a specific experimental scheme, the concepts are general. In this example - which forms the basis for the case study of Sect. \ref{sec:bootstrapping} - a multi-pulse experimental scheme is used to prepare an aligned molecular axis distribution, and AF photoionization measurements are made.

The preparation (or pump) step in this case is impulsive molecular alignment, in which one or more laser pulses, of short duration as comparted to
% $\tau_{pulse}<<\tau_{rot}$, 
the characteristic rotational timescale, are used to create a rotational wavepacket in the system. The evolution of this wavepacket corresponds to different ensemble alignments (LF projections), as a function of time. A probe-pulse of similar pulse-duration, and with controllable time-delay $\Delta t$, can then probe specific alignments as a function of time. In general, pulses in the femto-second regime are suitable for this type of experimental scheme, and the technique is quite widely applicable; further discussion can be found in Sect. \ref{sec:MF-control}.

The type of measurements shown are photoelectron images, which provide the angle and energy resolved photoelectron flux. Since the molecular axis distribution is time-dependent in this case,
%(the scheme illustrated is impulsive molecular alignment, which creates a time-dependent rotational wavepacket), 
a set of time-dependent measurements provide a set of observables with different spatial averaging at each measurement time $t$. These observables can be parametrized as a set of time-dependent parameters as shown in the lower right panel. The observables of interest - the photoelectron flux as a function of energy ($\epsilon$), ejection angles ($\theta,\phi$), and time ($t$) - can be written generally as an expansion in spherical harmonics:

\begin{equation}
\bar{I}(\epsilon,t,\theta,\phi)=\sum_{L=0}^{2n}\sum_{M=-L}^{L}\bar{\beta}_{L,M}(\epsilon,t)Y_{L,M}(\theta,\phi)\label{eq:AF-PAD-general}
\end{equation}

Here the flux in the laboratory frame (LF) or aligned frame (AF) is denoted $\bar{I}(\epsilon,t,\theta,\phi)$, with the bar signifying ensemble averaging, and the molecular frame (MF) flux by $I(\epsilon,t,\theta,\phi)$.  Similarly, the expansion parameters $\bar{\beta}_{L,M}(\epsilon,t)$ include a bar for the LF/AF case. These observables are generally termed photoelectron angular distributions (PADs), often with a prefix denoting the reference frame, e.g. LFPADs, MFPADs, and the associated expansion parameters $\bar{\beta}_{L,M}(\epsilon,t)$ are generically termed ``anisotropy" parameters. The polar coordinate system $(\theta,\phi)$ is referenced to an experimentally-defined axis in the LF/AF case (usually defined by the laser polarization), and the molecular symmetry axis in the MF, as indicated in the corresponding panels of Fig. \ref{781808}. 
% The spherical harmonic rank and order $(L,M)$ are constrained by experimental factors in the LF/AF, and $n$ is typically limited by the molecular alignment (which is correlated with the photon-order for gas phase experiments), and can be considered in terms of conservation of total angular momentum in the LF more generally \cite{Yang1948}. In the MF $n$ is constrained only by the maximum continuum electron angular momentum $n=l_{max}$ imparted by the scattering event \cite{Dill1976}. 
% VM: \textbf{Is the largest value of L necessarily even? I guess maybe so.} PH: yes, has to be from the 3js, i.e. always have Lmax = lmax + l'max = 2lmax.

The spherical harmonic rank and order of the observables, $(L,M)$, are constrained by experimental factors in the LF/AF, and $n$ is typically limited by the molecular alignment, which is correlated with the photon-order for gas phase experiments. Generally, this can be considered in terms of conservation of total angular momentum in the LF \cite{Yang1948}, and each photon imparts one unit of angular momentum. For basic cases these limits may be low: for instance, a simple 1-photon photoionization event ($n=1$) from an isotropic ensemble (zero net ensemble angular momentum) defines $L_{max}=2$; for cylindrically or axially symmetric cases (i.e. $D_{\infty h}$ symmetry) $M=0$ only. 

In the MF $n$ is constrained only by the maximum continuum electron angular momentum $n=l_{max}$ imparted by the scattering event \cite{Dill1976} (note lower-case $l$ for the electron angular momentum). For these cases, $l_{max}=4$ is often given as a reasonable rule-of-thumb for the continuum - hence $L_{max}=8$ - although in practice higher-$l$ may be populated. Further details are discussed below, with a realistic example case forming the basis of Sect. \ref{sec:bootstrapping}. (For further introductory discussion and examples of LF and MF PADs, see Refs. \cite{Reid2003,hockett2018QMP1}; a recent review article can be found in Ref. \cite{dowek2022TrendsAngleresolvedMolecular}.)

% A broad conceptual overview for a time-dependent measurement scheme and associated observables is illustrated in Fig. \ref{781808}; although the figure shows a specific experimental scheme, the concepts are general. In this case - which forms the case study of Sect. \ref{sec:bootstrapping} - a multi-pulse experimental scheme is used to prepare an aligned molecular axis distribution

Returning to Fig. \ref{781808}, note, in particular that the LF measurements (top panel) involve averaging over an ensemble of molecules with different orientations, leading to averaging over the molecular frame observables. The $\bar{\beta}_{L,M}(\epsilon,t)$ parameters in this case are shown in the lower right panel (and in more detail in Fig. \ref{720080}), and are constrained to $L_{max}=6$ and $M=0$ (cylindrical or axial symmetry). Two examples of the corresponding AFPADs $\bar{I}(\epsilon,t,\theta,\phi)$ are also shown, as simple polar plots, in the panel. These AFPADs display fairly simple, albeit time-dependent, angular structure. 
% The MF results (lower left panel) are highly structured: compare, in particular, the full MFPADs ($I(\theta,\phi)$ shown, with the relatively unstructured - but time-dependent - LFPADs ($\bar{I}(\theta,t)$) shown in the bottom right panel (see Fig. \ref{454268} for more detailed MFPADs). 
The corresponding MFPADs (lower left panel) are highly structured,  time-independent, quantities $I(\epsilon,\theta,\phi)$ in this case. The difference in the complexity of the MFPADs and AFPADs is typical of spatial averaging, and indicative of why one wishes to obtain MF results if possible. Crucially, these types of process are coherent, and the PADs are sensitive to the (relative) phases of the continuum electrons. PADs may also respond to other phase contributions depending on the type of experiment, for example the spatial averaging in an AF measurement is a coherent process.
% (based on the rotational wavepacket created in the pump process).
In cases - such as this example - where the time-dependence is purely geometric, and is separable from the ionization matrix elements, the total information content can be broadly viewed as the number of sets of $\{L,M\}$ at a given $\epsilon$. 
This represents a rich dataset for retrieving matrix elements and reconstruction of the MFPADs, and this fairly general case is explored in detail herein (Sect. \ref{sec:MF-recon-basic-intro} introduces the concepts, and Sect. \ref{sec:bootstrapping} presents a full numerical case-study). 

% In this particular example, each time-dependent measurement involves averaging over a different alignment of the ensemble, and this represents a rich dataset for retrieving matrix elements and reconstruction of the MFPADs (again, this is explored in detail in Sect. \ref{sec:bootstrapping}). 
% In cases - such as this example - where the time-dependence is purely geometric, and is separable from the ionization matrix elements, the total information content can be broadly viewed as the number of sets of $\{L,M\}$ at a given $\epsilon$.

% [Following is general-ish, but also quite similar to Sect. 4.1 - should make a more careful separation? Would make sense to have some key eqns. here, and elaborate in Sect. 4.]

% [\textbf{PH 27/07/22: tidied to here. TODO: combine the text below with Sect. 4.1, whole section as 3.4: Photoionization dynamics and retrieval. Also move final para up above, and link to dimensionality discussion.}]

% Underlying the observable is the photoelectron continuum wavefunction $\Psi_{e}(\boldsymbol{\mathbf{k}})$, prepared via photoionization, and assumed herein to be a 1-photon dipolar photoionization event. The photoelectron momentum vector is denoted generally by $\boldsymbol{\mathbf{k}}$, in the MF. %, which can be regarded as loosely interchangable with an $(\epsilon,\theta,\phi)$ spherical polar coordinate system in the MF. 
% The ionization matrix elements associated with this transition %\textbf{provide the set of quantum amplitudes} 
% provide the set of quantum amplitudes completely defining the final continuum scattering state,
% \begin{equation}
% \left|\Psi_c\right> = \sum{\int{\left|\Psi_{+};\bf{k}\right>\left<\Psi_{+};\mathbf{k}|\Psi_c\right> d\bf{k}}},
% \label{eq:cstate}
% \end{equation}
% % \textbf{where the sum is over states of the molecular ion $\left|\Psi_{+}\right>$. The number of these accessed depends on the nature of the ionizing pulse.} 
% where the sum is over states of the molecular ion $\left|\Psi_{+}\right>$. The number of these accessed depends on the nature of the ionizing pulse. Importantly, the observables are sensitive to the magnitudes and phases of these matrix elements. The ionization dipole matrix elements can be separated into radial (energy-dependent) and geometric (analytical) parts, and the former can be written generally as:

% \begin{equation}
% \left<\Psi_{+};\mathbf{k}|\Psi_c\right> \equiv r_{k,l,m}=\langle\Psi_{+};\,\psi_{l,m}(k)|\hat{\mathbf{\mu}}.\boldsymbol{\mathbf{E}}|\Psi_{i}\rangle
% \label{eq:r-kllam}
% \end{equation}

% Where the notation implies a perturbative photoionization event from an initial state $i$ to a particular ion plus electron state following absorption of a photon $h\nu$, % $|\Psi_{i}\rangle\stackrel{h\nu}{\rightarrow}|\Psi_{+};\,\Psi_{e}(\boldsymbol{\mathbf{k}})\rangle$, 
% $|\Psi_{i}\rangle+h\nu{\rightarrow}|\Psi_{+};\boldsymbol{\mathbf{k}}\rangle$, and the photoelectron wavefunction is expressed as a set of ``partial waves", with angular momentum components $(l,m)$ (note lower case notation for the partial wave components). Provided that the analytical solutions are known, knowledge of the radial matrix elements $r_{k,l,m}$ at a given $(\epsilon,t)$ thus equates to a full description of the system (and, hence, the observables). 
% %\textbf{The photoelectron angular distribution (PAD) at a given $(\epsilon,t)$ can then be determined by the squared projection of $\left|\Psi_c\right>$ onto a specific state $\left|\Psi_{+};\bf{k}\right>$, and therefore the amplitudes in Eq.~\ref{eq:r-kllam} which then also determine the observable anisotropy parameters $\beta_{L,M}(\epsilon,t)$. If a number of ionic states are accessed, each state is treated independently.}
% The photoelectron angular distribution (PAD) at a given $(\epsilon,t)$ can then be determined by the squared projection of $\left|\Psi_c\right>$ onto a specific state $\left|\Psi_{+};\bf{k}\right>$ (see Sect. \ref{sec:theoretical-techniques}), and therefore the amplitudes in Eq.~\ref{eq:r-kllam} which then also determine the observable anisotropy parameters $\beta_{L,M}(\epsilon,t)$. Typically, for reconstruction experiments, a given measurement will be selected to simplify this as much as possible by, e.g., populating only a single ionic state (or states for which the corresponding observables are experimentally energetically-resolvable), and with a bandwidth $d\bf{k}$ which is small enough such that the matrix elements can be assumed constant.
% % If a number of ionic states are accessed, each state is treated independently. [THIS IS MISLEADING, possibly - it's dependent on too many things!]
% %by knowledge of the anisotropy parameters $\beta_{L,M}(\epsilon,t)$, and does not necessarily require knowledge of the matrix elements.

% A broad conceptual overview for a time-dependent measurement scheme and associated observables is illustrated in Fig. \ref{781808}; although the figure shows a specific experimental scheme, the concepts are general. Note, in particular that the LF measurements involve averaging over an ensemble of molecules with different orientations, leading to averaging over the molecular frame observables, which are highly structured. Compare, in particular, the full MFPADs ($I(\theta,\phi)$ shown in the bottom left panel, with the relatively unstructured - but time-dependent - LFPADs ($\bar{I}(\theta,t)$) shown in the bottom right panel. In this particular example, each time-dependent measurement involves averaging over a different alignment of the ensemble, and represent a rich dataset for retrieving matrix elements and reconstruction of the MFPADs. In cases - such as this example - where the time-dependence is purely geometric, and is separable from the ionization matrix elements, the total information content can be broadly viewed as the number of sets of $\{L,M\}$ at a given $\epsilon$.

% The difficulty in reconstruction in general arises from the fact that there are typically many component partial waves (matrix elements) for even a simple system, and that determination of both magnitudes and phases is required. Hence this can be viewed as a form of quantum tomography, or a specific class of (quantum) phase-retrieval problems. In terms of the MF observable, these properties result in a quantity that may be highly structured and, hence, is (in general) particularly susceptible to orientational averaging. Furthermore, it may be particularly sensitive to averaging over other DOFs (e.g. vibronic states) and/or molecular dynamics, due to inherent sensitivity of the scattering process to molecular structure. Whilst a number of direct and indirect techniques have been used to obtain the relevant MF observables and/or dipole matrix elements, many outstanding questions remain, and this is an ongoing, interesting and challenging area of research \cite{hockett2018QuantumMetrologyPhotoelectrons,hockett2018QuantumMetrologyPhotoelectronsa}.
