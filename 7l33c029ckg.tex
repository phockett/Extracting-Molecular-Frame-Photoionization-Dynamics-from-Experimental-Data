\subsection{Observables, properties and dimensionality}

Fundamentally, experimental observables depend on intrinsic molecular properties, which may also be affected by experimental conditions/methods, and degrees-of-freedom/dynamics. In the  context of molecular physics, such properties are naturally expressed in the molecular frame - for instance: bond lengths, bond angles, polarisabilities, absorption cross-sections, dipole matrix elements and so on. Whilst it is often the case that such properties may be viewed semi-classically, e.g. the equilibrium or average bond-lengths of a system is often viewed as a molecular property, rather than the associated vibrational wavefunction, ultimately the dynamics of the . Many such properties may also be dynamic, and 

As well as MF observables, one may be interested in determining the underlying intrinsic molecular properties which govern the observable, and/or exploring extrinsic properties which modify or control the observable (e.g. application of laser field(s), preparation of specific molecular states or wavepacket dynamics). In some case the mapping between observable(s) and property/ies may be relatively direct, in other cases much less so!

A basic, traditional, example - at least for simple molecules - is the determination of bond lengths from LF measurements of rotational energy spectra. In this case, the intrinsic molecular properties (bond lengths and geometry) map relatively cleanly to the observable energy levels, thus one can uniquely determine the properties of interest via a suitable data analysis methodology. For instance, in the simplest case, a homonuclear diatomic system, the problem can be regarded as 1-dimensional, with a single bond length to be determined from a 1D experimental measurement (the rotational energy spectrum) made in the LF. In this case, there is no issue with orientational averaging, since the position of features in the energy spectrum - determined by the (rotational) transition moments (which are tied to the molecular symmetry axes) - are invariant to orientation, although their magnitudes (or, more precisely, their projection into the LF) are not. In this case, the intrinsic molecular properties can therefore be mapped to the LF observable relatively clearly, and the retrieval of these propetries from a measurement is similarly direct.

For more complicated systems, additional information may be required, e.g.

More generally, one can regard a hierarchy of difficulty in the determination of molecular properties, based on the complexity or dimensionality of the system and observable(s).
\begin{itemize}
\item Simple systems, “1D” methods, where a dataset of a single variable (e.g. rotational energy spectrum) is measured. For simple cases, e.g. rigid diatomic molecules, the determination of molecular properties - the rotational energy levels and, hence, the bond length - is relatively direct. [Illustration?]   
\item Intermediate cases, e.g. larger, but fairly rigid, molecules (for instance, benzene), with congested spectra, the “direct” determination of molecular properties may not be possible without more sophisticated modelling and ab initio computations, and/or additional information from higher-dimensionality observables (e.g. polarization-dependent energy spectra).
\item For the most complex cases, e.g. large, floppy molecules or complexes [examples...? recent floppy mol spectroscopy stuff?], determination of molecular properties in an absolute sense may be impossible (or, rather, may be a poorly posed question); high level computational results (if possible/tractable) may be required to understand the spectrum, and associated dynamics, along with multi-dimensional observables (e.g. temperature dependent spectra, dependence on isotopically-substituted species and so on).
\end{itemize}    

A (generally) more complicated example is the topic of this review, i.e. the determination of MF photoelectron distributions, and the intrinsic molecular photoionization dynamics which underlie the observables.  The former may involved direct or indirect techniques; the latter is usually defined as the retrieval of the (complex valued) ionization dipole matrix elements, hence is indirect by definition (since phase information may not be directly measured). The difficulty in this case arises from the fact that there are typically many components (matrix elements) for even a simple system, and that determination of both magnitudes and phases is required. (Hence this can be viewed as a form of quantum tomography, or a subset of (quantum) phase-retrieval problems.) In terms of the MF observable, these properties result in a quantity that may be highly structured and, hence, is (in general) particularly susceptible to orientational averaging. Furthermore, it may be particularly sensitive to averaging over other DOFs (e.g. vibronic states) and/or molecular dynamics, due to inherent sensitiviy to molecular structure. Whilst a number of direct and indirect techniques have been used to obtain the relevant MF observables and/or dipole matrix elements, many outstanding questions remain, and this is an ongoing, interesting and challenging area of research.
