\subsection{Observables, properties and dimensionality}

Fundamentally, experimental observables depend on intrinsic molecular properties. In the  context of molecular physics, such properties are naturally expressed in the molecular frame - for instance: bond lengths, bond angles, polarisabilities, absorption cross-sections, dipole matrix elements and so on. Whilst it is often the case that such properties may be viewed as stationary, or approximately semi-classical - e.g. the equilibrium or average bond-lengths of a system is often viewed as a molecular property, rather than the associated vibrational wavefunction - ultimately the dynamics of the system is also of great interest. In this sense, the observables in a given case may also be affected by experimental conditions/methods, and degrees-of-freedom/dynamics, and one seeks a "maximum information" measurement to try and understand or quantitatively determine the various effects at play.

As well as MF observables, one may be interested in determining the underlying intrinsic molecular properties which govern the observable, and/or exploring extrinsic properties which modify or control the observable (e.g. application of laser field(s), preparation of specific molecular states or wavepacket dynamics). In some case the mapping between observable(s) and property/ies may be relatively direct, in other cases much less so!

A basic, traditional, example - at least for simple molecules - is the determination of bond lengths from LF measurements of rotational energy spectra. In this case, the intrinsic molecular properties (equilibrium bond lengths and geometry) map relatively cleanly to the observable energy levels, thus one can uniquely determine the properties of interest via a suitable data analysis methodology (for a general introduction, see ref. \cite{hollasHighRes}). For instance, in the simplest case, a homonuclear diatomic system, the problem can be regarded as 1-dimensional, with a single bond length to be determined from a 1D experimental measurement (the rotational energy spectrum) made in the LF. In this case, there is no issue with orientational averaging, since the position of features in the energy spectrum - determined by the (rotational) transition moments (which are tied to the molecular symmetry axes) - are invariant to orientation, although their magnitudes (or, more precisely, their projection into the LF) are not. In this case, the intrinsic molecular properties can therefore be mapped to the LF observable relatively clearly, and the retrieval of these properties from a measurement is similarly direct. However, for more complicated systems, additional information may be required, either in terms of a richer ND measurement technique and/or a series of simple measurements with different ``control" parameters.

% For more complicated systems, additional information may be required, e.g.

More generally, one can regard a hierarchy of difficulty in the determination of molecular properties, based on the complexity or dimensionality of the system and observable(s).
\begin{itemize}
\item Simple systems, ``1D” methods, where a dataset of a single variable (e.g. rotational energy spectrum) is measured. For simple cases, e.g. rigid diatomic molecules, the determination of molecular properties - the rotational energy levels and, hence, the bond length - is relatively direct. (See, for example, Chpt. 5 in ref. \cite{hollasHighRes}.) % [Illustration?]   
\item Intermediate cases, e.g. larger, but fairly rigid, molecules (for instance, formaldehyde $H_2CO$ and aniline, $C_6NH_7$, as discussed in ref. \cite{hollasHighRes}), possibly with congested spectra, the ``direct” determination of molecular properties may not be possible without more sophisticated modelling and ab initio computations, and/or additional information from higher-dimensionality observables (e.g. polarization-dependent energy spectra).
\item For the most complex cases, e.g. large, floppy molecules or complexes \cite{bunkerMolSymm,schmiedt2015SymmetryExtremelyFloppy}, determination of molecular properties in an absolute sense may be impossible (or, rather, may be a poorly posed question); high level computational results (if possible/tractable) may be required to understand the spectrum, and associated dynamics, along with multi-dimensional observables (e.g. temperature dependent spectra, dependence on isotopically-substituted species and so on).
\end{itemize}    

A (generally) more complicated example is the topic of this article, i.e. the determination of MF photoelectron distributions, and the intrinsic molecular photoionization dynamics which underlie the observables.  The former may involved direct or indirect techniques; the latter is usually defined as the retrieval of the (complex valued) ionization dipole matrix elements, hence is indirect by definition (since phase information may not be directly measured). The observables of interest - the photoelectron flux as a function of energy, angle, and time - can be written generally as an expansion in spherical harmonics:

\begin{equation}
\bar{I}(E,t,\theta,\phi)=\sum_{L=0}^{2n}\sum_{M=-L}^{L}\bar{\beta_{L,M}}(E,t)Y_{L,M}(\theta,\phi)\label{eq:AF-PAD-general}
\end{equation}

Here the flux in the laboratory frame (LF) or aligned frame (AF) is denoted $\bar{I}(E,t,\theta,\phi)$, with the bar signifying ensemble averaging, and the molecular frame flux by $I(E,t,\theta,\phi)$. Similarly, the expansion parameters - usually termed ``anistoropy" parameters - include a bar for the LF/AF case. The polar coordinate system $(\theta,\phi)$ is referenced to an experimentally-defined axis in the LF/AF case (usually defined by the laser polarization), and the molecular symmetry axis in the MF. The spherical harmonic rank and order $(L,M)$ are constrained by experimental factors in the LF/AF, but only by the 

Underlying the observable is the photoelectron continuum wavefunction, prepared via photoionization, and assumed herein to be a 1-photon dipolar photoionization event. The ionization matrix elements associated with this transition can be regarded as completely defining the final continuum scattering state, and the observables are sensitive to the magnitudes and phases of these matrix elements. Knowledge of the matrix elements at a given $(E,t)$ thus equates to a full description of the system (and, hence, the observables), whilst the photoelectron angular distribution (PAD) at a given $(E,t)$ is defined by knowledge of the anisotropy parameters $\beta_{L,M}}(E,t)$.

A broad conceptual overview for a time-dependent measurement scheme and associated observables is illustrated in Fig. \ref{781808}; although the figure shows a specific experimental scheme, the concepts are general. Note, in particular that the LF measurements involve averaging over an ensemble of molecules with different orientations, leading to averaging over the molecular frame observables, which are highly structured. Compare, in particular, the full MFPADs ($I(\theta,\phi)$ shown in the bottom left panel, with the relatively unstructured - but time-dependent - LFPADs ($\bar{I}(\theta,t)$) shown in the bottom right panel. In this particular example, each time-dependent measurement involves averaging over a different alignment of the ensemble, and represent a rich dataset for retrieving matrix elements and reconstruction of the MFPADs.

The difficulty in reconstruction in general arises from the fact that there are typically many component partial waves (matrix elements) for even a simple system, and that determination of both magnitudes and phases is required. Hence this can be viewed as a form of quantum tomography, or a specific class of (quantum) phase-retrieval problems. In terms of the MF observable, these properties result in a quantity that may be highly structured and, hence, is (in general) particularly susceptible to orientational averaging. Furthermore, it may be particularly sensitive to averaging over other DOFs (e.g. vibronic states) and/or molecular dynamics, due to inherent sensitivity of the scattering process to molecular structure. Whilst a number of direct and indirect techniques have been used to obtain the relevant MF observables and/or dipole matrix elements, many outstanding questions remain, and this is an ongoing, interesting and challenging area of research \cite{hockett2018QuantumMetrologyPhotoelectrons,hockett2018QuantumMetrologyPhotoelectronsa}.
