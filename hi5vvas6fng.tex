\section{Overview}
The main aim of this topical review is to discuss the determination of molecular frame (MF) photoionization dynamics (and related properties) from laboratory frame (LF) measurements. This particular problem is a subset of the larger topic of determining quantum mechanical properties of molecules (and general quantum mechanical systems) which has been, of course, long at the heart of molecular physics, physical chemistry and related disciplines. Spectroscopy, in particular, has the underlying goal of the determination of atomic and molecular properties with high precision; the ``inverse” problem of transforming \textit{ab initio} (computational) results, which naturally start in the MF, to the LF (in order to compare with experimental measurements), also has a long and storied history. In both cases the issue is, in very general terms, one of complexity and averaging over unobserved quantities/properties/degrees-of-freedom of the system; furthermore, in many cases, certain properties may be poorly understood, may fundamentally obscure other properties of interest, and/or may not be readily computed. These issues are particularly relevant for the specific case of photoionization dynamics, which is an inherently complicated scattering problem, and may be strongly coupled to other molecular properties, i.e. electronic, vibrational and rotational dynamics. 

While many of these issues are general in quantum state reconstruction problems, photoionization dynamics is the focus of this review, and the determination of photoionization dynamics and correlated observables in the molecular frame the main topic of discussion. Historically this determination has been termed a ``complete" photoionization experiment, and has also recently been reframed in terms of quantum tomography and metrology, which has essentially the same aims of complete system reconstruction - in the photoionzation case the photoionization matrix elements fully describe the continuum state populated by photoionization or, equivalently, the continuum density matrix.

Herein, the problem of complexity is approached generally in terms of the dimensionality of the problem, and the information content of the measurement; examples are built-up and discussed in these terms. However, the aim here is not for a comprehensive review of the literature, but more a topical introduction and summary of recent progress in this area, including a (reader-extensible) tutorial overview of concepts, grounded in ab initio calculations (and open-science principles, with corresponding open-source code available online), and with the specific goals of introducing new researchers to this interesting topic, and (attempting to) build bridges between some historically disparate sub-topics/areas of the field.


% As stated previously, this topical review aims to discuss and survey current methodologies for measuring, or otherwise obtaining, MF photoionization properties. The aim here is not for a comprehensive literature review (for these, see, for example, refs. [....]), nor a thorough introduction to the core physics (for this, see, for example, refs. [...]), nor a re-tread of previously published materials. Instead, this review aims to present a (reader-extensible) tutorial overview of concepts, grounded in ab initio calculations (and open-science principles, with corresponding open-source code available online), and a survey of recent method development and experimental progress on the topic. In order to maintain relevance, and address any significant lacunae in the knowledge of the current authors,  a discussion group/wiki/Github repo [TBD] has also been created [ref/details....]. It is hoped that, in this way, this manuscript will become a living document, and a useful resource for interested researchers that will grow over time. It is also hoped that this manuscript, and especially the online discussions, will serve  to garner opinions from a cross-section of interested researchers, and help to bridge the gaps between the various, historically somewhat disparate, communities (e.g. spectroscopy, general AMO physics, quantum information etc.) intereseted in this topic.

In order to try and fulfill these aims, and to make a useful contribution to the community, this review aims to provide a number of supplementary resources for researches, and engender discussion on this topic - see Sect. \ref{sec:resources} for details.
% In particular the analysis routines demonstrated (along with relevant data) are available online, see Sect. \ref{sec:resources} for details. 
It is hoped that, in this way, this manuscript will become a living document, and a useful resource for interested researchers that will grow over time. It is also hoped that this manuscript, and especially the online discussions, will serve  to garner opinions from a cross-section of interested researchers, and help to bridge the gaps between the various, historically somewhat disparate, communities (e.g. spectroscopy, general AMO physics, quantum information etc.) intereseted in quantum state reconstruction in various cases.

\subsection{Outline}

This topical review is structured as follows:

- Sect. \ref{sec:Framing}: general introduction and discussion of the topic in broad terms, suitable for a general reader.
- Sect. \ref{sec:Concepts}: more detailed introduction to background theory and experimental techniques.
- Sect. \ref{sec:Recon}: worked examples for MF reconstruction for two retrieval protocols.
- Sect. \ref{sec:resources}