\section{Overview}
The main aim of this topical review is to discuss the determination of molecular frame (MF) photoionization dynamics (and related properties) from laboratory frame (LF) measurements. This particular problem is a subset of the larger topic of determining quantum mechanical properties of molecules (and general quantum mechanical systems) which has been, of course, long at the heart of molecular physics, physical chemistry and related disciplines. Spectroscopy, in particular, has the underlying goal of the determination of atomic and molecular properties with high precision; the ``inverse” problem of transforming \textit{ab initio} (computational) results, which naturally start in the MF, to the LF (in order to compare with experimental measurements), also has a long and storied history. In both cases the issue is, in very general terms, one of complexity and averaging over unobserved quantities/properties/degrees-of-freedom of the system; furthermore, in many cases, certain properties may be poorly understood, may fundamentally obscure other properties of interest, and/or may not be readily computed. These issues are particularly relevant for the specific case of photoionization dynamics, which is an inherently complicated scattering problem, and may be strongly coupled to other molecular properties, i.e. electronic, vibrational and rotational dynamics. 

While many of these issues are general in quantum state reconstruction problems, photoionization dynamics is the focus of this review, and the determination of photoionization dynamics and correlated observables in the molecular frame the main topic of discussion. Historically this determination has been termed a ``complete" photoionization experiment, and has also recently been reframed in terms of quantum tomography and metrology, which has essentially the same aims of complete system reconstruction - in the photoionzation case the photoionization matrix elements fully describe the continuum state populated by photoionization or, equivalently, the continuum density matrix.

Herein, the problem of complexity is approached generally in terms of the dimensionality of the problem, and the information content of the measurement; examples are built-up and discussed in these terms. However, the aim here is not for a comprehensive review of the literature, but more a topical introduction and summary of recent progress in this area, with the specific goals of introducing new researchers to this interesting topic, and (attempting to) build bridges between some historically disparate sub-topics/areas of the field.

In order to try and fulfill these aims, and to make a useful contribution to the community, this review aims to provide a number of supplementary resources for researches, and engender discussion on this topic. In particular the data and analysis routines used are available online, see Sect. \ref{sec:resources} for details.