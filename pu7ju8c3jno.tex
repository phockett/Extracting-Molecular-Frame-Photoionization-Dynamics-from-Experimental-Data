(For PDF version via arXiv: \href{http://arxiv.org/abs/2209.04301}{http://arxiv.org/abs/2209.04301}, for other online resources see Sect. \ref{sec:resources}.)

% Authors manually created for local builds

% \author[1]{Paul Hockett\thanks{paul@femtolab.ca} \orcid{0000-0001-9561-8433}}%
% \author[2]{Varun Makhija\thanks{vmakhija@umw.edu} \orcid{0000-0002-4975-4888}}
% \affil[1]{National Research Council of Canada, 100 Sussex Drive, Ottawa, ON, K1A 0R6, Canada}%
% \affil[2]{Department of Chemistry and Physics, University of Mary Washington, 1301 College Avenue, Fredericksburg VA, 22401.}%

% \vspace{-1em}

% \date{\today}


% \begingroup
% \let\center\flushleft
% \let\endcenter\endflushleft
% \maketitle
% \endgroup

% \section{Abstract}

% Methods for experimental reconstruction of molecular frame (MF) photoionization dynamics, and related properties - specifically MF photoelectron angular distributions (PADs) and continuum density matrices - are outlined and discussed. General concepts are introduced for the non-expert reader, and experimental and theoretical techniques are further outlined in some depth; particular focus is placed on a detailed example of numerical reconstruction techniques for matrix-element retrieval from time-domain experimental measurements making use of rotational-wavepackets (aligned frame measurements). Ongoing resources for interested researchers are also introduced, including sample data, reconstruction codes (written in python) and associated platform, and literature via online repositories; it is hoped these resources will be of ongoing use to the community.


% \section*{Document history}
%
% 10th March 2023: updated text and figures per referee comments. Updated to GH, postSubmission branch, https://github.com/phockett/Extracting-Molecular-Frame-Photoionization-Dynamics-from-Experimental-Data/tree/postSubmissionUpdates
%
% \begin{description}
% \item [{Status}] Submission in progress.
% \item [{05/09/22}] Current version more-or-less complete, updated on Github and also created Authorea fork.
% \item [{26/08/22}] 2nd Final pass in progress...
% \begin{itemize}
% \item TODO: finish Outlook (Sect. \ref{sec:summary-outlook}), tidy Sects. \ref{sec:Matrix-inversion-example} and \ref{app:mat-inversion}, decide on protocol diagrams (all vs. unified).
% \item Figures and notebooks updated.
% \item Sects. \ref{sec:MF-recon-expt} and \ref{sec:RWPs} added to better cover recon and RWPs.
% \item Sect. \ref{sec:recon-from-MFPADs} on recon from MF added following numerical tests.
% \end{itemize}
% \item [{12/06/22}] Final pass in progress...
% \begin{itemize}
% \item Online repos (Figshare, Zotero) configured, left empty.
% \item Some tidy up of Sects. \ref{sec:bootstrapping} and \ref{sec:Appendix-A} completed.
% \item TODO: finish demo notebooks \& final figures
% \begin{itemize}
% \item 27/07/22 Updated figures \& notebooks almost completed. Figs to doc, notebook to tidy then upload.
% \item 03/08/22 Tidied first half of text (including sub-sectioning) to Sect. 5, and Appendix. Broke references and PDF compilation? (Although HTML seems OK, and updating as sections edited.) Sect. 4 still needs some polishing. Most updated/final figs now in place.
% \end{itemize}
% \item TODO: tidy up intro sections. 05/08/22: now mostly done, MF experimental sections still needs some work though.
% \end{itemize}
% \item [{02/05/22}] Tidied up missing refs (OK on export, possibly not in HTML build?) \& intro sections (still needs quite a bit of work). Theory Sect 4.3 mostly fleshed-out.
% \item [{22/04/22}] Now (mostly) updated with interactive figs and mostly complete recon section.
% \item [{12/04/22}] Adding placeholder figures and notes from \href{https://pemtk.readthedocs.io/en/latest/fitting/PEMtk_analysis_demo_150621-tidy.html}{recon demo docs}.
% \item [{17/09/21}] Started actual recon section...
% \item [{03/09/21}] Revisiting and updating with recent progress \& results.
% \item [{07/07/20}] \href{https://github.com/phockett/Extracting-Molecular-Frame-Photoionization-Dynamics-from-Experimental-Data}{Pushed to Github}, use this for .bib management.
% \item [{03/05/20}] Outline/rough draft begun on Authorea, based on notes Dec. 2019.
% \end{description}
