\subsection{MF reconstruction via matrix inversion\label{sec:Matrix-inversion-example}}

The key aspects of this method involve first fitting out the ADMs $A^K_{QS}(t)$ from measured $\bar{\beta}^u_{LM}(t)$ in Eq.~\ref{eqn:beta-convolution-C} to retrieve the $\bar{C}^{LM}_{KQS}$ (the vector $\mathbf{C}^{lab}$), and construction of the matrix $\mathbf{{G}}^{LMP\Delta q }_{L'M'KS}$ which facilitates the extraction of the molecular frame coefficients $C^{LM}_{P\Delta q}$ (the vector $\mathbf{C}^{mol}$). The steps in the process are depicted in Fig.~\ref{731792}, which label's primary inputs as the `AF/LF measurements', referring to the $\bar{\beta}^u_{LM}(t)$, and the ADMs $A^K_{QS}(t)$ calculated for a grid of fluence and temperature values by solving the TDSE using the appropriate Hamiltonian for impulsive alignment~\cite{}. The `Alignment Retrieval' process requires solving the linear equations Eq.~\ref{eqn:beta-convolution-C} for the $C^{LM}_{P\Delta q}$ using these primary inputs over the entire gird of ADMs, and comparing to the measured data to find the bets fit solution. This provides the vector $\mathbf{C}^{lab}$ needed for the next step - solving the linear equations Eq.~\ref{eq:basic}. This also requires that the matrix $\mathbf{{G}}^{LMP\Delta q }_{L'M'KS}$ be invertable. This matrix therefore encodes the uniquely accessible MF information in the experiment. Specifically, rows in the matrix link the LF parameter $\bar{\beta}^u_{LM}(t)$  to an MF beta parameter $\beta_{LM}(\theta,\chi)$ that can be used to construct the MFPADs, $\theta$ and $\chi$ being polar and azimuthal angles of the laser polarization in the MF. Specifically, a particular row, or set of rows, being entirely zero may indicate that certain MF $\beta_{LM}$ are inaccessible by the set of LF measurements. For instance, in~\cite{gregory2021MolecularFramePhotoelectron} the $\mathbf{{G}}^{LMP\Delta q }_{L'M'KS}$ matrix for the $N_2(X^{1}\Sigma^{+}_{g}) \rightarrow N^+_2(X^{2}\Sigma^{+}_{g})$ ionization of N$_2$ with a single photon, from a RWP excited by a linearly polarized pulse was computed. Rows with $M = 1$ were all determined to be identically zero, therefore only allowing the retrieval of MFPADs with the ionizing field parallel and perpendicular to the molecular axis. In this case, the partial wave analysis involved in the previously described method, along with the non-linear fitting necessarry to retireve the radial dipole matrix elements is avoided. However, only two of the MFPADs shown in Fig.~\ref{731792} extracted by bootstrapping are retrieved by this method.\\

The utility of this method is better demonstrated by application to a polyatomic molecule of $D_{nh}$ symmetry. Gregory et al.~\cite{gregory2021MolecularFramePhotoelectron} also compute the $\mathbf{{G}}^{LMP\Delta q }_{L'M'KS}$ matrix for $C_2H_4(^1A_g) \rightarrow C_2H_4^+(^2B_{3u})$ ionization of $C_2H_4$, with $D_{2h}$ symmetry, from a RWP excited by linearly polarized light. In contrast with $N_2$, all rows of the $\mathbf{{G}}^{LMP\Delta q }_{L'M'KS}$ are non-zero for $C_2H_4$, but the system of linear equations is inconsistent since there are more MF parameters (in the vector $\mathbf{C}^{mol}$) than LF measurements (in the vector $\mathbf{C}^{lab}$). Nonetheless, the solution selected by Gregory et al. minimizes the retrieved vector $\mathbf{C}^{mol}$ and correctly reproduce the MFPADs for any orientation. This `extra' information available in the LF measurement for $C_2H_4$ compared to $N_2$ can be traced back to the fact that the RWP is 2D for an asymmetric top excited by linearly polaized light(cf Eq.~\ref{eq:mfrealsig}). While this method does not directly provide radial dipole matrix elements, the MFPADs produced can potentially be used as a constraint, along with additional experimental data, to determine these from a non-linear fit. 