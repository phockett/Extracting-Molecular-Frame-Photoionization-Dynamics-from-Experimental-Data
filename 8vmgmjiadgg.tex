\subsubsection{Bootstrapping basics: obtaining MFPADs}

With the matrix elements to hand, the MFPADs can also be computed, for any arbitrary molecular alignment and polarization. Some examples are shown in Fig. \ref{454268}, which shows results for the two sets of retrieved matrix elements discussed above (raw, and with phase-correction), along with the reference computational results (as used to originally generate the simulated data) and the differences between the phase-corrected case and the reference results.

A number of points are illustrated by the figure:

\begin{itemize}
\item The original/input matrix elements produce the MFPADs shown in the top row of the figure, which provide a reference for the fidelity of the reconstructed MFPADs. 
\item Polarization geometries are illustrated for four cases, labelled $z,x,y,d$, where $z$ is parallel to the bond axis, $x,y$ perpendicular and $d$ diagonal ($\pi/4$). In this case symmetry dictates that $z$ ($\sigma_u$ continuum) is cylindrically symmetric, the $x,y$ cases are a conjugate $\pi_u$ pair, and that the diagonal case mixes these components, hence breaks the symmetry and is sensitive to the relative phase between the continua.
\item The 2nd row illustrates the results from the phase-corrected parameters. These show generally good agreement with the reference case, apart from slightly exaggerated off-diagonal lobes for the $x,y$ results, consistent with the larger magnitudes for $|3,\pm1\rangle$ in the reconstructed matrix elements.
\item The 3rd row shows difference plots between the reference and phase-corrected results. This emphasizes the differences in the $|3,\pm1\rangle$ component magnitudes. The $z$ case shows very good agreement with the reference case (note the smaller magnitudes), whilst the $d$ case indicates some differences, again attributable to the differences in $|3,\pm1\rangle$ component magnitudes.
\item The bottom row illustrates the case of the phase-averaged (but not "corrected") parameters. This serves to highlight the sensitivity of the PADs to the relative phases of the matrix elements - here the $x,y$ results show poor agreement with the reference case due to averaging over the $\pm\phi$ pairs, and this also affects the $d$ case.

\end{itemize}

Overall, with careful treatment of the phases, it is clear that high-fidelity MFPADs can be recovered in the current case. Although the difference plots indicate some differences between the reconstructed and reference case, the absolute changes in the form of the MFPADs are fairly minor. The lack of an absolute phase is not an issue in general for MFPAD recovery, although this does constitute a loss of relative phase in the continuum wavefunctions as a function of energy. % More on this here? Or elsewhere? Maybe RABITT etc. discussion?

% To further investigate the different parameter sets discussed above, which were essentially identical in the AF-$\beta_{L,M}$ comparisons, Fig. [XX] illustrates some MFPADs for these cases too. [TODO - if interesting]