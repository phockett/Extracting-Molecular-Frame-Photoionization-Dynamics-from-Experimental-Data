\subsection{Molecular frame observables\label{sec:MF-intro}}

A very general issue, of fundamental interest, which falls in the latter categories - and lacks a general solution - is the measurement of observables which depend on the orientation of the molecular frame (MF). In general, this can be regarded as a geometric problem in the spatial (rotational) degrees of freedom, in which the intrinsic MF properties are averaged (or smeared out) in a given measurement. For instance, in a typical gas phase molecular spectroscopy measurement, the observables in the molecular frame  are averaged over all possible molecular orientations in a laboratory frame (LF) measurement. The inverse problem is also difficult, i.e. simulating specific experimental observables starting from \textit{ab initio} calculations in the molecular frame. In this case, to simulate LF results, knowledge of the degree and type of geometric averaging is generally required. In more complex cases \textit{ab initio} calculations may need to be carried out for various molecular orientations, or coupled DOFs such as vibrational motions. Of particular interest in the current case will be the aligned frame (AF), which signifies a measurement frame with some degree of spatial preparation (alignment or orientation of the molecular axis ensemble) of the sample via, for instance, photon absorption (see Sect. \ref{sec:MF-control} for further discussion of the AF).
%\textbf{, or in some cases ab initio calculations need to be carried out for various molecular orientations }. 

Despite, or perhaps because of, the inherent difficulties, obtaining  MF measurements remains a topic of great interest, and much progress in experimental methodologies has been made in recent decades \cite{Becker1998,Reid2003,Reid2012,kleinpoppen2013perfect,Yagishita2015,hockett2018QMP1}. 
% kleinpoppen2013perfect, Yagishita2015, hockett2018QuantumMetrologyPhotoelectrons}. %[JPB MFPADs special, check book refs.]. 
These methodologies can be classified, approximately, as sitting somewhere on the spectrum between (1) relatively direct and (2) indirect methodologies. In the limit of (1), the methodologies are designed to enable control and reconstruction of the MF, hence measure MF observables (somewhat) directly, with minimal data processing requirements. Another avenue to MF observables is the limiting case of indirect methodologies (2), which involve more simulation and have less stringent (or at least different) experimental requirements. In the current context, this can be defined as techniques which make use of detailed theory and analysis procedures to reconstruct MF properties from LF measurements. Broadly speaking, one can also consider indirect methods as a post-processing or analysis-based approach, with a significant theoretical and computational requirement (akin to computational imaging techniques, as well as many traditional energy-domain spectroscopy techniques). In contrast, direct methods are, conceptually, closer to purely experimental methods, with more emphasis placed on detection techniques and capabilities, although significant numerical data analysis may still be required. 

There is, naturally, significant overlap between these extreme case definitions, and most methodologies and extant demonstrations fall somewhere on the spectrum between direct (``purely" experimental) and indirect. In particular, indirect approaches will typically still require some degree of (potentially sophisticated) experimental control, and a set of associated measurements, to be useful. For example, a set of LF measurements with different polarization geometries or with an optically-prepared aligned ensemble may be required. Direct measurements usually require specific molecular behaviour(s) and/or preparation in order to measure suitable observables, due to experimental restrictions. A typical requirement in multi-particle coincidence imaging techniques may be, for example, molecular fragmentation following photoionization, since it is measurement of the fragments that allows for MF reconstruction in these methodologies. Historically different methods have typically been pursued by experimental communities making use of either laser-based (LF, AF and control type schemes) or synchrotron-based (direct MF measurements) experimental methods. Whilst the underlying photoionization and molecular physics is shared, the different classes of experiment typically suit soft or hard photon energies, for valence photoionization and multi-photon techniques, or core ionization and dissociative photoionization based techniques, respectively, although the split is not rigorous. However, there is increasingly overlap between the communities, particularly in the last decade or two. In particular, the advent of time-resolved hard photon sources and strong-field and emergent attosecond laser techniques has pushed developments on the ``tabletop" side (for a broad review of techniques in attoscience, see Ref. \cite{Krausz2009}). Meanwhile, the advent of free-electron lasers (FELs), and the increasing availability of modern ultrafast lasers at synchrotron and FEL facilities, has enabled multi-source experiments in the time or frequency domain spanning optical and X-ray wavelengths and methods. (For recent perspectives covering many of these topics, including recent ultrafast and X-ray developments, see, for instance, Refs. \cite{Young2018,ueda2019RoadmapPhotonicElectronic}.)

It is also noteworthy that in the context of ultracold physics a number of sophisticated techniques have been developed to coherently populate individual molecular eigenstates \cite{mitra2022QuantumControlMolecules}. While these techniques have yet to be applied to molecular photoionization, recent examples exist of similar techniques applied to atomic photoionization \cite{desilva2021CircularDichroismAtomic}.

% PH 27/07/22 added this to main text above
% \textbf{Maybe this goes here, maybe it doesn't, but we should probably mention it somewhere: It is noteworthy that in the context of ultracold physics a number of sophisticated techniques have been developed to coherently populate individual molecular eigenstates} \cite{mitra2022}. \textbf{While these techniques have yet to be applied to molecular photoionization, recent examples exists of similar techniques applied to atomic photoionization.} \cite{desilva2021}