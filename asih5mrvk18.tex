\subsection{Molecular frame observables}

A very general issue, of fundamental interest, which falls in the latter categories  - and lacks a general solution - is the measurement of observables which depend on the orientation of the molecular frame (MF). In general, this can be regarded as a geometric problem in the spatial (rotational) degrees of freedom, in which the intrinsic MF properties are averaged (or smeared out) in a given measurement. For instance, in a typical gas phase molecular spectroscopy measurement, the observables in the molecular frame  are averaged over all possible molecular orientations in a laboratory frame (LF) measurement. The inverse problem is also difficult, i.e. simulating specific experimental observables starting from ab initio calculations in the molecular frame. In this case, to simulate LF results, knowledge of the degree and type of geometric averaging is required. 

Despite, or perhaps because of, the inherent difficulties, obtaining  MF measurements remains a topic of great interest, and much progress in experimental methodologies has been made in recent decades \cite{,Becker1998,Reid2003,Reid2012, kleinpoppen2013perfect, Yagishita2015, hockett2018QuantumMetrologyPhotoelectrons} [JPB MFPADs special, check book refs.]. These methodologies can be classified, approximately, as (1) relatively direct, meaning that the methodologies seek to control or reconstruct the MF, hence measure MF observables, (somewhat) directly, with minimal data processing requirements. Another avenue to MF observables is via what may be called (2) indirect methodologies. In the current context, this can be defined as techniques which make use of detailed theory and analysis procedures to reconstruct MF properties from LF measurements. Broadly speaking, one can also consider indirect methods as a post-processing or analysis-based approach, with a significant theoretical and computational requirement (akin to computational imaging techniques, as well as many traditional energy-domain spectroscopy techniques), while direct methods are, conceptually, closer to purely experimental methods, with more emphasis placed on detection techniques and capabilities, although significant numerical data analysis may still be required. 

There is, naturally, significant overlap between these extreme case definitions, and most methodologies and extant demonstrations fall somewhere on the spectrum between direct ("purely" experimental) and indirect. In particular, indirect approaches will typically still require some degree of (potentially sophisticated) experimental control, and a set of associated measurements, to be useful - for example, LF measurements with different polarization geometries; direct measurements usually require specific molecular behaviour(s) and/or preparation in order to measure suitable observables, due to experimental restrictions.