\subsection{Final state density matrix\label{sec:density-mat-full}}

% Write the full final state as a density matrix:

% \begin{equation}
% \mathbf{\bar{\rho}}_{L,M}^{\zeta\zeta'}=\varUpsilon_{L,M}^{u,\zeta\zeta'}\mathbf{\rho}^{\zeta\zeta'}
% \end{equation}

% Here the density matrix can be interpreted as the final, LF/AF or
% MF density matrix, incorporating both the intrinsic and extrinsic
% effects (i.e. all channel couplings and radial matrix elements for
% the given measurement). The $L,M$notation indicates here that these
% dimensions should not be summed over, hence we can still write the
% tensor coupling into the $\beta_{L,M}^{u}$ parameters:

% \begin{equation}
% \beta_{L,M}^{u}=\sum_{\zeta,\zeta'}\mathbf{\bar{\rho}}_{L,M}^{\zeta\zeta'}
% \end{equation}

% [02/08/22 - moved text from main section. Keep that simple, more details here]

% \textbf{I don't think I fully understand this section. In the partial wave basis, wouldn't the density operator be,}
% \begin{equation}
% \rho(t) =\sum_{LM}\sum_{KQS}A^{K}_{QS}(t)\sum_{\zeta\zeta^{\prime}}\Gamma^{\zeta\zeta^{\prime}LM}_{KQS}|\zeta,\Psi_+\rangle\langle\zeta,\Psi_+|\mu_q\rho_i\mu_{q\prime}^{*}|\zeta^{\prime},\Psi_+\rangle\langle\zeta^{\prime},\Psi_+|
% \end{equation}
% \textbf{with} $\rho_i = |\Psi_i\rangle\langle\Psi_i|$\textbf{ ? It's likely I'm not fully understanding the notation here, because I'm used the notation in MG 2021. Either way, maybe we don't need to include the density matrix version here? Eq.~\ref{eq:cstate} is intended to drive home the quantum tomography perspective.}\\

As introduced in Sect. \ref{sec:density-mat-basic}, the (radial) density matrix can be expressed as the outer-product of the (radial) matrix elements. Following the channel function notation, it is also trivial to write the radial matrix elements in density matrix form in the $\zeta\zeta'$ representation:

\begin{equation}
\mathbf{\rho}^{\zeta\zeta'} = |\zeta\rangle\langle\zeta'| \equiv \mathbb{I}^{\zeta,\zeta'}\label{eqn:radial-density-mat}
\end{equation}

And the full final continuum state as a density matrix in the $\zeta\zeta'$ representation (with the observable dimensions $L,M$ explicitly included in the density matrix), which will also be dependent on the choice of channel functions ($u$):

\begin{equation}
\mathbf{\rho}_{L,M}^{u,\zeta\zeta'}=\varUpsilon_{L,M}^{u,\zeta\zeta'}\mathbb{I}^{\zeta,\zeta'}
\end{equation}

Here the density matrix can be interpreted as the final, LF/AF or
MF density matrix (depending on the channel functions used), incorporating both the intrinsic and extrinsic
effects (i.e. all channel couplings and radial matrix elements for
the given measurement), with dimensions dependent on the unique sets of quantum numbers required - in the simplest case, this will just be a set of partial waves $\zeta = (l,m)$. Note that this final state is distinct from the ``radial" density matrix (Eqn. \ref{eqn:radial-density-mat}), which encodes purely intrinsic (molecular scattering) photoionization dynamics (thus characterises the scattering event). The $L,M$ notation indicates here that these dimensions should not be summed over, hence the tensor coupling into the $\beta_{L,M}^{u}$ parameters can also be written in terms of the density matrix:

\begin{equation}
\beta_{L,M}^{u}=\sum_{\zeta,\zeta'}\mathbf{\rho}_{L,M}^{u,\zeta\zeta'}
\end{equation}

In fact, this form arises naturally since the $\beta_{L,M}^{u}$ terms are the state multipoles (geometric tensors) defining the system, which can be thought of as a coupled basis equivalent of the density matrix representations (see, e.g., ref. \cite{BlumDensityMat}, Chpt. 4.).

In a more traditional notation (cf. Eqn. \ref{eq:cstate}, see also Refs. \cite{gregory2021MolecularFramePhotoelectron} 
%[plus VM electron coherences? And some older refs]
), the density operator can be expressed as:

% \begin{equation}
% \rho(t) =\sum_{LM}\sum_{KQS}A^{K}_{QS}(t)\sum_{\zeta\zeta^{\prime}}\Gamma^{\zeta\zeta^{\prime}LM}_{KQS}|\zeta,\Psi_+\rangle\langle\zeta,\Psi_+|\mu_q\rho_i\mu_{q\prime}^{*}|\zeta^{\prime},\Psi_+\rangle\langle\zeta^{\prime},\Psi_+|
% \end{equation}


\begin{equation}
\rho(t) =\sum_{LM}\sum_{KQS}A^{K}_{QS}(t)\sum_{\zeta\zeta^{\prime}}\varUpsilon_{L,M}^{u,\zeta\zeta'}|\zeta,\Psi_+\rangle\langle\zeta,\Psi_+|\mu_q\rho_i\mu_{q\prime}^{*}|\zeta^{\prime},\Psi_+\rangle\langle\zeta^{\prime},\Psi_+|
\end{equation}


with $\rho_i = |\Psi_i\rangle\langle\Psi_i|$. This is, effectively, equivalent to an expansion in the various tensor operators defined above, in a state-vector notation.


% TODO: confirm notation here, may want to include more discussion from Blum? (See Fig. \ref{998904} for results, should also confirm representation here from outer product case, maybe plot LM representation too - see \href{https://epsproc.readthedocs.io/en/dev/methods/density_mat_notes_demo_300821.html}{ePSproc density matrix notes} esp. \href{https://epsproc.readthedocs.io/en/dev/methods/density_mat_notes_demo_300821.html#Density-matrix-from-geometric-tensors}{Density matrix from geometric tensors section}.)


\subsection{Matrix inversion formalism\label{app:mat-inversion}}

Following Gregory et. al. \cite{gregory2021MolecularFramePhotoelectron}, the formalism can also be rewritten with
the ADMs separate (see also Reid \& Underwood convolution form, \cite{Reid2000}), and the LF/AF given as (cf. Eqn. \ref{eq:BLM-tensor-AF}):

\begin{equation}
\bar{\beta}_{L,M}(E,t)=\sum_{K,Q,S}C_{K,Q,S}^{L,M}(E)A_{Q,S}^{K}(t)
\end{equation}

% ($Q=0$ only in Gregory et. al., don't recall if this is general limitation however - may be symmetry and/or dimensionality related?)
(Note that $Q=0$ only for the derivations in Gregory et. al. \cite{gregory2021MolecularFramePhotoelectron}, although the results should generalise - see discussion therein for details, particularly Sect. 3.2 and the appendices.)

And the MF (cf. Eqn. \ref{eq:BLM-tensor-MF}):

\begin{equation}
\beta_{L,M}(E,\Omega)=\sum_{P,R,\Delta q}C_{P,R}^{L,M}(E,\Delta q)D_{R,\Delta q}^{P}(\Omega)
\end{equation}

And the transformation can be written as (eqn. 1, 28):

\begin{equation}
\mathbf{C}^{mol}=\mathbf{G}\mathbf{C}^{lab},\label{eq:basic}
\end{equation}

\begin{equation}
C_{P,0}^{L,M}(E,\Delta q)=\hat{G}_{L'0KS}^{LMP\Delta q}\bar{C}_{K,0,S}^{L',0}(E)
\end{equation}

% (Here $M'=Q=R=0$, again not sure if this is required.)

And

\begin{equation}
\mathbf{G}_{L'M'KS}^{LMP\Delta q}=\mathbf{\Gamma}_{P0\Delta q}^{\zeta\zeta'LM}(\mathbf{\Gamma}_{K0S}^{\zeta\zeta'L^{\prime}M^{\prime}})^{+}
\end{equation}

Where $+$ denotes the Moore-Penrose inverse matrix.

\begin{equation}
\bar{C}_{KQS}^{LM}(\epsilon)=\sum_{\zeta\zeta'}d_{\zeta\zeta'}(\epsilon)\Gamma_{KQS}^{\zeta\zeta'LM}
\end{equation}

\begin{equation}
C_{PR}^{LM}(\epsilon,\Delta q)=\sum_{\zeta\zeta'}d_{\zeta\zeta'}(\epsilon)\Gamma_{PR\Delta q}^{\zeta\zeta'LM}
\end{equation}

\begin{equation}
D_{\zeta}(\epsilon)D_{\zeta'}^{*}(\epsilon)=d_{\zeta\zeta'}(\epsilon)
\end{equation}

In the current notation, we can rewrite:

\begin{equation}
\bar{C}_{KQS}^{LM}(\epsilon)=\sum_{\zeta\zeta'}\mathbb{I}_{\zeta\zeta'}^{\Gamma,\Gamma'}(\epsilon)\Gamma_{KQS}^{\zeta\zeta'LM}
\end{equation}

\begin{eqnarray}
\Gamma_{KQS}^{\zeta\zeta'LM} & = & (-1)^{M}[P]^{\frac{1}{2}}E_{P-R}(\hat{e};\mu_{0})(-1)^{(\mu'-\mu_{0})}\bar{\Lambda}_{R'}(\mu,P,R')B_{L,S-R'}(l,l',m,m')\Delta_{L,M}(K,Q,S)\\
 & = & \bar{\varUpsilon}_{L,M}^{u,\zeta\zeta'}/A_{Q,S}^{K}(t)
\end{eqnarray}

\begin{equation}
C_{PR}^{LM}(\epsilon,\Delta q)=\sum_{\zeta\zeta'}\mathbb{I}_{\zeta\zeta'}^{\Gamma,\Gamma'}(\epsilon)\Gamma_{PR\Delta q}^{\zeta\zeta'LM}
\end{equation}

\begin{equation}
\Gamma_{PR\Delta q}^{\zeta\zeta'LM}=(-1)^{M}(2P+1)^{\frac{1}{2}}E_{P-R}(\hat{e};\mu_{0})(-1)^{(\mu'-\mu_{0})}\bar{\Lambda}_{\Delta q}(\mu,P,\Delta q)B_{L,-M}(l,l',m,m')
\end{equation}

assuming $R'=\Delta q$, and that the rotational matrix element $D_{R,\Delta q}^{P}(\Omega)$
is computed independently (note that the current numerical function $\Lambda_{R',R}(R_{\hat{n}};\mu,P,R,R')$
could possibly be used directly here, but this version keeps the angle-dependence
separate as per the matrix inversion formalism).
%; \textcolor{red}{alternatively a new function might be used here?}).