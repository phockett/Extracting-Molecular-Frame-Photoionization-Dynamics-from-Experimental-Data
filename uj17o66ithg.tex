\subsection{Bootstrapping to the MF example case}

As detailed elsewhere [TBD/refs], the ``bootstapping" methodology proceeds via a multi-stage fitting protocol to obtain both the AF axis distribution (level 1 of the bootstrap, linear fit), and the photoionization matrix elements (level 2 of the bootstrap, non-linear fit). A schematic overview is given in Figure XX. In the simplest case, this proceeds in a direct manner, and a final set of results are obtained. However, further bootstrapping - in the statistical sense - can also be employed to probe the fidelity of the reconstruction with different samples from the source data (typically sub-sets from the measured time-steps and/or $\beta_{LM}$ parameter sub-sets), to further test the robustness of the results, and estimate uncertainties and/or improve upon them. Some of these possibilities are explored below, and further details may also be found in refs. \cite{hockett2018QMP1,hockett2018QMP2,marceau2017MolecularFrameReconstruction} (and references therein).


\subsubsection{Bootstrapping basics: dataset and setup}

For the example case, synthetic data was used, although the dataset was inspired by ref. \cite{marceau2017MolecularFrameReconstruction}. In that case, experimental results were obtained via a (dual) pump-probe scheme, where the IR pump pulses prepared a rotational wavepacket in $N_2$, and a time-delayed XUV probe pulse ionized the sample. Multiple channels were probed in this manner, for a range of time-delays, to provide a dataset of AF-$\beta_{LM}(E,t)$ parameters. Here the focus is on the photoionization step, information content and further investigation of the retrieval routines; this is facilitated by the following choices and data:

\begin{enumerate}
\item The same rotational wavepacket \& molecular axis distribution as obtained experimentally is assumed. Consequently, the 1st (linear) stage of the bootstrapping protocol is not investigated herein.\footnote{Although this may seem like a drastic omission, in general this stage of the methodology is expected to be quite robust, and this has been demonstrated in various investigations of rotational wavepackets and molecular alignment techniques. [REFS]}
\item To simulate the observables, photoionization matrix elements from ePolyScat \cite{Gianturco1994,Natalense1999} calculations are used [REFS: N2 ePSdata]. This, naturally, also provides a means for direct comparison and fidelity analysis of the retrieval protocol.
\item To investigate limitations of the numerical routines, noise and other artefacts can be added to the simulated data; different sub-sets of the data can also be readily analysed and compared.
\item Finally, it is of note that the numerical implementations are structured as a set of tensors, as close as possible to the formalism given above [TODO]. This provides a means to further investigate the information content of various parts of the problem, and investigate their influence on the retrieval - in particular, these can indicate aspects of the data which may be most sensitive to particular matrix elements, most susceptible to noise and so forth.
\end{enumerate}

The sample dataset is illustrated in Figure XX and, as mentioned previously, full numerical data can be accessed online [REF].

\subsubsection{Bootstrapping basics: 10-point fit}

As an initial test of the method, noise-free data was used, and 10 temporal points over the main revival feature were randomly selected. To fit the data, a set of fitting parameters were defined in (magnitude, phase) form. In this case, the parameter set was simply defined from the input matrix elements, including symmetry relations; in general this step in the protocol may require manual configuration for the problem at hand [MORE ON THIS? MATRIX METHOD TOO]. The PEMtk fitting matrix element fitting routine wraps the lmfit package [REF, https://lmfit.github.io/lmfit-py/, https://dx.doi.org/10.5281/zenodo.11813], in this case a standard Levenberg-Marquardt minimization was used, and all parameters allowed to float freely. To gain insight into the efficiency of the fitting routine, and the uniqueness of the fit results [REF], 1000 fits were performed on the same dataset, each seeded with randomised parameter values. (This methodology amounts to a statistical sampling of the solution hyperspace, which may be expected to contain some local minima in general high-dimensional cases. [REF]) Convergence criteria were set as .... Running in parallel on 28 cores of an AMD Threadripper XXXX based workstation, this required ~5GB of RAM and took ~1 hour; further benchmarks for the current codebase can be found at [REF].

Results are summarised in Figs. XX. A few conclusions from this case study:

\begin{enumerate}
\item Not all fits that converged - as defined by the criteria above - provided the reference matrix elements. This illustrates the importance of statistical sample of the solution hyperspace, and/or the use of further tests of the fit results [REF] in order to determine their veracity in general.
\item In this case, the lowest $\chi^2$ does correspond to the reference matrix elements. Further, in the noise-free case, the minima found in these cases is orders of magnitude lower than the next best parameter set, although this may not hold in general depending on the quality of the data (see Sect. XX), and the complexity of the parameter-set.
\item For the free floating parameter case, the results are shown in both raw form and with a reference phase set in Figs. XX and XX. This illustrates the phase sensitivity of the fitting, and also that this is a \textit{relative} phase. In a similar fashion to statistically sampling the parameter space with the seed parameters, allowing unconstrained parameters also helps to ensure that the full By setting a reference phase
\end{enumerate}

TO CONSIDER:

- Level of detail here, words vs. just pointing at the notebook.