\subsection{Bootstrapping to the MF example case}

As detailed elsewhere [TBD/refs], the ``bootstapping" methodology proceeds via a multi-stage fitting protocol to obtain both the AF axis distribution (level 1 of the bootstrap, linear fit), and the photoionization matrix elements (level 2 of the bootstrap, non-linear fit). A schematic overview is given in Figure XX. In the simplest case, this proceeds in a direct manner, and a final set of results are obtained. However, further bootstrapping - in the statistical sense - can also be employed to probe the fidelity of the reconstruction with different samples from the source data (typically sub-sets from the measured time-steps and/or $\beta_{LM}$ parameter sub-sets), to further test the robustness of the results, and estimate uncertainties.