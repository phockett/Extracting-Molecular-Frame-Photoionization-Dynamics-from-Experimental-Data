\subsection{Bootstrapping to the MF example case\label{sec:bootstrapping}}

% [TODO: outline figure for methods]

The ``generalised bootstrapping" methodology proceeds via a multi-stage fitting protocol to obtain both the AF axis distribution (level 1 of the bootstrap, linear fit), and the photoionization matrix elements (level 2 of the bootstrap, non-linear fit). A general overview is given in Fig. \ref{781808}, and a more detailed schematic overview is given in Fig. \ref{807606}. In the simplest case (as illustrated), this proceeds in a direct manner, and a final set of results are obtained. However, further bootstrapping - in the statistical sense - can also be employed to probe the fidelity of the reconstruction with different samples from the source data (typically sub-sets from the measured time-steps and/or $\beta_{LM}$ parameter sub-sets), to further test the robustness of the results, and estimate uncertainties and/or improve upon them. Some of these possibilities are explored below, and further details may also be found in Refs. \cite{hockett2018QMP1,hockett2018QMP2,marceau2017MolecularFrameReconstruction} (and references therein).


