\subsection{Bootstrapping to the MF example case}

As detailed elsewhere [TBD/refs], the ``bootstapping" methodology proceeds via a multi-stage fitting protocol to obtain both the AF axis distribution (level 1 of the bootstrap, linear fit), and the photoionization matrix elements (level 2 of the bootstrap, non-linear fit). A schematic overview is given in Figure XX. In the simplest case, this proceeds in a direct manner, and a final set of results are obtained. However, further bootstrapping - in the statistical sense - can also be employed to probe the fidelity of the reconstruction with different samples from the source data (typically sub-sets from the measured time-steps and/or $\beta_{LM}$ parameter sub-sets), to further test the robustness of the results, and estimate uncertainties and/or improve upon them. Some of these possibilities are explored below, and further details may also be found in refs. \cite{hockett2018QMP1,hockett2018QMP2,marceau2017MolecularFrameReconstruction} (and references therein).


\subsubsection{Bootstrapping basics: dataset and fitting}

For the example case, synthetic data is used, inspired by ref. \cite{marceau2017MolecularFrameReconstruction}. In that case, experimental results were obtained via a (dual) pump-probe scheme, where the pump IR pulses prepared a rotational wavepacket in $N_2$, and a time-delayed XUV probe pulse ionized the sample via multiple channels. Here the focus is on the photoionization step, and investigating the reconstruction