\subsection{Bootstrapping to the MF example case}

As detailed elsewhere [TBD/refs], the ``bootstapping" methodology proceeds via a multi-stage fitting protocol to obtain both the AF axis distribution (level 1 of the bootstrap, linear fit), and the photoionization matrix elements (level 2 of the bootstrap, non-linear fit). A schematic overview is given in Figure XX. In the simplest case, this proceeds in a direct manner, and a final set of results are obtained. However, further bootstrapping - in the statistical sense - can also be employed to probe the fidelity of the reconstruction with different samples from the source data (typically sub-sets from the measured time-steps and/or $\beta_{LM}$ parameter sub-sets), to further test the robustness of the results, and estimate uncertainties and/or improve upon them. Some of these possibilities are explored below, and further details may also be found in refs. \cite{hockett2018QMP1,hockett2018QMP2,marceau2017MolecularFrameReconstruction} (and references therein).


\subsubsection{Bootstrapping basics: dataset and setup}

For the example case, synthetic data was used, although the dataset was inspired by ref. \cite{marceau2017MolecularFrameReconstruction}. In that case, experimental results were obtained via a (dual) pump-probe scheme, where the IR pump pulses prepared a rotational wavepacket in $N_2$, and a time-delayed XUV probe pulse ionized the sample - this scheme is illustrated in Fig. \ref{781808}. Multiple channels were probed in this manner, for a range of time-delays, to provide a dataset of AF-$\beta_{LM}(E,t)$ parameters. Here the focus is on the photoionization step, information content and further investigation of the retrieval routines; this is facilitated by the following choices and data:

\begin{enumerate}
\item The same rotational wavepacket \& molecular axis distribution as obtained experimentally is assumed. Consequently, the 1st (linear) stage of the bootstrapping protocol is not investigated herein.\footnote{Although this may seem like a drastic omission, in general this stage of the methodology is expected to be quite robust, and this has been demonstrated in various investigations of rotational wavepackets and molecular alignment techniques. [REFS]}
\item To simulate the observables, photoionization matrix elements from ePolyScat \cite{Lucchese1986,Gianturco1994,Natalense1999,luccheseEPolyScatUserManual} calculations are used [REFS: N2 ePSdata]. In the example illustated herein, matrix elements for $E_{ke}=1~eV$ were used, and are given explicitly in Table \ref{tab:inputMatE}. This, naturally, also provides a means for direct comparison and fidelity analysis of the retrieval protocol.
\item To investigate limitations of the numerical routines, noise and other artiefacts can be added to the simulated data; different sub-sets of the data can also be readily analysed and compared.
\item Finally, it is of note that the numerical implementations are structured as a set of tensors, as close as possible to the formalism given above [TODO - should be in theory sect above]. This provides a means to further investigate the information content of various parts of the problem, and investigate their influence on the retrieval - in particular, these can indicate aspects of the data which may be most sensitive to particular matrix elements, most susceptible to noise and so forth. The numerics are currently implemented using the Xarray python library [REFS].
\end{enumerate}

The sample dataset used for the results presented herein is illustrated in Figure \ref{952933} and, as mentioned previously, full numerical data can be accessed online [REF]. In this case only a 1~ps subset of the simulation data was used, over the main revival feature. Figure \ref{952933} shows both the full simulation results and the sub-selected points (13 data points) with random noise added (up to 10\%), the latter is used as the input dataset for matrix element retrieval in the following sections. [May want to show more simulated data here?]


