\subsection{Theoretical techniques \& analysis methodologies}

To cover 

\begin{itemize}
\item Theory sophistication and availability
\item Data analysis techniques
\item Development of new formalism/theory, relation to phase retrieval problems, etc.
\end{itemize}

A number of authors have treated MFPADs and related problems, see Appendix XX for some examples [TODO]. Herein, a geometric tensor based formalism is developed, which is close in spirit to the treatments given by Underwood and co-workers [REFS], but further separates various sets of physical parameters into dedicated tensors; this allows for a unified theoretical and numerical treatment, where the latter computes properties as tensor variables which can be further manipulated and investigated.\footnote{This is in contradistinction to standard numerical routines in which the requisite terms are usually computed from vectorial and/or nested summations, which can be somewhat opaque to detailed interpretation. It should be emphasized, however, that the underlying quantities are essentially identical in all these approaches, with a set of coupled angular-momenta defin only the manner of computation and notation varies.} Furthermore, the tensors can readily be converted to a density matrix representation [BLUM, ZARE], which is more natural for some quantities, and also emphasizes the link to quantum state tomography and other quantum information techniques. Much of the theoretical background, as well as application to aspects of the current problem, can be found in the textbooks of Blum [REF] and Zare [REF].