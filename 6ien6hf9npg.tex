\subsection{Theoretical techniques \& analysis methodologies}

To cover 

\begin{itemize}
\item Theory sophistication and availability
\item Data analysis techniques
\item Development of new formalism/theory, relation to phase retrieval problems, etc.
\end{itemize}

\subsubsection{Tensor formulation of photoionization}

A number of authors have treated MFPADs and related problems, see Appendix \ref{sec:theory-lit} for some examples. Herein, a geometric tensor based formalism is developed, which is close in spirit to the treatments given by Underwood and co-workers \cite{Reid2000, Stolow2008, Underwood2000}, but further separates various sets of physical parameters into dedicated tensors; this allows for a unified theoretical and numerical treatment, where the latter computes properties as tensor variables which can be further manipulated and investigated.%\footnote{It should be emphasized, however, that the underlying physical quantities are essentially identical in all these approaches, with a set of coupled angular-momenta defining the geometrical part of the photoionization problem, despite these differences in the details of the theory and notation. The current tensorial numerical implementation is in contradistinction to standard numerical routines in which the requisite terms are usually computed from vectorial and/or nested summations, which can be somewhat opaque to detailed interpretation, and typically implement the full computation of the observables in one computational routine. The main cost of a tensor-based approach is that more RAM is required to store the full set of tensor variables.} 
Furthermore, the tensors can readily be converted to a density matrix representation [BLUM, ZARE], which is more natural for some quantities, and also emphasizes the link to quantum state tomography and other quantum information techniques. Much of the theoretical background, as well as application to aspects of the current problem, can be found in the textbooks of Blum [REF] and Zare [REF].

It should be emphasized, however, that the underlying physical quantities are essentially identical in all these approaches, with a set of coupled angular-momenta defining the geometrical part of the photoionization problem, despite these differences in the details of the theory and notation. The current tensorial numerical implementation is in contradistinction to standard numerical routines in which the requisite terms are usually computed from vectorial and/or nested summations, which can be somewhat opaque to detailed interpretation, and typically implement the full computation of the observables in one computational routine. The main cost of a tensor-based approach is that more RAM is required to store the full set of tensor variables.

\begin{eqnarray}
\beta_{L,-M}^{\mu_{i},\mu_{f}}(\epsilon) & = & (-1)^{M}\sum_{P,R',R}(2P+1)^{\frac{1}{2}}{E_{P-R}(\hat{e};\mu_{0})}\\
 & \times &\sum_{l,m,\mu}\sum_{l',m',\mu'}(-1)^{(\mu'-\mu_{0})}{\Lambda_{R',R}(R_{\hat{n}};\mu,P,R,R')B_{L,-M}(l,l',m,m')}\\
 & \times & I_{l,m,\mu}^{p_{i}\mu_{i},p_{f}\mu_{f}}(\epsilon)I_{l',m',\mu'}^{p_{i}\mu_{i},p_{f}\mu_{f}*}(E)
\end{eqnarray}

[Formalism from 
\href{https://epsproc.readthedocs.io/en/dev/methods/ePSproc_geom_methods_summary_190821-v1-tidy.html}{ePSproc notes}


Also need to develop language a bit + radial matrix elements > Rlmlam form and tensor.

Add appendix with definition of each tensor.]


\subsubsection{Information content \& channel functions}

- Experimental information content.
- Channel fns. as geometric tensor multiplication.


\subsubsection{Retrieval techniques}

- General notes on fitting? Some already below, and may stay there.
- Matrix inversion method overview?
