\subsection{Theoretical techniques \& analysis methodologies}

To cover 

\begin{itemize}
\item Theory sophistication and availability
\item Data analysis techniques
\item Development of new formalism/theory, relation to phase retrieval problems, etc.
\end{itemize}

\subsubsection{Tensor formulation of photoionization}

A number of authors have treated MFPADs and related problems, see Appendix \ref{sec:theory-lit} for some examples. Herein, a geometric tensor based formalism is developed, which is close in spirit to the treatments given by Underwood and co-workers \cite{Reid2000, Stolow2008, Underwood2000}, but further separates various sets of physical parameters into dedicated tensors; this allows for a unified theoretical and numerical treatment, where the latter computes properties as tensor variables which can be further manipulated and investigated.%\footnote{It should be emphasized, however, that the underlying physical quantities are essentially identical in all these approaches, with a set of coupled angular-momenta defining the geometrical part of the photoionization problem, despite these differences in the details of the theory and notation. The current tensorial numerical implementation is in contradistinction to standard numerical routines in which the requisite terms are usually computed from vectorial and/or nested summations, which can be somewhat opaque to detailed interpretation, and typically implement the full computation of the observables in one computational routine. The main cost of a tensor-based approach is that more RAM is required to store the full set of tensor variables.} 
Furthermore, the tensors can readily be converted to a density matrix representation \cite{BlumDensityMat, zareAngMom}, which is more natural for some quantities, and also emphasizes the link to quantum state tomography and other quantum information techniques. Much of the theoretical background, as well as application to aspects of the current problem, can be found in the textbooks of Blum \cite{BlumDensityMat} and Zare \cite{zareAngMom}.

Within this treatment, the MF observables can be defined as:

\begin{eqnarray}
\beta_{L,-M}^{\mu_{i},\mu_{f}}(\epsilon) & = & (-1)^{M}\sum_{P,R',R}(2P+1)^{\frac{1}{2}}{E_{P-R}(\hat{e};\mu_{0})}\\
 & \times &\sum_{l,m,\mu}\sum_{l',m',\mu'}(-1)^{(\mu'-\mu_{0})}{\Lambda_{R',R}(R_{\hat{n}};\mu,P,R,R')B_{L,-M}(l,l',m,m')}\\
 & \times & I_{l,m,\mu}^{p_{i}\mu_{i},p_{f}\mu_{f}}(\epsilon)I_{l',m',\mu'}^{p_{i}\mu_{i},p_{f}\mu_{f}*}(\epsilon)\label{eq:BLM-tensor-MF}
\end{eqnarray}

And the LF/AF as:

\begin{eqnarray}
\bar{\beta}_{L,-M}^{\mu_{i},\mu_{f}}(E,t) & = & (-1)^{M}\sum_{P,R',R}{[P]^{\frac{1}{2}}}{E_{P-R}(\hat{e};\mu_{0})}\\
 & \times &\sum_{l,m,\mu}\sum_{l',m',\mu'}(-1)^{(\mu'-\mu_{0})}{\Lambda_{R'}(\mu,P,R')B_{L,S-R'}(l,l',m,m')}\\
 & \times &I_{l,m,\mu}^{p_{i}\mu_{i},p_{f}\mu_{f}}(E)I_{l',m',\mu'}^{p_{i}\mu_{i},p_{f}\mu_{f}*}(E)\sum_{K,Q,S}\Delta_{L,M}(K,Q,S)A_{Q,S}^{K}(t)\label{eq:BLM-tensor-AF}
\end{eqnarray}

In both cases a set of geometric tensor terms are required, which are fully defined in Appendix \ref{appendix:formalism}; these terms provide details of:

\begin{itemize}
\item ${E_{P-R}(\hat{e};\mu_{0})}$: polarization geometry \& coupling with the electric field.
\item $B_{L,S-R'}(l,l',m,m')$: geometric coupling of the partial waves into the $\beta_{L,M}$ terms (spherical tensors).
\item $\Lambda_{R'}(\mu,P,R')$: frame couplings and rotations.
\item $\Delta_{L,M}(K,Q,S)$: alignment frame coupling.
\item $A_{Q,S}^{K}(t)$: ensemble alignment described as a set of axis distribution moments (ADMs).
\end{itemize}

And \(I_{l,m,\mu}^{p_{i}\mu_{i},p_{f}\mu_{f}}(\epsilon)\) are the (radial) dipole ionization matrix elements, as a function of energy \(\epsilon\). These matrix elements are essentially identical to the simplified forms $r_{k,l,m}$ defined in Eqn. \ref{eq:r-kllam}, except with additional indices to label symmetry and polarization components
defined by a set of partial-waves \(\{l,m\}\), for polarization component \(\mu\) (denoting the photon angular momentum components) and channels (symmetries) labelled by initial and final state indexes \({p_{i}\mu_{i},p_{f}\mu_{f}}\). The notation here follows that used by ePolyScat, and these matrix elements again represent the quantities  to be obtained numerically from data analysis, or from an \href{https://epsproc.readthedocs.io/en/latest/ePS_ePSproc_tutorial/ePS_tutorial_080520.html\#Theoretical-background}{ePolyScat (or similar) calculation}. 

Note that, in this case as given, time-dependence arises purely from the \(A_{Q,S}^{K}(t)\) terms in the AF case, and the electric field term currently describes only the photon angular momentum coupling,
although can in principle also describe time-dependent/shaped fields. Similarly, a time-dependent initial state (e.g.~vibrational wavepacket) could also describe a time-dependent MF case.

It should be emphasized, however, that the underlying physical quantities are essentially identical in all the theoretical approaches, with a set of coupled angular-momenta defining the geometrical part of the photoionization problem, despite these differences in the details of the theory and notation. 


\subsubsection{Channel functions \& density matrix representation}

A further simplification can be made, to provide the observables in terms of ``channel functions", which define the ionization continuum for a given set of parameters $u$ (e.g. defined by an experimental configuration),

\begin{equation}
\beta_{L,M}^{u}=\sum_{\zeta,\zeta'}\varUpsilon_{L,M}^{u,\zeta\zeta'}\mathbb{I}^{\zeta\zeta'}\label{eqn:channel-fns}
\end{equation}

% \sum_{\zeta,\zeta'}\varUpsilon_{L,M}^{u,\zeta\zeta'}\mathbb{I}_{\zeta\zeta'}^{\Gamma,\Gamma'}=
% \equiv\sum_{\zeta,\zeta'}\varUpsilon_{L,M}^{u,\zeta\zeta'}\mathbf{\rho}^{\zeta\zeta'}

Where $\zeta,\zeta'$ collect all the required quantum numbers, and define all (coherent) pairs of components. The term \mathbb{I}^{\zeta\zeta'} denotes the coherent square of the ionization matrix elements:

\begin{equation}
\mathbb{I}^{\zeta,\zeta'}=I^{\zeta}(\epsilon)I^{\zeta'*}(\epsilon)
\end{equation}

This is effectively a convolution equation (cf. refs. \cite{Reid2000,gregory2021MolecularFramePhotoelectron}) with channel functions, for a given ``experiment'' $u$, summed over all terms $\zeta,\zeta'$.

This tensorial form is numerically implemented in the ePSproc codebase \cite{ePSprocGithub}, and is in contradistinction to standard numerical routines in which the requisite terms are usually computed from vectorial and/or nested summations, which can be somewhat opaque to detailed interpretation, and typically implement the full computation of the observables in one computational routine. The PEMtk codebase \cite{hockett2021PhotoelectronMetrologyToolkit} implements matrix element retrieval based on this formalism, with pre-computation of all the geometric tensor components (channel functions) prior to a fitting protocol for matrix element analysis, essentially a fit to Eqn. \ref{eqn:channel-fns}, with terms $I^{\zeta}(\epsilon)$ as the unknowns. The main cost of a tensor-based approach is that more RAM is required to store the full set of tensor variables.

Following this notation, it is also trivial to write the radial matrix elements in density matrix form in the $\zeta\zeta'$ representation:

\begin{equation}
\mathbf{\rho}^{\zeta\zeta'} = |\zeta\rangle\langle\zeta'| \equiv \mathbb{I}^{\zeta,\zeta'}
\end{equation}

And the full final state as a density matrix in the $\zeta\zeta'$ representation:

\begin{equation}
\mathbf{\bar{\rho}}_{L,M}^{\zeta\zeta'}=\varUpsilon_{L,M}^{u,\zeta\zeta'}\mathbb{I}^{\zeta,\zeta'}
\end{equation}

Here the density matrix can be interpreted as the final, LF/AF or
MF density matrix, incorporating both the intrinsic and extrinsic
effects (i.e. all channel couplings and radial matrix elements for
the given measurement), with dimensions dependent on the unique sets of quantum numbers required (in the simplest case, this will just be a set of partial waves $\zeta = (l,m)$). The $L,M$ notation indicates here that these dimensions should not be summed over, hence the tensor coupling into the $\beta_{L,M}^{u}$ parameters can also be written in terms of the density matrix:

\begin{equation}
\beta_{L,M}^{u}=\sum_{\zeta,\zeta'}\mathbf{\bar{\rho}}_{L,M}^{\zeta\zeta'}
\end{equation}

In fact, this form arises naturally since the $\beta_{L,M}^{u}$ terms are the state multipoles (geometric tensors) defining the system, which can be thought of as a coupled basis equivalent of the density matrix representations (see, e.g., ref. \cite{BlumDensityMat}, Chpt. 4.).

TODO: confirm notation here, may want to include more discussion from Blum? (See Fig. \ref{998904} for results, should also confirm representation here from outer product case, maybe plot LM representation too - see \href{https://epsproc.readthedocs.io/en/dev/methods/density_mat_notes_demo_300821.html}{ePSproc density matrix notes} esp. \href{https://epsproc.readthedocs.io/en/dev/methods/density_mat_notes_demo_300821.html#Density-matrix-from-geometric-tensors}{Density matrix from geometric tensors section}.)


\subsubsection{Information content}



% [Formalism from 
% \href{https://epsproc.readthedocs.io/en/dev/methods/ePSproc_geom_methods_summary_190821-v1-tidy.html}{ePSproc notes}


% Also need to develop language a bit + radial matrix elements > Rlmlam form and tensor.

% Add appendix with definition of each tensor.]




As discussed in ref. \cite{hockett2018QMP2}
- Experimental information content.
- Channel fns. as geometric tensor multiplication.


\subsubsection{Retrieval techniques}

- General notes on fitting? Some already below, and may stay there.
- Matrix inversion method overview?
