\subsection{Theoretical techniques \& analysis methodologies}

To cover 

\begin{itemize}
\item Theory sophistication and availability
\item Data analysis techniques
\item Development of new formalism/theory, relation to phase retrieval problems, etc.
\end{itemize}

\subsubsection{Tensor formulation of photoionization}

A number of authors have treated MFPADs and related problems, see Appendix \ref{sec:theory-lit} for some examples. Herein, a geometric tensor based formalism is developed, which is close in spirit to the treatments given by Underwood and co-workers \cite{Reid2000, Stolow2008, Underwood2000}, but further separates various sets of physical parameters into dedicated tensors; this allows for a unified theoretical and numerical treatment, where the latter computes properties as tensor variables which can be further manipulated and investigated.%\footnote{It should be emphasized, however, that the underlying physical quantities are essentially identical in all these approaches, with a set of coupled angular-momenta defining the geometrical part of the photoionization problem, despite these differences in the details of the theory and notation. The current tensorial numerical implementation is in contradistinction to standard numerical routines in which the requisite terms are usually computed from vectorial and/or nested summations, which can be somewhat opaque to detailed interpretation, and typically implement the full computation of the observables in one computational routine. The main cost of a tensor-based approach is that more RAM is required to store the full set of tensor variables.} 
Furthermore, the tensors can readily be converted to a density matrix representation \cite{BlumDensityMat, zareAngMom}, which is more natural for some quantities, and also emphasizes the link to quantum state tomography and other quantum information techniques. Much of the theoretical background, as well as application to aspects of the current problem, can be found in the textbooks of Blum \cite{BlumDensityMat} and Zare \cite{zareAngMom}.

Within this treatment, the MF observables can be defined as:

\begin{eqnarray}
\beta_{L,-M}^{\mu_{i},\mu_{f}}(\epsilon) & = & (-1)^{M}\sum_{P,R',R}(2P+1)^{\frac{1}{2}}{E_{P-R}(\hat{e};\mu_{0})}\\
 & \times &\sum_{l,m,\mu}\sum_{l',m',\mu'}(-1)^{(\mu'-\mu_{0})}{\Lambda_{R',R}(R_{\hat{n}};\mu,P,R,R')B_{L,-M}(l,l',m,m')}\\
 & \times & I_{l,m,\mu}^{p_{i}\mu_{i},p_{f}\mu_{f}}(\epsilon)I_{l',m',\mu'}^{p_{i}\mu_{i},p_{f}\mu_{f}*}(\epsilon)\label{eq:BLM-tensor-MF}
\end{eqnarray}

And the LF/AF as:

\begin{eqnarray}
\bar{\beta}_{L,-M}^{\mu_{i},\mu_{f}}(E,t) & =(-1)^{M} & \sum_{P,R',R}{[P]^{\frac{1}{2}}}{E_{P-R}(\hat{e};\mu_{0})}\sum_{l,m,\mu}\sum_{l',m',\mu'}(-1)^{(\mu'-\mu_{0})}{\Lambda_{R'}(\mu,P,R')B_{L,S-R'}(l,l',m,m')}I_{l,m,\mu}^{p_{i}\mu_{i},p_{f}\mu_{f}}(E)I_{l',m',\mu'}^{p_{i}\mu_{i},p_{f}\mu_{f}*}(E)\sum_{K,Q,S}\Delta_{L,M}(K,Q,S)A_{Q,S}^{K}(t)\label{eq:BLM-tensor-AF}
\end{eqnarray}

Where \(I_{l,m,\mu}^{p_{i}\mu_{i},p_{f}\mu_{f}}(\epsilon)\) are the (radial) dipole ionization matrix elements, as a function of energy \(\epsilon\). These matrix elements are essentially identical to the simplified forms $r_{k,l,m}$ defined in Eqn. \ref{eq:r-kllam}, except with additional indices to label symmetry and polarization components
defined by a set of partial-waves \(\{l,m\}\), for polarization component \(\mu\) (denoting the photon angular momentum components) and channels (symmetries) labelled by initial and final state indexes \({p_{i}\mu_{i},p_{f}\mu_{f}}\). The notation here follows that used by ePolyScat, and these matrix elements again represent the quantities  to be obtained numerically from data analysis, or from an \href{https://epsproc.readthedocs.io/en/latest/ePS_ePSproc_tutorial/ePS_tutorial_080520.html\#Theoretical-background}{ePolyScat (or similar) calculation}. 

In both cases a set of geometric tensor terms are required, which are fully defined in Appendix \ref{appendix:formalism}; these terms provide details of 

\begin{itemize}
\item ${E_{P-R}(\hat{e};\mu_{0})}$: polarization geometry \& coupling with the electric field.
\item $B_{L,S-R'}(l,l',m,m')}$: geometric coupling of the partial waves into the $\beta_{L,M}$ terms (spherical tensors).
\item $\Delta_{L,M}(K,Q,S)$: frame couplings and rotations.
\item $A_{Q,S}^{K}(t)$: 
\end{itemize}



Note that, in this case, time-dependence arises purely from the
\(A_{Q,S}^{K}(t)\) terms in the AF case, and the electric field term
currently describes only the photon angular momentum coupling,
time-dependent/shaped fields are not yet supported (as of v1.3.0, but
will be soon). Similarly, a time-dependent initial state
(e.g.~vibrational wavepacket) could also describe a time-dependent MF
case, but is currently not included here.

It should be emphasized, however, that the underlying physical quantities are essentially identical in all these approaches, with a set of coupled angular-momenta defining the geometrical part of the photoionization problem, despite these differences in the details of the theory and notation. The current tensorial numerical implementation is in contradistinction to standard numerical routines in which the requisite terms are usually computed from vectorial and/or nested summations, which can be somewhat opaque to detailed interpretation, and typically implement the full computation of the observables in one computational routine. The main cost of a tensor-based approach is that more RAM is required to store the full set of tensor variables.

[Formalism from 
\href{https://epsproc.readthedocs.io/en/dev/methods/ePSproc_geom_methods_summary_190821-v1-tidy.html}{ePSproc notes}


Also need to develop language a bit + radial matrix elements > Rlmlam form and tensor.

Add appendix with definition of each tensor.]


\subsubsection{Information content \& channel functions}

- Experimental information content.
- Channel fns. as geometric tensor multiplication.


\subsubsection{Retrieval techniques}

- General notes on fitting? Some already below, and may stay there.
- Matrix inversion method overview?
