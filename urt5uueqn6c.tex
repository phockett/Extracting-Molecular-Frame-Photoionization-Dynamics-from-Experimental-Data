% *** Now above, and skip noise-free case
% \subsubsection{Bootstrapping basics: noisy data}

% To investigate the effects of noise, fitting was repeated for the same set of data-points, but with increasing amounts of white/pink-noise added.

% TODO





% \subsubsection{Further bootstrapping: sub-sample size \& fiedelity}

% In cases with noisy data, and/or where uncertainties in fitted parameters remain large, fitting to larger and/or alternative sub-sets of the data may be pursued.

% TODO

\subsubsection{Further bootstrapping: sub-sample size \& fidelity}

In cases with noisy data, and/or where uncertainties in fitted parameters remain large, fitting to larger and/or alternative sub-sets of the data may be pursued.
An obvious route to improvement of the retrieved matrix elements is via the inclusion of additional data points. In the original experimental demonstration \cite{marceau2017MolecularFrameReconstruction}, after an initial 10-point fit, further data was included, up to 89 data points, to minimise uncertainties in the retrieved matrix elements.\footnote{This procedure was only tested for the X-state in the experimental case, not for the A or B-state datasets. %See ref. \cite{marceau2017MolecularFrameReconstruction}
and especially the supplementary materials for more details.} Another, similar, route is via traditional statistical bootstrapping, in which fitting is tested for various randomly-selected sub-sets of the data (but the size of the fitted data set is not varied). Additionally, if the channel functions (fitting basis set) are investigated (see Sect. \ref{sec:bootstrapping-info-sensitivity}), additional data points may be selected to enhance the sensitivity to certain partial wave components.

In the current case, work is ongoing, %with the PEMtk codebase, 
but some effort has been made to investigate sampling strategies. In particular, preliminary work using different sample sizes plus additional statistical weightings (statistical bootstrapping) has yielded very promising results. For example, a dataset of 30 t-points equally spaced over the range 3.5~-~4.5~ps (as compared to 4~-~5~ps in the previous demonstration above), with random (Poissionian) statistical weightings additionally applied, yielded:

\begin{itemize}
\item 30~\% convergence (best $\chi^2$) in 100 test fits.
\item Standard deviations on the retrieved parameters of $<10^{-4}$ (cf. Table \ref{tab:matE}).
\item Fideleities on retrieved parameters typically $<10\%$.
\end{itemize}

However, whilst this appears promising, it is of note that the trends in the fidelity of the retrieved matrix element remained similar to those shown in Table \ref{tab:matE}, with the $l=3$ cases less well-defined. This indicates that a more fundamental information-content limitation remained. Additionally, other tests with the same initial dataset but different statistical weightings did not yield the same high percentage of converged fits with low standard deviation (hence retrieval quality), indicating that more careful and methodical work is required here, perhaps with more sophisticated statistical techniques applied.





\subsubsection{Bootstrapping technique outlook \& future directions}

In the current case, it is clear that the bootstrapping methodology for obtaining full photoionization matrix elements from time-domain, AF datasets, works well. The current numerical routines in the PEMtk package (and back-end libraries) are relatively stable and fast, and amenable to detailed inspection. However, a range of outstanding questions and routes of investigation remain, for instance:

\begin{itemize}
\item Scaling to larger problems (larger molecules, more matrix elements). This is currently under investigation, but - based on previous work \cite{hockett2009RotationallyResolvedPhotoelectron, hockett2018QMP2} - it is anticipated that small polyatomic molecules should be tractable to the basic approach. 
\item More sophisticated approaches, for instance careful sub-selection of data based on the channel functions, data obtained for additional polarization geometries or with shaped laser pulses \cite{hockett2014CompletePhotoionizationExperiments, hockett2015CoherentControlPhotoelectron, hockett2015CompletePhotoionizationExperiments,hockett2018QMP1}. Such approaches may be expected to yield higher fidelity reconstructions for small polyatomic systems, and may be required for more complex cases.
\item Faster numerical routines, in particular via GPU-based numerics.
\item Implementation of additional fitting routines, both from standard numerical methods, and also from related specialised domain problems, for example phase-retrieval methods developed for optical interferograms and spectrograms (e.g. general FROG-type retrieval methods \cite{trebino2000FrequencyResolvedOpticalGating}, ptychography/holographic techniques \cite{Spangenberg2015b, Spangenberg2015c}) and homotopy \cite{Sommese2005} approaches may be applicable.
\item Correlations and overlaps with other related/emerging photoionization techniques may also prove fruitful. For instance, photoionization matrix elements are of interest in high-harmonic spectroscopy schemes \cite{Lock2012}, angle-resolved RABBITT \cite{hockett2017AngleresolvedRABBITTTheory,villeneuve2017CoherentImagingAttosecond} and general attosecond ``clocking" and time-delay measurement techniques. In many cases these methods are also directly sensitive to the energy-dependence of the matrix elements, thus providing additional information relative to a basic 1-photon ionization study (albeit with additional complexity), and have recently been of great theoretical interest in the \textit{ab initio} photoionization community \cite{Feist2014,benda2022AnalysisRABITTTime}.

\end{itemize}

% In future it will also be interesting to consider 
Machine learning (ML), and particularly recent ``deep learning" techniques (often also generically referred to as AI), are currently in vogue and evolving rapidly. Such approaches may also present interesting opportunities for MF retrieval problems. Perhaps the main strength of these methods is that, in favourable cases, very complex problems may be treated without complete computation and/or understanding of the underlying physics (e.g. the recent success of DeepMind/AlphaFold in protein folding \cite{eisenstein2021ArtificialIntelligencePowers,jumper2021HighlyAccurateProtein}); the use of ML/AI may, therefore, be particularly interesting for cases which are otherwise intractable to a full \textit{ab initio} analysis, e.g. complicated molecular dynamics or strongly-coupled light-matter systems, where the step-wise and separable approaches detailed herein will currently fail, and closed-form equations/solutions do not exist. A secondary use may be as fast fitting algorithms for cases of the type discussed herein, where the physics is understood, albeit complicated. This is much less interesting scientifically since it does not present a ``new" capability, although may still prove fruitful if the increase in speed (or robustness) is significant compared to current methods, e.g. to allow for real-time analysis during experimental runs. Furthermore, use in this vein, but with the aim of solving a fully coupled problem without the necessity of a bootstrapping or multi-step type of analysis (i.e. combining the various stages illustrated in Figs. \ref{807606}, \ref{731792} into a single ML-based routine) is certainly worth pursuing, with the potential to create a method that simultaneously retrieves both the alignment and photoelectron properties from experimental data without additional researcher intervention. The difficulty in all cases will likely be the generation of a suitable data set for the ML/AI training procedure, which is required before the routines can be deployed on new problems. Significant introductory and general discussion to the (rapidly evolving) topic can be found online and, for example, in Refs. \cite{LeCun2015,carleo2019MachineLearningPhysical,davies2021AdvancingMathematicsGuiding}. Some recent use of ML/AI in the AMO context has proved successful, e.g. Refs. \cite{he2022FilmingMoviesAttosecond,hegazy2022BayesianInferencingDeterministic} tackle specific data-analysis and reconstruction type problems; Refs. \cite{lu2022FastInitioPotential,nandi2021DmachineLearningPotential} examine ML for \textit{ab initio} computation of potential energy surfaces; Ref. \cite{carleo2019MachineLearningPhysical} provides a broad review of ML in the physical sciences, including quantum state reconstruction, as well as particle physics, cosmology and materials science.
%; significant introductory and general discussion to the (rapidly evolving) topic can be found online and, for example, in Refs. \cite{LeCun2015,davies2021AdvancingMathematicsGuiding}.

% TO add: protein folding refs, ref Isaac for quantum stuff? More outlook? Also ref citation, https://www.nature.com/articles/s41467-022-32313-0
% Per VM: https://www.nature.com/articles/s41586-021-04086-x
% Robustness as well as speed to mention?
% 09/03/23: added some more notes and refs, now completed.
