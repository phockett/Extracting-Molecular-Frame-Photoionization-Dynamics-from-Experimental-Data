\subsubsection{Bootstrapping basics: noisy data}

To investigate the effects of noise, fitting was repeated for the same set of data-points, but with increasing amounts of white/pink-noise added.

TODO

\subsubsection{Further bootstrapping: sub-sample size \& fiedelity}

In cases with noisy data, and/or where uncertainties in fitted parameters remain large, fitting to larger and/or alternative sub-sets of the data may be pursued.

TODO

\subsubsection{Further bootstrapping: sub-sample size \& fiedelity}

An obvious route to improvement of the retrieved matrix elements is via the inclusion of additional data points. In the original demonstration \cite{marceau2017MolecularFrameReconstruction}, after an initial 10-point fit, further data was included up to 89 $\beta_{LM}(t)$ [CHECK] to minimise uncertainties in the retrieved matrix elements.\footnote{This procedure was only tested for the X-state in the experimental case, not for the A or B-state datasets, %see ref. \cite{marceau2017MolecularFrameReconstruction}
and especially the supplementary materials for more details.} Another, similar, route is via traditional statistical bootstrapping, in which fitting is tested for various randomly-selected sub-sets of the data (but the size of the fitted data set is not varied). Additionally, if the channel functions (fitting basis set) are investigate (see Sect. XX below), additional data points may be selected to enhance the sensitivity to certain partial wave components.




\subsubsection{Further bootstrapping: information content \& sensitivity}

As well as considering the results from full fits of the data, the inherent sensitivity of various aspects of the problem can also be investigated. In general, this will depend on the details of the problem at hand (symmetry, ADMs etc.), but can in essence be considered independently of the matrix elements themselves via ``channel functions" or equivalent [WILL BE DISCUSSED ABOVE? ALSO DENSITY MAT?]. In the PEMtk routines, the various component tensors are computed and packaged as a basis set prior to fitting, and can be further examined independently.


\subsubsection{Bootstrapping technique outlook \& future directions}

In the current case, it is clear that the bootstrapping methodology for obtaining full photoionization matrix elements from time-domain, AF datasets, works well. The current numerical routines in the PEMtk package (and back-end libraries) are relatively stable and fast, and amenable to detailed inspection. However, a range of outstanding questions and routes of investigation remain, for instance:

\begin{itemize}
\item Scaling to larger problems (larger molecules, more matrix elements). This is currently under investigation, but - based on previous work [REF] - it is anticipated that small polyatomic molecules should be tractable to the basic approach. 
\item More sophisticated approaches, for instance careful sub-selection of data based on the channel functions, data obtained for additional polarization geometries or with shaped laser pulses [REF]. Such approaches may be expected to yield higher fidelity reconstructions for small polyatomic systems, and may be required for more complex cases.
\item Faster numerical routines, in particular via GPU-based numerics.
\item Implementation of additional fitting routines, both from standard numerical methods (EG...), and also from related specialised domain problems, for example FROG-type retrieval methods and homotophy approaches. [REFS] 
\end{itemize}
