\subsubsection{Bootstrapping basics: density matrix representation}

An alternative test of fidelity can be investigated via a density matrix representation of the results, this is shown in Fig. \ref{998904}. Overall the agreement is good for the real values (righthand column of Fig. \ref{998904}), as expected from the values in Table \ref{tab:matE}, and the differences (bottom row) are generally $<10~\%$. However, for the imaginary values (middle column) the loss of the sign of the phases leads to larger differences in the off-diagonal density matrix elements, and the possibility of phase flips, which is immediately evident from the inverted patterns visible in this case (top middle and middle plots). The absolute value plots do not indicate the sign difference directly, but it is reflected in larger differences observed in these plots (differences $>>10~\%$), although the absolute values and observed pattern for the retrieved matrix elements compares well with the reference case. (In this case, the differences were obtained from the complex-valued density matricies)

Perhaps more interesting/useful in the density matrix representation is the visualisation of the phase relations between the matrix elements (off-diagonal elements), and the ability to quickly check the overall pattern of the matrix elements, and check that no phase-relations are missing, or that very different patterns/sets of matrix elements are retrieved. Furthermore, the density matrix elements also provide a complete description of the photoionization event, and can be used as the starting point for further analysis [refs - Blum?].

TODO: more to say, other metrics for fidelity?
