\subsubsection{Further bootstrapping: information content \& sensitivity}

As well as considering the results from full fits of the data, the inherent sensitivity of various aspects of the problem can also be investigated. In general, this will depend on the details of the problem at hand (symmetry, ADMs etc.), but can in essence be considered independently of the matrix elements themselves via ``channel functions" or equivalent [WILL BE DISCUSSED ABOVE? ALSO DENSITY MAT?]. In the PEMtk routines, the various component tensors are computed and packaged as a basis set prior to fitting, and can be further examined independently. 

For full details see \href{https://pemtk.readthedocs.io/en/latest/fitting/PEMtk_fitting_basis-set_demo_050621-full.html}{the PEMtk docs}.

Fig. \ref{776753} illustrates the geometric coupling term $B_{L,M}$, hence the sensitivity of different (L,M) terms to the matrix element products. This term incorporates the coupling of the partial wave pairs, $|l,m\rangle$ and $|l',m'\rangle$, into the term $B_{L,M}$, where $\{L,M\}$ are observable total angular momenta, and $M = S-R_{p}$ ([full definition here](https://epsproc.readthedocs.io/en/dev/methods/geometric_method_dev_260220_090420_tidy.html#B_{L,M}-term)), hence indicates which terms are allowed for a given set of partial waves - in the current test case, $l=1,3$ only (as defined by the known matrix elements). This is essentially a way to visualize the general selection rules into the observable: for instance, only terms $l=l'$ and $m=-m'$ contribute to the overall photoinoization cross-section term ($L=0, M=0$). However, since these terms are fairly simply followed algebraically in this case, via the rules inherent in the 3j product, this is not particularly insightful. These visualizations will become more useful when dealing with real sets of matrix elements, and specific polarization geometries, which will further modulate the $B_{L,M}$ terms - for example, in the current case with $S=0, R_p=0$ (this is typically the case for a cylindrical only $M=0$ terms will contribute.

