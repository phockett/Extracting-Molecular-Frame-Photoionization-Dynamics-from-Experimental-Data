\subsubsection{Further bootstrapping: information content \& sensitivity}

As well as considering the results from full fits of the data, the inherent sensitivity of various aspects of the problem can also be investigated. In general, this will depend on the details of the problem at hand (symmetry, ADMs etc.), but can in essence be considered independently of the matrix elements themselves via ``channel functions" or equivalent [WILL BE DISCUSSED ABOVE? ALSO DENSITY MAT?]. In the PEMtk routines, the various component tensors are computed and packaged as a basis set prior to fitting, and can be further examined independently. 

For full details see \href{https://pemtk.readthedocs.io/en/latest/fitting/PEMtk_fitting_basis-set_demo_050621-full.html}{the PEMtk docs}.

Fig. \ref{776753} illustrates the geometric coupling term $B_{L,M}$, hence the sensitivity of different (L,M) terms to the matrix element products. This term incorporates the coupling of the partial wave pairs, $|l,m\rangle$ and $|l',m'\rangle$, into the term $B_{L,M}$, where $\{L,M\}$ are observable total angular momenta, and $M = S-R_{p}$, 
% ([full definition here](https://epsproc.readthedocs.io/en/dev/methods/geometric_method_dev_260220_090420_tidy.html#B_{L,M}-term)), 
hence indicates which terms are allowed for a given set of partial waves - in the current test case, $l=1,3$ only (as defined by the known matrix elements). This is essentially a way to visualize the general selection rules into the observable: for instance, only terms $l=l'$ and $m=-m'$ contribute to the overall photoinoization cross-section term ($L=0, M=0$). However, since these terms are fairly simply followed algebraically in this case, via the rules inherent in the 3j product, this is not particularly insightful. These visualizations will become more useful when dealing with real sets of matrix elements, and specific polarization geometries, which will further modulate the $B_{L,M}$ terms - for example, in the current case with $S=0, R_p=0$ (this is typically the case for a cylindrically-symmetric experimental configuration) only $M=0$ terms will contribute.

Fig. \ref{652406} is more complicated, and illustrates the tensor product $\Lambda_{R}\otimes E_{PR}(\hat{e})\otimes \Delta_{L,M}(K,Q,S)\otimes A^{K}_{Q,S}(t)$, expanded over all quantum numbers.
%(see [full definition here](https://epsproc.readthedocs.io/en/dev/methods/geometric_method_dev_pt3_AFBLM_090620_010920_dev_bk100920.html#\beta_{L,M}^{AF}-rewrite)). 
This term, therefore, incorporates all of the dependence (or response) of the AF-$\beta_{LM}$s on the polarisation state, and the axis distribution. In this case, it's clear that there's a significant response to the alignment in the $L=0,2$ terms, some response in $L=4$ and - for the most part - no significant contribution from higher-order terms (threshold of 0.01), for the selected set of ADMs. This visualisation is potentially useful for planning measurements sensitive to certain properties, for example, in this case the $L=6$ term is significant only over a small $t$-range, so this region could be targeted experimentally to obtain data more sensitive to higher-order $l$-wave term couplings (per Fig. \ref{776753}\ref{776753}). Conversely, the $L=0,2$ response term is quite symmetric over the half-revival, so making experimental measurements at $t$-points symmetrically over this feature will provide redundant, but not additional, information content to the dataset for matrix-element retrieval.

Finally, Fig. \ref{676540}\ref{676540} illustrates the full response function, which is essentially the complete geometric basis set,
%([give or take a phase term or two](https://epsproc.readthedocs.io/en/dev/methods/geometric_method_dev_pt3_AFBLM_090620_010920_dev_bk100920.html#\beta_{L,M}^{AF}-rewrite)), 
hence equivalent to the AF-$\beta_{LM}$ if the ionization matrix elements were set to unity. This illustrates not only the coupling of the geometric terms into the observable $L,M$, but also how the partial wave $|l,m\rangle$ terms map to the observables. There's a lot of information here. A few observations in this case:

\begin{itemize}
\item The largest reponse is in the total cross-section ($\beta_{0,0}$), which is ~2 times larger than any other term; 
\item this response is similar for both the $l=1$ and $l=3$ contributions, hence we expect similar sensitivity to both partial cross-sections in this case.
\item For $L=2$ and $S-Rp = 0$, there is only a significant (above threshold) response around the centre of the selected time window, hence the main dip in the ADMs, corresponding to anti-alignment in this case.
\item The parameters indicate enhanced sensitivity to higher-order terms ($L=6$) just after the main revival feature. This was already apparent in Fig. \ref{652406}, and is correlated with the larger $K=4,6$ terms in the rotational this region (see Fig. \ref{720080})

\end{itemize}

TODO: More to say on terms $S-Rp != 0$? Check formalism - may have some inconsistency in $M$ definition here too?
