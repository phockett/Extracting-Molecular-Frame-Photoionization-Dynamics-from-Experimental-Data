% 12/07/22 - started, but ran out of steam - might skip it?
\subsection{Complete experiments literature sample\label{sec:CompleteLit}}

% 12/07/22 - started, but ran out of steam - might skip it? (Mainly pulling from QMet Vol. 2 Chpt. 8)

The topic of complete photoionization experiments has been recently reviewed in \cite{kleinpoppen2013perfect,hockett2018QMP2}, see also older review articles \cite{Becker1998,Reid2003,Kleinpoppen2005}. Some representative and noteworthy examples are also listed below (see also Sect. \ref{appendix:MF-expt} above for cases involving MF measurements).

\begin{itemize}
\item First experimental demonstration for photoionization of an atomic target by Berry and coworkers \cite{Duncanson1976}, who studied
$Na(^{2}P_{1/2},\,^{2}P_{3/2})$.
\item First molecular demonstration for $NO$, from the Zare group, with state-resolved LF measurements including linear and circularly polarized fields \cite{Reid1992} (see also prior work \cite{Allendorf1989,Leahy1991,Reid1991}).
\item General theoretical discussion on complete experiments in atoms and molecules \cite{Cherepkov2005}.
\item Time-resolved rotational-wavepacket methods applied to $NO$ (narrow wavepacket) \cite{Tsubouchi2004,Tang2010}, see also Refs. \cite{Suzuki2006} (review) and \cite{Suzuki2007} (theory \& analysis).
\item Non-linear polyatomic demonstration with state-resolved LF measurements ($NH_3$) \cite{hockett2009RotationallyResolvedPhotoelectron}.
\item Theoretical investigation of matrix element retrieval for photoelectron imaging experiments \cite{Ramakrishna2012} (see also \cite{Ramakrishna2013} for rotational wavepacket reconstruction from complementary methods including high-harmonic generation).  
\item Complete experiments with polarization shaping \cite{hockett2014CompletePhotoionizationExperiments,hockett2015CompletePhotoionizationExperiments}.
\item Time-resolved rotational-wavepacket method applied to $N_2$ (broad wavepacket) and bootstrap technique development \cite{marceau2017MolecularFrameReconstruction} (as per Sect. \ref{sec:bootstrapping} herein.)
\end{itemize} 