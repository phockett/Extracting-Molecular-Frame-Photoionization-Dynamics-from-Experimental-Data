\section{Resources\label{sec:resources}}

The following online resources are available, and it is hoped that interested readers will find these useful, and contribute.

\begin{itemize}
\item Manuscipt text, live \href{https://www.authorea.com/users/71114/articles/447808-extracting-molecular-frame-photoionization-dynamics-from-experimental-data}{on Authorea (includes interactive figures)} and \href{https://github.com/phockett/Extracting-Molecular-Frame-Photoionization-Dynamics-from-Experimental-Data}{Github}. Post-publication reader comments \& suggestions can be added directly via the web.
\item \href{https://www.zotero.org/groups/4733878/molecular_frame_pads_measurements_and_reconstruction}{Full bibliography, via a Zotero group \textit{``Molecular Frame PADs Measurements and Reconstruction"}} \cite{hockettZoteroGroupsMolecular}. Additional references and suggestions can be added to the group, and it is hoped that this will grow to provide a useful community resource.
\item Code for AF computation and matrix element retrieval, as illustrated in Sect. \ref{sec:Recon}. 
\begin{itemize}
\item Full Jupyter notebooks and source data are \href{http://dx.doi.org/10.6084/m9.figshare.20293782}{available on Figshare, DOI: 10.6084/m9.figshare.20293782}. This also includes a complete archive of the python software releases and environment used during this work \cite{hockett2022MFreconFigshare}.
\item The open-source python libraries \href{https://epsproc.readthedocs.io}{ePSproc} \cite{ePSprocAuthorea,ePSprocGithub,ePSprocDocs} and \href{https://pemtk.readthedocs.io}{PEMtk} \cite{hockett2021PEMtkDocs, hockett2021PEMtkGithub} developed for these problems. (See Sects. \ref{sec:numerics-intro}, \ref{sec:numerical-notes} for further discussion.)
\end{itemize}
\item Discussion forum at \href{https://amoopenscience.femtolab.ca/}{AMO Open Science}. We hope interested readers will use this forum for general discussion on the topic, if only to suggest other venues for discussion.
\item A Docker-based distribution of various codes for tackling photoionization problems is also available from the \href{https://github.com/phockett/open-photoionization-docker-stacks}{Open Photoionization Docker Stacks} project, which aims to make these tools more accessible to interested researchers \cite{hockettOpenPhotoionizationDocker}.
\end{itemize}



% TODO: add a plain-text landing page for all of the above (OSF/Figshare?) to ensure it's visible in PDF/print versions.