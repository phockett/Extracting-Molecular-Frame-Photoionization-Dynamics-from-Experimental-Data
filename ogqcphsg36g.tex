\subsection{Experimental techniques}

Following the above discussions, experimental methodologies for the determination of MF observables can be viewed from the perspective of direct and indirect techniques. In the direct case, access to the MF is sought via essentially one of two schemes:

\begin{enumerate}
\item Alignment/orientation of the molecular frame.
\item Post-hoc reconstruction of the molecular frame from a suitable measurement.
\end{enumerate}

The first category covers a range of techniques. For gas phase experiments, the most common methods involve creating some form of alignment or orientation \footnote{In the technical sense, alignment retains mirror symmetry along an aligned axis, while orientation implies full 3D control over a set of axes. The term "orientation" is used herein as synonomous with the MF for an arbitrary molecular system, but in some cases - e.g. homonuclear diatomics - alignment may be sufficient for observation of MF observables.} in the gas phase molecular ensemble, which defines a relationship between the LF and MF. In general, measurements made from such an ensemble can be termed as corresponding to "the aligned frame (AF)", and may still involve averaging over some DOFs; in the limit of perfect orientation, the AF and MF are conformal/indistinguishable. Perhaps the simplest AF technique is the creation of alignment via a single-photon pump process (as used in many resonance-enhanced mulit-photon ionization (REMPI) type experimental schemes); in this case a parallel or perpendicular transition moment will create a $cos^2$ or $sin^2$ distribution, respectively, of the corresponding molecular axis. (Any such axis distribution, in which there is a defined arrangement of axes created in the LF, can be discussed, and characterised, in terms of the axis distribution moments (ADMs), typically expressed as an expansion in spherical harmonic functions.)

Further control can be gained via a single, or sequence of, N-photon transitions, or strong-field mediated techniques. Of the latter, adiabatic and non-adiabatic alignment methods are particularly powerful, and make use of a strong, slowly-varying or impulsive laser field respectively. (Here the "slow" and "impulsive" time-scales are defined in relation to molecular rotations, roughly on the ps time-scale, with ns and fs laser fields corresponding to the typical slow and fast control fields.) In the former case, the molecular axis, or axes, will gradually align along the electric-field vector(s) while the field is present; in the latter a broad rotational wavepacket can be created, initiating complex rotational dynamics including field-free revivals of ensemble alignment. Both techniques are powerful, but multiple laser fields are typically required in order to control more than one molecular axis, leading to relatively complex experimental requirements. The absolute degree of alignment obtained in a given case is also dependent on a number of intrinsic and experimental properties, including the molecular polarisability tensor, rotational temperature and separability of the rotational degrees of freedom from other DOFs (loosely speaking, this can be considered in terms of the stiffness of the molecule). Therefore, although general in principle, in practice not all molecular targets are amenable to "good" (i.e. a high degree of) alignment. 

Whilst gas phase alignment experiments can become rather complex, multi-pulse affairs, they are increasingly popular in the AMO community for a number of possible reasons - conceptually and experimentally, they are a relatively tractable extension to existing techniques, they are interesting experiments in their own right, and, practically, they are usually feasible with existing high-power pulsed laser sources in the ns to fs regime. 

An alternative, very different, technique of orientational control is via embedding the target species in a matrix, or via deposition on a surface, which defines a spatial orientation. This approach has been taken by the surface science community, in particular in ARPES and SERS studies. In these techniques, molecular orientation is well-defined, but at the expense of interactions with the bulk. Such techniques are readily applied to a range of targets using existing experimental apparatus and techniques (hence their ready adoption, analogous to the ready adoption of multi-pulse laser schemes in the gas phase community), combined with suitably-prepared surfaces. Of particular note in this regard is SERS work making use of functionalised nano-particles for "single molecule" fluorescence studies. [more to say on ARPES?]

[Anything on cold molecules?]

The second category covers methodologies which make use of experimental information to reconstruct, post-facto, molecular alignment at the time of a light-matter interaction. This usually involves making a "kinematically complete" class of measurement, which provides the full energy or momentum partitioning of the products of a light-matter interaction. In order for the alignment of a given axis to be defined in this case, there must be a clear energy partitioning defining it - typically dissociation is required, although some axes in a given problem may be inferred from related or proxy measurements. A trivial example is the dissociation of a diatomic molecule, in this case measuring the momentum of just one product atom/ion will enable the original orientation of the molecule to be determined. Combined with measurement of an electron, in coincidence, the MF photoelectron distribution can be recovered from a set of such measurements. Further extensions to such a measurement can probe additional dynamics in the MF, for instance electron-electron correlation effects in double ionization or bond-length dependence of the observable when recoil energy of the ion is measured [ref. multiple early 2000 Dorner papers]. For larger molecules this becomes more complicated, and additionally requires that axial-recoil conditions are fulfilled (i.e. energy is not partitioned into other DOFs during dissociation). Such measurements are, therefore, well-suited to diatomics, and small polyatomics, and light-matter interactions involving core-ionization(s) events. For valence studies these techniques are less directly applicable, since dissociative events are less common (and may be slow/complex), although potentially can be applied in Coulomb-explosion imaging type scenarios[refs...?]. Another caveat for this class of measurement is the requirement for coincidence (or covariance) data collection, which usually limits count rates significantly, as well as the total number of products which may be feasibly measured [4...5...?]. For these various reasons, amongst others, these experiments have typically been performed at synchrotons with high repetition rates (high KHz, MHz) and at hard photon energies - although laser-based experiments are also relatively common. [somethign about detector tech...?][COLTRIMS, x-ray diffraction]

A final note on experimental methods is the possibilities afforded by technological (rather than conceptual) developments. Naturally, the techniques above rely on a certain degree of experimental and technological sophistication, and this, of course, generally increases over time as a technique matures and "enabling" technologies develop. Examples, in this context, are the development of, and gradual improvements in, particle imaging detectors. Such progress allows for more sophisticated experiments with, e.g. multi-particle coincidence detection, higher detection rates, energy-multiplexed measurements and so on. In short, the experimental dimensionality or information content can be increased. For developing general methods, which can be applied in complex cases, this becomes increasingly important (as do related technological capabilities, such as data storage and processing, to handle the enhanced complexity). Historically, photoionization measurement capabilities have gone from 1D (flux) and 2D (flux and kinetic energy) in the early days, advanced to sequential or parallel measurements at different angles for various flavours of angle-resolved studies, to full 3D or 4D capabilities in modern "imaging" type systems (flux and 2D or 3D kinetic energy vector resolution), with the possibility of additional experimental measurement dimensions such as time, polarisation, laser power, wavelength...