\subsection{Experimental techniques\label{sec:experimentalTechniques}}

Following the above discussions, experimental methodologies for the determination of MF observables can be viewed from the perspective of direct and indirect techniques. 

In the direct case, access to the MF is sought via essentially one of two schemes:

\begin{enumerate}
\item Alignment/orientation of the molecular frame (Sect. \ref{sec:MF-control}).
\item Post-hoc reconstruction of the molecular frame from a suitable measurement (Sect. \ref{sec:fixed-in-space}).
\end{enumerate}

Indirect techniques are the main focus of this manuscript, and experimental implications are briefly introduced in Sect. \ref{sec:MF-recon-expt}.

% [UPDATE 24/08/22 - added Sect. \ref{sec:MF-recon-expt} on recon from experiments, hopefully doesn't repeat anything elsewhere.

% UPDATE 29/07/22 - this section now sub-divided and tidied up a bit, still needs some polishing and ref integration. For MF case now in appendix \ref{appendix:MF-expt}, but not so much for alignment, although partially covered in Sect. \ref{sec:CompleteLit}.]


\subsubsection{Molecular alignment and control: conforming the MF to the LF\label{sec:MF-control}}

The first category covers a range of techniques. For gas phase experiments, the most common methods involve creating some form of alignment or orientation \footnote{In the technical sense, alignment retains 
% \textbf{inversion symmetry in the LF, while orientation typically implies reduction of the LF symmetry to match the molecular point group symmetry}.
inversion symmetry in the LF, while orientation typically implies reduction of the LF symmetry to match the molecular point group symmetry. The term "orientation" is used herein as synonomous with the MF for an arbitrary molecular system, but in some cases - e.g. homonuclear diatomics - alignment may be sufficient for observation of MF observables.} 
in the gas phase molecular ensemble, which defines a relationship between the LF and MF. In general, measurements made from such an ensemble can be termed as corresponding to ``the aligned frame (AF)", and may still involve averaging over some DOFs; in the 
% \textbf{classical} 
classical limit of perfect orientation, the AF and MF are conformal/indistinguishable. 

Perhaps the simplest AF technique is the creation of alignment via a single-photon pump process (as used in many resonance-enhanced mulit-photon ionization (REMPI) type experimental schemes, which may even be rotational-state selected); in this case a parallel or perpendicular transition moment will create a $\cos^2(\theta)$ or $\sin^2(\theta)$ distribution, respectively, of the corresponding molecular axis. Any such axis distribution, in which there is a defined arrangement of axes created in the LF, can be discussed, and characterised, in terms of the axis distribution moments (ADMs), typically expressed as an expansion in Wigner D-Matrix Elements (see Sect. \ref{sec:full-tensor-expansion}). These are spherical multipole tensors, and can be equivalently described by a density matrix \cite{BlumDensityMat}. Many authors have address aspects of this problem in the past in frequency-domain work, see, for instance, the textbooks of Zare \cite{zareAngMom} and Blum \cite{BlumDensityMat}, treatments for various experimental cases in Refs. \cite{Docker1988, Dubs1989, Greene1983}, and application in complete photoionization experiments in Refs. \cite{Leahy1991,hockett2009RotationallyResolvedPhotoelectron}.
% \textbf{Wigner D-Matrix Elements}.)

Further control can be gained via a single, or sequence of, N-photon transitions, or strong-field mediated techniques. Of the latter, adiabatic and non-adiabatic alignment methods are particularly powerful, and make use of a strong, slowly-varying or impulsive laser field respectively. (Here the ``slow" and ``impulsive" time-scales are defined in relation to molecular rotations, roughly on the ps time-scale, with ns and fs laser fields corresponding to the typical slow and fast control fields.) In the former case, the molecular axis, or axes, will gradually align along the electric-field vector(s) while the field is present; in the latter a broad rotational wavepacket (RWP) can be created, initiating complex rotational dynamics including field-free revivals of ensemble alignment. Both techniques are powerful, but multiple laser fields are typically required in order to control more than one molecular axis, leading to relatively complex experimental requirements. The absolute degree of alignment obtained in a given case is also dependent on a number of intrinsic and experimental properties, including the molecular polarisability 
% \textbf{and moment of inertia} 
and moment of inertia tensors, rotational temperature and separability of the rotational degrees of freedom from other DOFs (loosely speaking, this can be considered in terms of the stiffness of the molecule). Therefore, although general in principle, in practice not all molecular targets are amenable to ``good" (i.e. a high degree of) alignment. For more general details, see, for example, Refs. \cite{ koch2019QuantumControlMolecular, Stapelfeldt2003}, and for applications in photoionization see Sect. \ref{sec:theory-lit}.

Whilst gas phase alignment experiments can become rather complex, multi-pulse affairs, they are increasingly popular in the AMO community for a number of possible reasons - conceptually and experimentally, they are a relatively tractable extension to existing techniques, they are interesting experiments in their own right, and, practically, they are usually feasible with existing high-power pulsed laser sources in the ns to fs regime. Alignment techiques have been combined with a range of different probes, including non-linear and high-harmonic optical probes, as well as photoionization-based methods - for recent reviews see \cite{hasegawa2015NonadiabaticMolecularAlignment,koch2019QuantumControlMolecular}. 

An alternative, very different, technique of orientational control is via embedding the target species in a matrix, or via deposition on a surface, which defines a spatial orientation. This approach has been taken primarily by the surface science community, and the required methods may often be readily applied to a range of targets using existing experimental apparatus and techniques (hence their ready adoption, analogous to the ready adoption of multi-pulse laser schemes in the gas phase community), combined with suitably-prepared surfaces. Although not the topic of this manuscript, there is certainly work in this vein conceptually related to the discussions herein, including ARPES (angle-resolved photoemission spectroscopy) and SERS (surface-enhanced, coherent anti-Stokes Raman scattering) studies. In these techniques, molecular orientation is well-defined, but at the expense of interactions with the bulk - although the latter may also be of interest and/or probed by the measurement. 
Examples include ARPES studies in which orbital densities of adsorbed species are reconstructed (``orbital tomography") \cite{Puschnig2009a,dauth2014AngleResolvedPhotoemission}, and SERS work making use of functionalised nano-particles for ``single molecule" fluorescence studies of vibrational wavepackets \cite{Yampolsky2014}. 
% [more to say on ARPES?] \textbf{ Don't know enough about this stuff, so will refrain from commenting.}

% [Anything on cold molecules?] \textbf{ I think the single sentence in 3.2 referring to the review article is sufficient. We can move it here if that works better.} Agreed!

% \textbf{I think an overview of retrieval using RWPs is appropriate here...Happy to write it if we agree.} - yes, maybe as a full subsubsection here?

% \subsubsection{Freely rotating molecules: MF via time evolution\label{sec:RWPs}}

% The efforts to align and orient molecules discussed in the previous sections necessarily led to detailed studies of the rotational dynamics of molecules after interaction with a non-resonant femtosecond laser pulse. A significant outcome of these studies has been the development of a reliable model capable of accurate simulations of rotational wavepacket dynamics that quantitatively agree with experimental results. By measurement of a signal from a time evolving rotational wavepacket, this ability to accurately simulate the wavepacket dynamics can be used to reconstruct the measured signal in the molecular frame. Since in this case the time resolved measurement constitutes a set of measurements of the same quantity from a variety of molecular axes distributions, it is reasonable to conclude that if the axes distributions are known, and provided a large enough space of orientations is explored by the molecule over the experimental time window, the molecular frame signal should be extractable. 

% This is relatively straight forward for a signal that is a single number in the MF for a given polarization of the light, such as the photoionisation yield. Such a signal may, in general, be expressed as an expansion,
% \begin{equation}
% S(\theta,\chi)=\sum_{jk}C_{jk}D^{j}_{0k}(\theta,\chi),
% \label{eq:mfrealsig}
% \end{equation}
% where $\theta$ and $\chi$ are the spherical polar and azimuthal angles of the linearly polarized electroic field vector generating the signal; $C_{jk}$ are unknown expansion coefficients; and $D^{j}_{0k}$ are the Wigner D-Matrix elements, a basis on the space of orientations. A time resolved measurement of $S$ from a rotational wavepacket is the quantum expectation value of this expression,
% \begin{equation}
% \langle S \rangle(t) = \sum_{jk}C_{jk}\langle D^{j}_{0k} \rangle (t).
% \end{equation}
% Since the rotational wavepacket can be accurately simulated, the $\langle D^{j}_{0k} \rangle (t)$ are considered known. The time resolved signal $\langle S \rangle(t)$ being measured, the unknown coefficients $C_{jk}$ can be determined by linear regression (albeit overdetermined), and the molecular frame signal in Eq.~\ref{re:mfrealsig} constructed. In this form the method was initially applied to strong field ionization and dubbed Orientation Reconstruction through Rotational Coherence Spectroscopy (ORRCS). It has since been applied to strong field ionization of various molecules, strong field dissociation and few-photon ionization.

% The case of PADs is a more challenging one, since they are not generally described by Eq.~\ref{eq:mfrealsig}. Instead, both LFPADs and MFPADs are determined by the radial dipole matrix elements as described above (\textbf{refer to section 3 here!}). The authors of this manuscript, with a number of collaborators, demonstrated that these matrix elements can also be retrieved for one-photon ionisation of N$_2$ by time resolved measurements of LFPADs from a rotational wavepacket. This method is focus of sections X and Y (\textbf{refer to apprpriate section(s)!}) below, with additional details and results provided on the case of radial matrix element extraction for N$_2$.   In follow up work, it was shown that for molecules with $D_{nh}$ point group symmetry the retrieval of the MFPAD is possible directly, bypassing the radial matrix elements. 



\subsubsection{Fixed-in-space molecules: MF via axis reconstruction\label{sec:fixed-in-space}}

The second category covers methodologies which make use of experimental information to reconstruct, post-facto, molecular alignment at the time of a light-matter interaction. This usually involves making a ``kinematically complete" class of measurement, which provides the full energy or momentum partitioning of the products of a light-matter interaction. In order for the alignment of a given axis to be defined in this case, there must be a clear energy partitioning defining it - typically dissociation is required, although some axes in a given problem may be inferred from related or proxy measurements. The simplest example is the dissociation of a diatomic molecule, in this case measuring the momentum of just one product atom/ion will enable the original orientation of the molecule to be determined 
% \textbf{provided the molecule does not rotate while dissociating}. 
provided the molecule does not rotate while dissociating (i.e. photodisociation is also sudden/impulsive, or equivalently geometrically-uncoupled - otherwise this remains a DOF to be averaged over, resulting in ``recoil-frame" (RF) measurements!). Combined with measurement of an electron, in coincidence, the MF photoelectron distribution can be recovered from a set of such measurements. Recent discussions and reviews of this area can be found in refs. \cite{Yagishita2005,Reid2012,dowek2012PhotoionizationDynamicsPhotoemission,Yagishita2015,jahnke2022PhotoelectronDiffraction}, and some (representative) examples from the literature are given in Sect. \ref{appendix:MF-expt}.


Further extensions to such a measurement can probe additional dynamics in the MF, for instance electron-electron correlation effects in double ionization \cite{Akoury2007}. %or bond-length dependence of the observable when recoil energy of the ion is measured [ref. multiple early 2000 Dorner papers. \textbf{Also Dowek and Lucchese should be cited somewhere here.} ]. 
For larger molecules this becomes more complicated, and additionally requires that axial-recoil conditions are fulfilled (i.e. energy is not partitioned into other DOFs during dissociation). Such measurements are, therefore, well-suited to diatomics, and small polyatomics, and light-matter interactions involving core-ionization(s) events. For valence studies these techniques are less directly applicable, since dissociative events are less common (and may be slow/complex), although potentially can be applied in Coulomb-explosion imaging (CEI) type scenarios. In CEI methods, intense fields are used to strip multiple electrons from the target system, causing the molecule to (Coulombically) explode. Provided the intense pulse is short (relative to time-scales of atomic motion), measurement of multiple ionic fragments yields a map of the geometry of the system at the instant of the interaction \cite{stapelfeldt1998TimeresolvedCoulombExplosion,Underwood2015,Slater2015}. 

Another caveat for this class of measurement is the requirement for coincidence (or covariance) data collection, which typically limits count rates significantly, as well as the total number of products which may be feasibly measured - although the requirements may be somewhat relaxed for covariance studies. For these various reasons, amongst others, these experiments have typically been performed at synchrotons (and, recently, at FELs) with high repetition rates (high KHz, MHz) and at hard photon energies - although laser-based experiments are also relatively common, particularly in the strong-field community, and may become more so as sources with high-repetition rates and high peak intensities become commercially available. 

Recent state-of-the-art MF measurements have successfully measured 4-fold coincidences, and 5-particle covariance maps, in order to map polyatomic molecules and vibrational dynamics; %[Dorner and Boll refs]
for a recent example, illustrating the power of such a technique for CEI imaging of iodopyridine and iodopyrazine, see Ref.   \cite{boll2022XrayMultiphotoninducedCoulomb}. However, to date these techniques have mainly been used as structural (nuclear) probes; in terms of MFPAD measurements, these techniques remain very challenging, since the CEI schemes required produce many secondary electrons which cannot typically be distinguished. 


% [somethign about detector tech...?][COLTRIMS \cite{Moshammer1996,Dorner1997a,Dorner2000}, x-ray diffraction] \cite{Akoury2007}

\subsubsection{Post-processing \& ``complete" photoionization studies: MF via reconstruction\label{sec:MF-recon-expt}}

% \textbf{[Added this section 23/08/22 to tie together experimental techniques with analysis - needs some work and refs still, bit generic at the moment]}

A significant issue with ``direct" experimental approaches to the MF is the difficulty of the measurement, and the degree of MF fidelity obtained, particularly in more complex systems. (Related to this is the issue of whether the free system is measured, or whether it is purturbed in some way, e.g. by an alignment laser field or a coupled system.) A complementary approach is to employ a post-processing approach in which underlying MF properties, possibly even the full set of photoionization matrix elements, are sought from LF or AF measurements, as already introduced in Sect. \ref{sec:MF-recon-basic-intro}. Such schemes are potentially demanding and complex in terms of the computational effort required to post-process the experimental data, but may also be significantly less demanding experimentally than direct MF measurements; such schemes additionally have the potential to provide more fundamental information on photoionization dynamics.

Experimental techniques to generate suitable datasets for analysis are many and varied, and include the direct analysis of MF measurements to obtain the underlying photoionization dynamics, the use of frequency- resolved methods, and the use of molecular alignment techniques. Some representative examples of such ``complete" photoionization studies from the literature are given in Sect. \ref{sec:CompleteLit}. %, and it is also of note that other fields have approached similar . 
The main focus of Sect. \ref{sec:Recon} is the analysis of time-resolved data from an aligned system with a prepared RWP, although retrieval from the MF is also investigated herein (Sect. \ref{sec:recon-from-MFPADs}). Whilst the RWP case is experimentally similar to the ``direct" MF measurement case outlined above, the requirements on the degree of alignment are much reduced, since the fidelity arises from the analysis, rather than the absolute maximum degree of alignment obtained. Specifically, the fidelity of the reconstruction can be considered as a result of the total information content of the time-domain measurement (related to the number of distinct molecular axis distributions probed), as distinct from the ``direct" case which is constrained by a single measurement at the best alignment or orientation obtained - in practice this needs to be extremely good to truly approach MF information in general, and this may not be possible in many cases; it will also be constrained in general by the symmetry of the problem. For an example case study, see Ref. \cite{reid2018AccessingMolecularFramea}, which suggests $\langle\cos^2(\theta)\rangle>0.9$ as a \textit{minimum} requirement, for measurements of relatively simple, cylindrically-symmetric, MFPADs; another illustrative example of the loss of information in the MF to LF transformation can be found in Ref. \cite{Underwood2000}.
%[Refs \cite{Underwood2000} plus Seidemann works and Reid gaussian wavepacket - could cite some specific numbers from these. Big RWP paper too?]
% (see Sect. \ref{sec:bootstrapping-info-sensitivity})

This implies that reconstruction methods may be rather more general, and the RWP case in particular is expected to be applicable to any molecular system, although outstanding questions on the required and obtainable information content remain (see Sect. \ref{sec:info-content}). Questions of fidelity of reconstruction are also a matter of ongoing research, since this will, again, depend on both the type and nature of the experimental measurements, and the reconstruction methodology, both of which may involve or assume certain additional DOFs or physical behaviours which are required for tractable theory but may not hold in practice. Examples include cases with multiple conformers, floppy systems and the presence of additional (e.g. vibrational) dynamics. In all cases progress may be possible, but will require additional experimental and/or computational effort to control, isolate, simulate or reconstruct the additional DOFs. (See Ref. \cite{Takatsuka2000} for a ``basic" theoretical vibrational wavepacket example in $Na_2$, and for a more complex case in $CS_2$ Ref. \cite{wang2017MonitoringNonadiabaticDynamics}.)

A benchmark example is the retrieval of the photoionization dynamics of $N_2$, since this is a simple, fairly rigid, system amenable to experimental control ($NO$ has also been investigated by a number of authors, see Sect. \ref{sec:CompleteLit}). The authors of this manuscript, with a number of collaborators, demonstrated that photoionization matrix elements can be retrieved for one-photon ionisation of $N_2$ by time resolved measurements of LFPADs from a rotational wavepacket. The experiments did achieve a relatively high degree of alignment via a two-pulse pump scheme, with a maximum $\langle\cos^2(\theta)\rangle$ of $\sim 0.8$, and 11 temporal data-points (obtained over the half and full RWP revivals) were found to be sufficient for full matrix element retrieval and MFPAD reconstruction. This methodology, ``bootstrapping to the molecular frame", is the focus of Sect. \ref{sec:bootstrapping} below, with additional details and results provided for the case of radial matrix element extraction for N$_2$.  In follow up work, it was shown that for molecules with $D_{nh}$ point group symmetry the retrieval of the MFPAD is possible directly via a matrix inversion methodology, bypassing the difficulty of extracting the radial matrix elements, and this is discussed in Sect. \ref{sec:Matrix-inversion-example}.

% [TODO - also longer RWP section to be written? UPDATE: now Sect. \ref{sec:RWPs}, also moved above para from there.
% Also Appendix section?]



\subsubsection{Technology and outlook}

A final note on experimental methods is the possibilities afforded by technological (rather than conceptual) developments. Naturally, the techniques above rely on a certain degree of experimental and technological sophistication, and this, of course, generally increases over time as a technique matures and ``enabling" technologies develop. Examples, in this context, are the development of, and gradual improvements in, particle imaging detectors. Such progress allows for more sophisticated experiments with, e.g. multi-particle coincidence detection, higher detection rates, energy-multiplexed measurements and so on. In short, the experimental dimensionality or information content can be increased. For developing general methods, which can be applied in complex cases, this becomes increasingly important (as do related technological capabilities, such as data storage and processing, to handle the enhanced complexity). 

Historically, photoionization measurement capabilities have gone from 1D (flux) and 2D (flux and kinetic energy) in the early days, advanced to sequential or parallel measurements at different angles for various flavours of angle-resolved studies, and to full 3D or 4D capabilities in modern ``imaging" type systems (flux and 2D or 3D kinetic energy vector resolution), with the possibility of additional experimental measurement dimensions such as time, polarisation, laser power, wavelength and so forth. Advances in experimental methods, in particular the development of high count-rate 3D detectors, is ongoing (see, for examples and historical context, \cite{Parker1997,Dorner1997a,Continetti2001,Vallance2013,chandler2017PerspectiveAdvancedParticle}). 

As well as detector technology, general developments in experimental methods, system integration, computational power and so forth further enable novel techniques and/or fusion of existing methodologies. For instance, it is of note that with the advent of more sophisticated laser-based techniques, and short-pulse FELs, the combination of laser alignment techniques with dissociative photoionization measurements is also now increasingly common; other examples include increasingly sophisticated multi-pulse measurements and measurements with shaped laser pulses. An illustrative example of the rich data available from a modern, sophisticated, experimental scheme - CEI with a pixel-imaging mass-spectroscopy (PImMS) camera and covariance analysis - see \cite{Slater2015}. Further examples of experimental developments from the literature, with respect to matrix element retrieval, can be found in Sect. \ref{sec:CompleteLit}.


% Integrate this part above?
% Beyond the basic case, increasingly sophisticated experimental schemes are now routine for a number of labs \cite{Reid2012}, examples include: state-selected measurements; pump-probe schemes for time-resolved PADs (\cite{Seideman2002} [plus Wu, others?]); control schemes with shaped laser pulse(s), or multipulse schemes, controlling bound or continuum dynamics [Eliott refs, Baumert and related refs]; schemes coupling continuum states for energy-dependent phase measurements [rabbit refs.]; recoil frame (RF) and aligned frame (AF) measurements. These last cases are the focus herein, since they provide a relatively general route to MF observables and related properties.



% LIST IN PROGRESS 03/09/21 
% NOTE: doesn't currently include any dynamical cases.
% TO CONSIDER: move to end as supplementals, and link to Zotero library for "ongoing" version?
% 09/09/21 now moved, use this style generally to allow organic growth... easy way to link it?

% ADDITIONAL MFPAD notes...
% Complete expts from MFPADS: \cite{Shigemasa1995,Cherepkov2000,Hikosaka2000,Yagishita2005}
% NOTES to add on RF vs. MF?