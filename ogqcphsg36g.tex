\section{Concepts & techniques}

Photoionization dynamics[ePSproc example walk-through? Or notebooks from QM book?]

\subsection{Experimental techniques}

Following the above discussions, experimental methodologies for the determination of MF observables can be viewed from the perspective of direct and indirect techniques. In the direct case, access to the MF is sought via essentially one of two schemes:Orientation of the molecular frame.Post-hoc reconstruction of the molecular frame from a suitable measurement.The first category covers a range of techniques. For gas phase experiments, the most common methods involve creating some form of alignment or orientation (in the technical sense, alignment retains mirror symmetry along an aligned axis, while orientation implies full 3D control over a set of axes - orientation is used herein as synonmous with the MF for an arbitrary molecular system, but in some cases - e.g. homonuclear diatomics - alignment may be sufficient for MF observables) in the gas phase molecular ensemble, which defines a relationship between the LF and MF. In general, this can be termed as the aligned frame (AF), and may still involve averaging over some DOFs; in the limit of perfect orientation, the AF and MF are conformal/indistinguishable. Perhaps the simplest AF technique is the creation of alignment via a single-photon pump process (as used in many resonance-enhanced mulit-photon ionization (REMPI) type experimental schemes); in this case a parallel or perpendicular transition moment will create a $cos^2$ or $sin^2$ distribution, respectively, along the corresponding symmetry axis. 

