\section{Concepts \& techniques\label{sec:Concepts}}

\subsection{Photoionization dynamics} 
[ePSproc example walk-through? Or notebooks from QM book?]

The core physics of photoionization has been covered extensively in the literature, and only a very brief overview is provided here with sufficient detail to define the MF; the reader is referred to the literature listed at the end of this section for further details.

Photoionization describes the coupling of an initial state of the system to a final state (photoion(s) plus free photoelectron(s)), coupled by an electric field/photon. Very generically, this can be written as a matrix element $\langle\Psi_i|\hat{\Gamma}(E)|\Psi_f\rangle$, where $\hat{\Gamma}(E)$ defines the light-matter coupling operator (depending on the electric field $E$), and $\Psi_i$, $\Psi_f$ the total wavefunctions of the initial and final states respectively. 

There are many flavours of this fundamental light-matter interaction, depending on system and coupling; the discussion here is confined initially to the simplest case of single-photon absorption, in the weak field (or purturbative), dipole regime, resulting in a single photoelectron.
% RENDERING ISSUES HERE? REFS in footnote seem to be this issue
\footnote{For more discussion of these various approximations in photoionzation, see refs. [see raw tex]}  %\cite{Seideman2002,Seideman2001} . } 
As a first approximation, this can usually be treated as a single active electron (SAE) problem, in which the remainder of the system is static during the photoionization process (impulsive, or sudden, approximation): this allows the problem to be defined in terms terms of three key components: (1) the initial (ionizing) state (electronic) wavefunction, (2) the structure of the continuum (free electron) wavefunction, (3) the dipole matrix elements coupling these (single electron) wavefunctions. The corresponding photoionization matrix elements can be written as $\langle\psi_i|\hat{\mu}.E|\psi_f\rangle$, where the wavefunction pertain to the active (ionized) electron, and $\hat{\mu}.E$ is the usual dipole term.

However, this problem still remains rather complicated (hence interesting), since the structure of the initial and continuum states depends sensitively on the geometry (atomic positions and electron distribution, i.e. the full vibronic wavefunction) of the ionizing system. Nonetheless, significant progress can be made in both experimental analysis and \textit{ab initio} theory in this reduced case, and the approximations are valid for many interesting real cases (e.g. small, relatively rigid, polyatomics). Naturally, this zero-order treatment also provides a framework within which other effects can be recognised and understood, in terms of which physical assumptions are broken.

Although the core physics is complicated, relatively high-dimensionality observables are possible for photoelectron measurements, hence experimental progress can be made to understand these light-matter interactions. The (energy resolved) \textit{photoelectron angular distribution} (PAD) is the key observable, which may be measured in the LF or MF (see Sect. XX), and additionally interrogated as a function of other experimental parameters: of particular interest herein will be the electric field polarization, and orientation in the MF. 

Beyond the basic case, increasingly sophisticated experimental schemes are now routine for a number of labs \cite{Reid2012}, examples include: state-selected measurements; pump-probe schemes for time-resolved PADs (\cite{Seideman2002} [plus Wu, others?]); control schemes with shaped laser pulse(s), or multipulse schemes, controlling bound or continuum dynamics [Eliott refs, Baumert and related refs]; schemes coupling continuum states for energy-dependent phase measurements [rabbit refs.]; recoil frame (RF) and aligned frame (AF) measurements. These last cases are the focus herein, since they provide a relatively general route to MF observables and related properties.

[List of refs by topic here? Seems better than prose style, and can be correlated with Zotero lib.]

