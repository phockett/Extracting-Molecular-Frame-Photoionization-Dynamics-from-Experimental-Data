\section{Concepts \& techniques}

\subsection{Photoionization dynamics} 
[ePSproc example walk-through? Or notebooks from QM book?]

The core physics of photoionization has been covered extensively in the literature, and only a very brief overview is provided here; the reader is referred to the literature listed at the end of this section for further details.

Photoionization describes the coupling of an initial state of the system to a final state (photoion(s) plus free photoelectron(s)), coupled by an electric field/photon, vis. $\langle\Psi_i|\hat{\Gamma}E|$. There are many flavours of this fundamental light-matter interaction, depending on system and coupling; the discussion here is confined initially to the simplest case of single-photon absorption, in the weak field, dipole regime, resulting in a single photoelectron. In essence, photoionization in this limit can be considered in terms of two key components: (1) the structure of the continuum (free electron) wavefunction 