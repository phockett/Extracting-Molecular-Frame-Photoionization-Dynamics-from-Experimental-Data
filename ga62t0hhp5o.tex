

\subsubsection{Bootstrapping basics: dataset and setup}

For the example case, synthetic data was used, although the dataset was inspired by ref. \cite{marceau2017MolecularFrameReconstruction}. In that case, experimental results were obtained via a (dual) pump-probe scheme, where the IR pump pulses prepared a rotational wavepacket in $N_2$, and a time-delayed XUV probe pulse ionized the sample - this scheme is illustrated in Fig. \ref{781808}. Multiple channels were probed in this manner, for a range of time-delays, to provide a dataset of AF-$\beta_{LM}(E,t)$ parameters. Here the focus is on the photoionization step, information content and further investigation of the retrieval routines; this is facilitated by the following choices and data:

\begin{enumerate}
\item The same rotational wavepacket \& molecular axis distribution as obtained experimentally is assumed. Consequently, the 1st (linear) stage of the bootstrapping protocol is not investigated herein. Although this may seem like a drastic omission, in general this stage of the methodology is expected to be quite robust, and this has been demonstrated in various investigations of rotational wavepackets and molecular alignment techniques, see discussion in Sect. \ref{sec:RWPs}.
% \footnote{Although this may seem like a drastic omission, in general this stage of the methodology is expected to be quite robust, and this has been demonstrated in various investigations of rotational wavepackets and molecular alignment techniques. [REFS]}
\item To simulate the observables, photoionization matrix elements from ePolyScat \cite{Lucchese1986, Gianturco1994, Natalense1999, luccheseEPolyScatUserManual} calculations were used. The calculations made use of $N_2$ electronic structure input, computed with Gamess \cite{gamess, Gordon} (RHF/MP2/6-311G, bond length $1.07$~\AA). In the example illustated herein, matrix elements for $E_{ke}=1~eV$ were used, and are given explicitly in Table \ref{tab:inputMatE}. This, naturally, also provides a means for direct comparison and fidelity analysis of the retrieval protocol (Sect. \ref{sec:bootstrap-fidelity}); the density matrix can also be used for analysis (Sect. \ref{sec:den-mat-N2}). In the current study only ionization of the $3\sigma_g$ orbital (HOMO) is investigated, corresponding to formation of  $X^2\Sigma_{g}^{+}$ ionic ground-state, i.e. $N_2(X^{1}\Sigma^{+}_{g}) \rightarrow N^+_2(X^{2}\Sigma^{+}_{g})$, generically denoted as the $X$-channel, or via the ionizing orbital as $N_2(3\sigma_g^{-1})$, for a more compact notation. 
\item To investigate limitations of the numerical routines, noise and other artiefacts can be added to the simulated data; different sub-sets of the data can also be readily analysed and compared.
\item Finally, it is of note that the numerical implementations are structured as a set of tensors, as close as possible to the formalism given above (Sect. \ref{sec:tensor-formulation}). This provides a means to further investigate the information content of various parts of the problem, and investigate their influence on the retrieval - in particular, these can indicate aspects of the data which may be most sensitive to particular matrix elements, most susceptible to noise and so forth. This is explored in Sect. \ref{sec:bootstrapping-info-sensitivity}
% The numerics are currently implemented using the Xarray python library [REFS].
\end{enumerate}

The sample dataset used for the results presented herein is illustrated in Figure \ref{720080} and, as mentioned previously, full numerical data can be accessed online (Sect. \ref{sec:resources}). In this case only a 1~ps subset of the simulation data was used, over the main revival feature. Figure \ref{720080} shows both the full simulation results and the sub-selected points (13 data points) with random noise added (up to 10\%), the latter is used as the input dataset for matrix element retrieval in the following sections. % [May want to show more simulated data here? DONE]


