14/09/21 - some additional formalism dev in progress (from local Lyx doc notes).

See \href{https://epsproc.readthedocs.io/en/dev/methods/ePSproc_geom_methods_summary_190821-v1-tidy.html}{https://epsproc.readthedocs.io/en/dev/methods/ePSproc\_geom\_methods\_summary\_190821-v1-tidy.html} 

\section{Matrix elements}

Matrix element square (ePS notation):

\begin{equation}
I_{l,m,\mu}^{p_{i}\mu_{i},p_{f}\mu_{f}}(E)I_{l',m',\mu'}^{p_{i}\mu_{i},p_{f}\mu_{f}*}(E)
\end{equation}

Rewrite as single (tensor) term:

\begin{equation}
\mathbb{I}_{lm\mu,l'm'\mu'}^{\Gamma,\Gamma'}=I_{l,m,\mu}^{\Gamma}(E)I_{l',m',\mu'}^{\Gamma'*}(E)
\end{equation}

And the symmetry/channel indices are denoted by $\Gamma,\Gamma'$
for brevity (in general, $\Gamma=\Gamma'$ since the continua are
orthogonal).

(TODO: decide on notation here - $I$ following ePS notation, $D$
following old notation (+ Jon), or $R$ for radial...?)

This is effectively a density matrix (?) (see \href{https://epsproc.readthedocs.io/en/dev/methods/ePSproc_geom_methods_summary_190821-v1-tidy.html}{density mat dev notes}).

\begin{equation}
\mathbb{I}_{lm\mu,l'm'\mu'}^{\Gamma,\Gamma'}\equiv\mathbf{\rho}_{lm\mu,l'm'\mu'}^{\Gamma,\Gamma'}
\end{equation}

Where the matrix elements are:

\begin{equation}
\mathbf{\rho}_{lm\mu,l'm'\mu'}^{\Gamma,\Gamma'}=I_{l,m,\mu}^{p_{i}\mu_{i},p_{f}\mu_{f}}(E)I_{l',m',\mu'}^{p_{i}\mu_{i},p_{f}\mu_{f}*}(E)
\end{equation}

Rewrite MF eqn with new term:

\begin{equation}
\beta_{L,-M}^{\Gamma,\Gamma'}(E)=(-1)^{M}\sum_{P,R',R}(2P+1)^{\frac{1}{2}}E_{P-R}(\hat{e};\mu_{0})\sum_{l,m,\mu}\sum_{l',m',\mu'}(-1)^{(\mu'-\mu_{0})}\Lambda_{R',R}(R_{\hat{n}};\mu,P,R,R')B_{L,-M}(l,l',m,m')\mathbb{I}_{lm\mu,l'm'\mu'}^{\Gamma,\Gamma'}(E)
\end{equation}

And AF:

\begin{equation}
\bar{\beta}_{L,-M}^{\Gamma,\Gamma'}(E,t)=(-1)^{M}\sum_{P,R',R}[P]^{\frac{1}{2}}E_{P-R}(\hat{e};\mu_{0})\sum_{l,m,\mu}\sum_{l',m',\mu'}(-1)^{(\mu'-\mu_{0})}\bar{\Lambda}_{R'}(\mu,P,R')B_{L,S-R'}(l,l',m,m')\mathbb{I}_{lm\mu,l'm'\mu'}^{\Gamma,\Gamma'}(E)\sum_{K,Q,S}\Delta_{L,M}(K,Q,S)A_{Q,S}^{K}(t)
\end{equation}

In this case, $F_{LS}(q,q')\approx\sum_{\Gamma,\Gamma'}\sum_{l,m,\mu}\sum_{l',m',\mu'}B_{L,S-R'}(l,l',m,m')\mathbb{I}_{lm\mu,l'm'\mu'}^{\Gamma,\Gamma'}(E)$,
where $F_{LS}(q,q')$ is from Jon's derivation (eqn. 46), except his
also includes $\varepsilon(\epsilon)$ field term (this is the field
averaged FT/envelope). Note also that the $A_{Q,S}^{K}$ (ADMs) are
generally defined by rotational state populations as per eqn. 47.

\section{Channel functions}

From book, vol. 2, chpt. 12

\begin{equation}
\beta_{L,M}^{u}=\sum_{\zeta,\zeta'}\varUpsilon_{L,M}^{u,\zeta\zeta'}\Lambda^{\zeta\zeta'}
\end{equation}

and 

\begin{equation}
\Lambda^{\zeta\zeta'}=\sum_{\Gamma,\Gamma'}\sum_{\mu,\mu'}\sum_{h,h'}b_{hl\lambda}^{\Gamma\mu*}b_{h'l'\lambda'}^{\Gamma'\mu'}\boldsymbol{D}_{hl}^{\Gamma\mu*}(q,\,k)\boldsymbol{D}_{h'l'}^{\Gamma'\mu'}(q',\,k)
\end{equation}

In the current notation, this can be written as:

\begin{equation}
\beta_{L,M}^{u}=\sum_{\zeta,\zeta'}\varUpsilon_{L,M}^{u,\zeta\zeta'}\mathbb{I}_{\zeta\zeta'}^{\Gamma,\Gamma'}=\sum_{\zeta,\zeta'}\varUpsilon_{L,M}^{u,\zeta\zeta'}\mathbb{I}^{\zeta\zeta'}\equiv\sum_{\zeta,\zeta'}\varUpsilon_{L,M}^{u,\zeta\zeta'}\mathbf{\rho}^{\zeta\zeta'}
\end{equation}

This is basically the convolution form with channel functions, for
a given ``experiment'' $u$, summed over all terms $\zeta,\zeta'$
(the 2nd form assumes that the symmetries are lost in here too, the
3rd just rewrites as a density matrix).

The full forms are, therefore:

\begin{equation}
\varUpsilon_{L,M}^{u,\zeta\zeta'}=(-1)^{M}(2P+1)^{\frac{1}{2}}E_{P-R}(\hat{e};\mu_{0})(-1)^{(\mu'-\mu_{0})}\Lambda_{R',R}(R_{\hat{n}};\mu,P,R,R')B_{L,-M}(l,l',m,m')
\end{equation}

\begin{equation}
\bar{\varUpsilon}_{L,M}^{u,\zeta\zeta'}=(-1)^{M}[P]^{\frac{1}{2}}E_{P-R}(\hat{e};\mu_{0})(-1)^{(\mu'-\mu_{0})}\bar{\Lambda}_{R'}(\mu,P,R')B_{L,S-R'}(l,l',m,m')\Delta_{L,M}(K,Q,S)A_{Q,S}^{K}(t)
\end{equation}

\section{Final state density matrix}

Write the full final state as a density matrix:

\begin{equation}
\mathbf{\bar{\rho}}_{L,M}^{\zeta\zeta'}=\varUpsilon_{L,M}^{u,\zeta\zeta'}\mathbf{\rho}^{\zeta\zeta'}
\end{equation}

Here the density matrix can be interpreted as the final, LF/AF or
MF density matrix, incorporating both the intrinsic and extrinsic
effects (i.e. all channel couplings and radial matrix elements for
the given measurement). The $L,M$notation indicates here that these
dimensions should not be summed over, hence we can still write the
tensor coupling into the $\beta_{L,M}^{u}$ parameters:

\begin{equation}
\beta_{L,M}^{u}=\sum_{\zeta,\zeta'}\mathbf{\bar{\rho}}_{L,M}^{\zeta\zeta'}
\end{equation}


\section{Matrix inversion formalism\labe}

Following Gregory et. al., the formalism can also be rewritten with
the ADMs kept separate (see also Reid \& Underwood convolution form),
with the LF/AF given as (eqn. 4):

\begin{equation}
\bar{\beta}_{L,M}(E,t)=\sum_{K,Q,S}C_{K,Q,S}^{L,M}(E)A_{Q,S}^{K}(t)
\end{equation}

($Q=0$ only in Gregory et. al., don't recall if this is general limitation
however - may be symmetry and/or dimensionality related?)

And the MF (eqn. 10):

\begin{equation}
\beta_{L,M}(E,\Omega)=\sum_{P,R,\Delta q}C_{P,R}^{L,M}(E,\Delta q)D_{R,\Delta q}^{P}(\Omega)
\end{equation}

And the transformation can be written as (eqn. 1, 28):

\begin{equation}
\mathbf{C}^{mol}=\mathbf{G}\mathbf{C}^{lab},\label{eq:basic}
\end{equation}

\begin{equation}
C_{P,0}^{L,M}(E,\Delta q)=\hat{G}_{L'0KS}^{LMP\Delta q}\bar{C}_{K,0,S}^{L',0}(E)
\end{equation}

(Here $M'=Q=R=0$, again not sure if this is required.)

And

\begin{equation}
\mathbf{G}_{L'M'KS}^{LMP\Delta q}=\mathbf{\Gamma}_{P0\Delta q}^{\zeta\zeta'LM}(\mathbf{\Gamma}_{K0S}^{\zeta\zeta'L^{\prime}M^{\prime}})^{+}
\end{equation}

Where $+$ denotes the Moore-Penrose inverse matrix.

\begin{equation}
\bar{C}_{KQS}^{LM}(\epsilon)=\sum_{\zeta\zeta'}d_{\zeta\zeta'}(\epsilon)\Gamma_{KQS}^{\zeta\zeta'LM}
\end{equation}

\begin{equation}
C_{PR}^{LM}(\epsilon,\Delta q)=\sum_{\zeta\zeta'}d_{\zeta\zeta'}(\epsilon)\Gamma_{PR\Delta q}^{\zeta\zeta'LM}
\end{equation}

\begin{equation}
D_{\zeta}(\epsilon)D_{\zeta'}^{*}(\epsilon)=d_{\zeta\zeta'}(\epsilon)
\end{equation}

In the current notation, we can rewrite...

\begin{equation}
\bar{C}_{KQS}^{LM}(\epsilon)=\sum_{\zeta\zeta'}\mathbb{I}_{\zeta\zeta'}^{\Gamma,\Gamma'}(\epsilon)\Gamma_{KQS}^{\zeta\zeta'LM}
\end{equation}

\begin{eqnarray}
\Gamma_{KQS}^{\zeta\zeta'LM} & = & (-1)^{M}[P]^{\frac{1}{2}}E_{P-R}(\hat{e};\mu_{0})(-1)^{(\mu'-\mu_{0})}\bar{\Lambda}_{R'}(\mu,P,R')B_{L,S-R'}(l,l',m,m')\Delta_{L,M}(K,Q,S)\\
 & = & \bar{\varUpsilon}_{L,M}^{u,\zeta\zeta'}/A_{Q,S}^{K}(t)
\end{eqnarray}

\begin{equation}
C_{PR}^{LM}(\epsilon,\Delta q)=\sum_{\zeta\zeta'}\mathbb{I}_{\zeta\zeta'}^{\Gamma,\Gamma'}(\epsilon)\Gamma_{PR\Delta q}^{\zeta\zeta'LM}
\end{equation}

\begin{equation}
\Gamma_{PR\Delta q}^{\zeta\zeta'LM}=(-1)^{M}(2P+1)^{\frac{1}{2}}E_{P-R}(\hat{e};\mu_{0})(-1)^{(\mu'-\mu_{0})}\bar{\Lambda}_{\Delta q}(\mu,P,\Delta q)B_{L,-M}(l,l',m,m')
\end{equation}

assuming $R'=\Delta q$, and that the rotational matrix element $D_{R,\Delta q}^{P}(\Omega)$
is computed independently (the current numerical function $\Lambda_{R',R}(R_{\hat{n}};\mu,P,R,R')$
could possibly used directly here, but this version keeps the angle-dependence
separate as per the matrix inversion formalism; \textcolor{red}{alternatively
a new function might be used here?}).