% From Pandas Latex export + added caption & label wrapper
% TODO: update with Styled table (PD>1.3)
\begin{table}

% Version paramsInputRelabelled_latex_210422.tex
\begin{tabular}{llllrrrl}
\toprule
   &   &    &    &          comp &     m &      p & labels \\
Cont & l & m & $\mu$ &               &       &        &        \\
\midrule
PU & 1 & -1 &  1 &  1.163-1.354j & 1.785 & -0.861 &   1,-1 \\
   &   &  1 & -1 &  1.163-1.354j & 1.785 & -0.861 &    1,1 \\
   & 3 & -1 &  1 & -0.803-0.017j & 0.803 & -3.120 &   3,-1 \\
   &   &  1 & -1 & -0.803-0.017j & 0.803 & -3.120 &    3,1 \\
SU & 1 &  0 &  0 & -2.317+1.359j & 2.686 &  2.611 &    1,0 \\
   & 3 &  0 &  0 &  1.106-0.087j & 1.109 & -0.079 &    3,0 \\
\bottomrule
\end{tabular}

\caption{\label{tab:matE}Input matrix elements for the simulated data, from ePolyScat calculations for $N_2$ at 1~eV, (comp)lex values, and in (m)agnitudes and (p)hase form. Note that the continuum symmetry is split into PU and SU components, correlated with $\mu=\pm1$ and $}
\end{table}