\section{Reconstruction examples \& recent developments in time-domain measurements\label{sec:Recon}}

In this section/the remainder of the manuscript, reconstruction of MFPADs from time-domain measurements are considered using two methodologies:

\begin{enumerate}
\item ``Bootstrapping" to the MF \cite{hockett2018QMP1,hockett2018QMP2,marceau2017MolecularFrameReconstruction}. The protocol outline is shown in Fig. \ref{807606}, and discussed in Sect. \ref{sec:bootstrapping}.
\item Matrix reconstruction with Moore-Penrose inversion \cite{gregory2021MolecularFramePhotoelectron}. The protocol outline is shown in Fig. \ref{731792}, and discussed in Sect. \ref{sec:Matrix-inversion-example}.
\end{enumerate}

The techniques are closely related, and both make use of rotational wavepackets (geometrical coherences) to mediate the LF/AF information content of a set of measurements in the time-domain, but differ in the ``directness" of the reconstruction. In the former case, the aim is full matrix element retrieval (Sect. \ref{sec:bootstrap-fidelity}) or, equivalently, continuum density matrix reconstruction (Sect. \ref{sec:den-mat-N2}), and the MFPADs can then be computed from these ``complete" results (Sect. \ref{sec:bootstrap-MFPADs}). In the latter case of matrix reconstruction the MFPADs are determined, essentially, via a transformation matrix (Sect. \ref{sec:Matrix-inversion-example}), and full matrix element retrieval is not necessary. The former is therefore a full quantum state retrieval or quantum tomography, whilst the latter represents a reconstruction of the MF observables which is more akin to classical tomographic methods, albeit with some phase information retained. %\cite{gregory2021MolecularFramePhotoelectron}. %[TBD-\textbf{This is a complicated discussion which we can have, or just cite}~\cite{gregory2021MolecularFramePhotoelectron}.\textbf{It's in there, and I'm not sure it will add much to this review.}]. UPDATE: now partly addressed in main sect.
However, the matrix reconstruction method does not require time-consuming data fitting, and should also scale more readily to larger systems (with some caveats), so should be advantageous in problems where only the MFPADs are sought.

Additionally, matrix element retrieval from MF observables is briefly addressed, the protocol is outlined in Fig. \ref{671760}, and discussed in Sect. \ref{sec:recon-from-MFPADs}. This protocol is essentially identical to the level 2 bootstrapping case, but with different geometric parameters and input dataset. This provides a route to quantum state reconstruction from direct MF measurements, \textit{or reconstructed MF observables}, hence provides a protocol which can be used to extend the matrix inversion method if desired, albeit with associated information content restrictions.

Finally, it is of note that the MF matrix element retrieval protocol of Fig. \ref{671760} is rather generic, and also forms the basis for other similar methods, e.g. cases where LF measurements are obtained as a function of polarization geometry. The difference in general is the exact form and information content of the input dataset, and the geometric parameters required for the given case (model).

% The PEMtk python package [REF] implements both routines [TODO], and full computational notebooks and numerics for these examples are available online [REFS].

The PEMtk python package \cite{hockett2021PEMtkDocs, hockett2021PEMtkGithub} currently implements the level 2 bootstrapping routines (Fig. \ref{807606}), and MF retrieval (Fig. \ref{671760}), and full computational notebooks, including source data and numerics, for these examples are available online \cite{hockett2022MFreconFigshare}. The matrix inversion technique will be implemented soon, examples shown herein are reproduced from ref. \cite{gregory2021MolecularFramePhotoelectron}. It is hoped that interested readers will make use of these materials, and that this presentation will encourage other readers to try (and build upon) what has - up until quite recently - been a rather challenging and involved numerical analysis task. A full list of resources is given in Sect. \ref{sec:resources}, and additional numerical notes in Sect. \ref{sec:numerical-notes}.