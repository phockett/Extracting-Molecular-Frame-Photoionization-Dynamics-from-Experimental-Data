\section{Reconstruction examples \& recent developments in time-domain measurements}

In this section/the remainder of the manuscript, reconstruction of MFPADs from time-domain measurements are considered using two methodologies:

\begin{enumerate}
\item ``Bootstrapping" to the MF \cite{hockett2018QMP1,hockett2018QMP2,marceau2017MolecularFrameReconstruction}
\item Matrix reconstruction with Moore-Penrose inversion \cite{gregory2021MolecularFramePhotoelectron}
\end{enumerate}

The techniques are closely related, and both make use of rotational wavepackets (geometrical coherences) to mediate the LF/AF information content of a set of measurements in the time-domain, but differ in the "directness" of the reconstruction: in the former case, the aim is full matrix element retrieval, and the MFPADs can then be computed from these "complete" results; in the latter the MFPADs are determined, essentially, via a transformation matrix, and full matrix element retrieval is not necessary. The former is therefore a full quantum state retrieval or tomography, whilst the latter represents a partial recovery of the MF observables.

The PEMtk python package [REF] implements both routines [TODO], and full computational notebooks for these examples are available online [REFS].