\section{Reconstruction examples \& recent developments in time-domain measurements}

In this section/the remainder of the manuscript, reconstruction of MFPADs from time-domain measurements are considered using two methodologies:

\begin{enumerate}
\item ``Bootstrapping" to the MF \cite{hockett2018QMP1,hockett2018QMP2,marceau2017MolecularFrameReconstruction}
\item Matrix reconstruction with Moore-Penrose inversion \cite{gregory2021MolecularFramePhotoelectron}
\end{enumerate}

The techniques are closely related, and both make use of rotational wavepackets (geometrical coherences) to mediate the LF/AF information content of a set of measurements in the time-domain, but differ in the ``directness" of the reconstruction: in the former case, the aim is full matrix element retrieval, and the MFPADs can then be computed from these ``complete" results; in the latter the MFPADs are determined, essentially, via a transformation matrix, and full matrix element retrieval is not necessary. The former is therefore a full quantum state retrieval or quantum tomography, whilst the latter represents a reconstruction of the MF observables which is more akin to classical tomographic methods, albeit with some phase information retained [TBD]. However, the matrix reconstruction method does not require time-consuming data fitting, and should also scale more readily to larger systems (with some caveats), so should be advantageous in problems where only the MFPADs are sought.

% The PEMtk python package [REF] implements both routines [TODO], and full computational notebooks and numerics for these examples are available online [REFS].

The PEMtk python package [REF] currently implements the bootstrapping routines, and full computational notebooks, including source data and numerics, for these examples are available online [REFS]. The matrix inversion technique will be implemented soon, examples shown herein are reproduced from ref. \cite{gregory2021MolecularFramePhotoelectron}. It is hoped that interested readers will make use of these materials, and that this presentation will encourage other readers to try (and build upon) what has - up until quite recently - been a rather challenging and involved numerical analysis task.