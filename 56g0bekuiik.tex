\section{Appendix B - Further formalism\label{sec:Appendix-B}}

A complete accounting of the formalism used herein is given in the following sections, expanding on the brief introduction of Sect. \ref{sec:tensor-formulation}, along with some additional notes for interested readers.

\subsection{Full Photoionization Formalism \label{appendix:formalism}}

% [NOTE: text here \href{https://epsproc.readthedocs.io/en/dev/methods/ePSproc_geom_methods_summary_190821-v1-tidy.html}{from main ePSproc docs} via Pandoc, needs some editing.]

The equations for the \(\beta_{LM}\) parameters, in the molecular and
lab frames (MF \& AF respectively), written in terms of geometric
tensor parameters, are given in Sect. \ref{sec:full-tensor-expansion}. The necessary tensor components are further detailed below. For further discussion and derivations see Refs. \cite{Reid2000,Underwood2000,Stolow2008}, and Ref. \cite{hockett2018QMP1} for a summary including various different formalisms; further examples can also be found in the ePSproc documentation \cite{ePSprocDocs}. For general discussion on tensor methods in atomic and molecular spectroscopy and related problems, see Refs. \cite{BlumDensityMat,zareAngMom}. 

Note, also, that several equivalent formalisms have been presented in the literature (Sect. \ref{sec:theory-lit}), often specialised to a particular problem or choice of phase conventions. In the following the equations are general, although the phase conventions are chosen to match those used in ePolyScat (Sect. \ref{sec:mat-ele-conventions}), hence the numerics for the analysis presented herein. In practice the ePSproc codes allow the user to set these conventions as desired. %[More on phase conventions - choice of harmonics, symmetrization etc...? Refs?]

% For the MF, denoted \(\beta_{LM}\)
% (\href{https://epsproc.readthedocs.io/en/latest/methods/geometric_method_dev_pt2_170320_v140420.html}{full
% development notes here}):

% \begin{eqnarray}
% \beta_{L,-M}^{\mu_{i},\mu_{f}}(E) & = & (-1)^{M}\sum_{P,R',R}(2P+1)^{\frac{1}{2}}{E_{P-R}(\hat{e};\mu_{0})}\sum_{l,m,\mu}\sum_{l',m',\mu'}(-1)^{(\mu'-\mu_{0})}{\Lambda_{R',R}(R_{\hat{n}};\mu,P,R,R')B_{L,-M}(l,l',m,m')}I_{l,m,\mu}^{p_{i}\mu_{i},p_{f}\mu_{f}}(E)I_{l',m',\mu'}^{p_{i}\mu_{i},p_{f}\mu_{f}*}(E)
% \end{eqnarray}

% For the AF, denoted \(\bar{\beta}_{LM}\)
% (\href{https://epsproc.readthedocs.io/en/latest/methods/geometric_method_dev_pt3_AFBLM_090620_010920_dev_bk100920.html}{full
% development notes here}):

% \begin{eqnarray}
% \bar{\beta}_{L,-M}^{\mu_{i},\mu_{f}}(E,t) & =(-1)^{M} & \sum_{P,R',R}{[P]^{\frac{1}{2}}}{E_{P-R}(\hat{e};\mu_{0})}\sum_{l,m,\mu}\sum_{l',m',\mu'}(-1)^{(\mu'-\mu_{0})}{\Lambda_{R'}(\mu,P,R')B_{L,S-R'}(l,l',m,m')}I_{l,m,\mu}^{p_{i}\mu_{i},p_{f}\mu_{f}}(E)I_{l',m',\mu'}^{p_{i}\mu_{i},p_{f}\mu_{f}*}(E)\sum_{K,Q,S}\Delta_{L,M}(K,Q,S)A_{Q,S}^{K}(t)\label{eq:BLM-tidy-prod-2}
% \end{eqnarray}

% Where \(I_{l,m,\mu}^{p_{i}\mu_{i},p_{f}\mu_{f}}(E)\) are the (radial)
% dipole ionization matrix elements, as a function of energy \(E\),
% obtained from an
% \href{https://epsproc.readthedocs.io/en/latest/ePS_ePSproc_tutorial/ePS_tutorial_080520.html\#Theoretical-background}{ePolyScat
% (or other) calculation}, defined by a set of partial-waves \(\{l,m\}\),
% for polarizations \(\mu\) and channels (symmetries) labelled by initial
% and final state indexes \({p_{i}\mu_{i},p_{f}\mu_{f}}\).

% In both cases a set of geometric tensor terms are required, defined
% below. Note that, in this case, time-dependence arises purely from the
% \(A_{Q,S}^{K}(t)\) terms in the AF case, and the electric field term
% currently describes only the photon angular momentum coupling,
% time-dependent/shaped fields are not yet supported (as of v1.3.0, but
% will be soon). Similarly, a time-dependent initial state
% (e.g.~vibrational wavepacket) could also describe a time-dependent MF
% case, but is currently not included here.

\subsubsection{Electric field term}\label{electric-field-term}

The
\href{https://epsproc.readthedocs.io/en/latest/methods/geometric_method_dev_260220_090420_tidy.html\#E_\%7BP,R\%7D-tensor}{coupling
of two 1-photon terms can be written as a tensor contraction} \cite{BlumDensityMat,zareAngMom}:

\begin{equation}
E_{P,R}(\hat{e})=[e\otimes e^{*}]_{R}^{P}=[P]^{\frac{1}{2}}\sum_{p}(-1)^{R}\left(\begin{array}{ccc}
1 & 1 & P\\
p & R-p & -R
\end{array}\right)e_{p}e_{R-p}^{*}
\label{eq:EPR-defn-1}
\end{equation}

Where \(e_{p}\) and \(e_{R-p}\) define the field strengths for the
polarizations \(p\) and \(R-p\), which are coupled into the spherical
tensor \(E_{PR}\). For the simplest case of a linearly-polarized field, $p=0$, and only terms $P=0,2$ with $R=0$ (i.e. $E_{0,0}$ and $E_{2,0}$) are non-zero.

Note this notation implicitly describes only the time-independent photon angular momentum coupling,
but time-dependent/shaped laser fields can be readily incorporated by allowing for time-dependent fields $e_{p}(t)$ (see, for instance, Ref. \cite{hockett2015CompletePhotoionizationExperiments}).
% are not yet supported (as of v1.3.0, but will be soon).

\subsubsection{\texorpdfstring{\(B_{L,M}\)
term}{B\_\{L,M\} term}}\label{b_lm-term}

The coupling of the partial wave pairs, \(|l,m\rangle\) and
\(|l',m'\rangle\), into the observable set of \(\{L,M\}\) is
\href{https://epsproc.readthedocs.io/en/latest/methods/geometric_method_dev_260220_090420_tidy.html\#B_\%7BL,M\%7D-term}{defined
by a tensor contraction with two 3j terms} \cite{zareAngMom}.

\begin{equation}
B_{L,M}=(-1)^{m}\left(\frac{(2l+1)(2l'+1)(2L+1)}{4\pi}\right)^{1/2}\left(\begin{array}{ccc}
l & l' & L\\
0 & 0 & 0
\end{array}\right)\left(\begin{array}{ccc}
l & l' & L\\
-m & m' & M
\end{array}\right)
\label{eq:BLM-func-defn}
\end{equation}

Note for the AF case the terms may be reindexed by $M=S-R'$ (see below). This allows for all MF projections to contribute, rather than a single specified polarization geometry.

\subsubsection{\texorpdfstring{\(\Lambda\)
Term}{\textbackslash{}Lambda term}}\label{lambda-term}

A general geometric projection term can be defined in the MF and AF.
For the \href{https://epsproc.readthedocs.io/en/latest/methods/geometric_method_dev_260220_090420_tidy.html\#/Lambda-Term}{MF
projection term}, \(\Lambda_{R',R}(R_{\hat{n}})\):

\begin{equation}
\Lambda_{R',R}(R_{\hat{n}})=(-1)^{(R')}\left(\begin{array}{ccc}
1 & 1 & P\\
\mu & -\mu' & R'
\end{array}\right)D_{-R',-R}^{P}(R_{\hat{n}})
\label{eq:lambda-func-defn-MF}
\end{equation}

This is similar to the $E_{P,R}$ term, and essentially rotates the field defined in the LF into
the MF by a set of rotations (Euler angles) defined by $R_{\hat{n}}=\{\chi,\Theta,\Phi\}$.

For
\href{https://epsproc.readthedocs.io/en/latest/methods/geometric_method_dev_pt3_AFBLM_090620_010920_dev_bk100920.html\#/beta_\%7BL,M\%7D\%5E\%7BAF\%7D-rewrite}{the
AF case}, a simplified form can be used, since there is no single
orientation/rotation defined in relation to the MF, and the relations are defined by the
molecular axis distribution (see below). 

\begin{equation}
\bar{\Lambda}_{R'}=(-1)^{(R')}\left(\begin{array}{ccc}
1 & 1 & P\\
\mu & -\mu' & R'
\end{array}\right)\equiv\Lambda_{R',R'}(R_{\hat{n}}=0)
\label{eq:lambda-func-defn-AF}
\end{equation}

The notation here implies that the electric field and axis distributions are expanded about the same axis ($R_{\hat{n}}=0$), which corresponds to the simplest parallel align-probe field geometry. Additional frame rotation(s) may be applied in some cases (e.g. for crossed-polarization of the alignment and probe fields).

\subsubsection{Alignment term\label{alignment-term}}

The axis distribution moments (ADMs) define the LF in this case, and are given above as a set of parameters $A_{Q,S}^{K}(t)$. These give rise to couplings which can be written as:

\begin{equation}
\Delta_{L,M}(K,Q,S)=(2K+1)^{1/2}(-1)^{K+Q}\left(\begin{array}{ccc}
P & K & L\\
R & -Q & -M
\end{array}\right)\left(\begin{array}{ccc}
P & K & L\\
R' & -S & S-R'
\end{array}\right)
\label{eq:delta-func-defn}
\end{equation}

Hence the coupling between the LF and MF is, effectively, defined by the final term in the AF observable (Eqn. \ref{eq:BLM-tensor-AF}):

\begin{equation}
\sum_{K,Q,S}\Delta_{L,M}(K,Q,S)A_{Q,S}^{K}(t)
\end{equation}

And the observable is restricted to $L_{max}=|P+K|$, and $M=0$ only if  $Q=S=R=0$ (the usual cylindrically symmetric case).
% [TODO: still need to test/verify this for some other cases in the current codes - only used for Q=S=0 cases so far]

\subsubsection{Dipole matrix elements and conventions\label{sec:mat-ele-conventions}}

% TODO - wavefunction conventions, symmetrized harmonics.

Herein the numerical form of the dipole matrix elements is chosen to match the definitions used by ePolyScat \cite{Gianturco1994,Natalense1999,Toffoli2007}. In the notation of Ref. \cite{Toffoli2007}:

\begin{equation}
I_{l,m,\mu}^{p_{i}\mu_{i},p_{f}\mu_{f}}(\epsilon)=\langle\Psi_{i}^{p_{i},\mu_{i}}|\hat{d_{\mu}}|\Psi_{f}^{p_{f},\mu_{f}}\varphi_{klm}^{(-)}\rangle\label{eq:eps-I}
\end{equation}

\begin{equation}
T_{\mu_{0}}^{p_{i}\mu_{i},p_{f}\mu_{f}}(\theta_{\hat{k}},\phi_{\hat{k}},\theta_{\hat{n}},\phi_{\hat{n}})=\sum_{l,m,\mu}I_{l,m,\mu}^{p_{i}\mu_{i},p_{f}\mu_{f}}(\epsilon)Y_{lm}^{*}(\theta_{\hat{k}},\phi_{\hat{k}})D_{\mu,\mu_{0}}^{1}(R_{\hat{n}})\label{eq:eps-TMF}
\end{equation}

\begin{equation}
I_{\mu_{0}}(\theta_{\hat{k}},\phi_{\hat{k}},\theta_{\hat{n}},\phi_{\hat{n}})=\frac{4\pi^{2}\epsilon}{cg_{p_{i}}}\sum_{\mu_{i},\mu_{f}}|T_{\mu_{0}}^{p_{i}\mu_{i},p_{f}\mu_{f}}(\theta_{\hat{k}},\phi_{\hat{k}},\theta_{\hat{n}},\phi_{\hat{n}})|^{2}\label{eq:eps-MFPAD}
\end{equation}

In this formalism:
\begin{itemize}
\item $I_{l,m,\mu}^{p_{i}\mu_{i},p_{f}\mu_{f}}(\epsilon)$ is the radial part of the dipole matrix element (as per Eqns. \ref{eq:r-kllam}, \ref{eqn:I-zeta} herein), determined from the initial and final state electronic wavefunctions $\Psi_{i}^{p_{i},\mu_{i}}$and $\Psi_{f}^{p_{f},\mu_{f}}$,
and the radial part of the photoelectron wavefunction $\varphi_{klm}^{(-)}$ (cf. $\phi_{lm}(k)$ in Eqn. \ref{eq:r-kllam}, and where $^{(-)}$ denotes outgoing wave normalisation) and dipole operator $\hat{d_{\mu}}$. Here the wavefunctions are indexed by irreducible representation (i.e. symmetry) by the labels $p_{i}$ and $p_{f}$, with components $\mu_{i}$ and $\mu_{f}$ respectively; $l,m$ are angular momentum components, $\mu$ is the projection of the polarization into the MF. Each energy and irreducible representation corresponds to a calculation in ePolyScat.
\item $T_{\mu_{0}}^{p_{i}\mu_{i},p_{f}\mu_{f}}(\theta_{\hat{k}},\phi_{\hat{k}},\theta_{\hat{n}},\phi_{\hat{n}})$
is the full matrix element (expanded in polar coordinates) in the
MF, where $\hat{k}$ denotes the direction of the photoelectron $\mathbf{k}$-vector, and $\hat{n}$ the direction of the polarization vector $\mathbf{n}$ of the ionizing light. This is equivalent to the full photoelectron wavefunction denoted $\Psi_e(\mathbf{k})$ in Sect. \ref{sec:dynamics-intro}.
\item $Y_{lm}^{*}(\theta_{\hat{k}},\phi_{\hat{k}})$ is a spherical harmonic. Note the conjugate form here.
\item $D_{\mu,-\mu_{0}}^{1}(R_{\hat{n}})$ is a Wigner rotation matrix element, with a set of Euler angles $R_{\hat{n}}=(\phi_{\hat{n}},\theta_{\hat{n}},\chi_{\hat{n}})$, which rotates/projects the polarization into the MF .
\item $I_{\mu_{0}}(\theta_{\hat{k}},\phi_{\hat{k}},\theta_{\hat{n}},\phi_{\hat{n}})$ is the final (observable) MFPAD, for a polarization $\mu_{0}$ and summed over all symmetry components of the initial and final states, $\mu_{i}$ and $\mu_{f}$. Note that this sum can be expressed as an incoherent summation, since these components are (by definition) orthogonal.
\item $g_{p_{i}}$ is the degeneracy of the state $p_{i}$.
\end{itemize}

As noted previously, in general there are multiple equivalent definitions for these terms used by different authors; the ePolyScat definitions above apply to the test matrix elements used herein (e.g. Table \ref{tab:inputMatE}), which lists $I_{l,m,\mu}^{p_{i}\mu_{i},p_{f}\mu_{f}}(\epsilon)$ for continuum symmetry components $p_f=\sigma_u,\pi_u$. These matrix elements are normalised to the total cross-section (in Mb), and include symmetrization. 

% In the ePSproc codebase, various switches can also be set to define alternative choices, e.g. conjugate forms, use of real harmonics etc., for computation of observables and matrix element retrieval. By default the following conventions/libraries are used:

% \begin{itemize}
% \item Angular momentum functions (Wigner D and 3js) are currently implemented directly, or via the Spherical Functions library \cite{boyle2022SphericalFunctions}, and have been tested for consistency with the definitions in Zare (for details see \href{https://epsproc.readthedocs.io/en/latest/tests/Spherical_function_testing_Aug_2019.html}{the ePSproc docs} \cite{hockett2020EPSprocPostprocessingEPolyScat}).
% \item Spherical harmonics are defined with the usual physics conventions: orthonormalised, and including the Condon-Shortley phase. Numerically they are implemented directly or via SciPy's \verb+sph_harm+ function (see \href{https://docs.scipy.org/doc/scipy/reference/generated/scipy.special.sph_harm.html}{the SciPy docs for details} \cite{SciPyDocumentation}. Further manipulation and conversion between different normalisations can be readily implemented with the SHtools library \cite{wieczorek2018SHToolsToolsWorking,SHtoolsGithub}.
% \item General tensor 
% \end{itemize}

For cases where symmetrization is not included in the matrix elements directly, it can be addressed via the use of symmetrized (or generalised) harmonics, which essentially provide correctly symmetrized expansions of spherical harmonics for a given irreducible representation, $\Gamma$. These can be defined by linear combinations of spherical harmonics (see Refs.\cite{Altmann1963a,Altmann1965,Chandra1987} for more):

\begin{equation}
X_{hl}^{\Gamma\mu*}(\theta,\phi)=\sum_{\lambda}b_{hl\lambda}^{\Gamma\mu}Y_{l,\lambda}(\theta,\phi)\label{eq:symm-harmonics}
\end{equation}


where: 

\begin{itemize}
\item $\Gamma$ is an irreducible representation, 
\item ($l$, $\lambda$) define the usual spherical harmonic indicies (rank, order)
\item $b_{hl\lambda}^{\Gamma\mu}$ are symmetrization coefficients, 
\item index $\mu$ allows for indexing of degenerate components,
\item $h$ indexs cases where multiple components are required with all other quantum numbers identical. 
\end{itemize}
    
The exact form of these coefficients will depend on the point-group of the system, see, e.g. Refs. \cite{Chandra1987,Reid1994}; for numerical implementation notes in PEMtk see Sect. \ref{sec:numerical-notes}.
% Computation of $X_{hl}^{\Gamma\mu*}(\theta,\phi)$ is currently implemented in the PEMtk codebase, making use of libmsym \cite{johansson2017AutomaticProcedureGeneratinga,johansson2022LibmsymGithub} (symmetry coefficients) and SHtools \cite{wieczorek2018SHToolsToolsWorking,SHtoolsGithub} (general spherical harmonic handling and conversion). For worked examples, see \href{https://pemtk.readthedocs.io/en/latest/sym/pemtk_symHarm_demo_160322_tidy.html}{the PEMtk docs} \cite{hockett2021PEMtkDocs}.

% \textbf{Refs for the full AF-PAD formalism above:}

% \begin{enumerate}
% \def\labelenumi{\arabic{enumi}.}
% % \tightlist
% \item
%   Reid, Katharine L., and Jonathan G. Underwood. ``Extracting Molecular
%   Axis Alignment from Photoelectron Angular Distributions.'' The Journal
%   of Chemical Physics 112, no. 8 (2000): 3643.
%   https://doi.org/10.1063/1.480517.
% \item
%   Underwood, Jonathan G., and Katharine L. Reid. ``Time-Resolved
%   Photoelectron Angular Distributions as a Probe of Intramolecular
%   Dynamics: Connecting the Molecular Frame and the Laboratory Frame.''
%   The Journal of Chemical Physics 113, no. 3 (2000): 1067.
%   https://doi.org/10.1063/1.481918.
% \item
%   Stolow, Albert, and Jonathan G. Underwood. ``Time-Resolved
%   Photoelectron Spectroscopy of Non-Adiabatic Dynamics in Polyatomic
%   Molecules.'' In Advances in Chemical Physics, edited by Stuart A.
%   Rice, 139:497--584. Advances in Chemical Physics. Hoboken, NJ, USA:
%   John Wiley \& Sons, Inc., 2008.
%   https://doi.org/10.1002/9780470259498.ch6.
% \end{enumerate}

% Where {[}3{]} has the version as per the full form above (full
% asymmetric top alignment distribution expansion).